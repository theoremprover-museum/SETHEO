%%%%%%%%%%%%%%%%%%%%%%%%%%%%%%%%%%%%%%%%%%%%%%%%%%%
%   SETHEO MANUAL
%	(c) J. Schumann, O. Ibens
%	TU Muenchen 1995
%
%	%W% %G%
%%%%%%%%%%%%%%%%%%%%%%%%%%%%%%%%%%%%%%%%%%%%%%%%%%%
\section{Installing the Binaries}\label{sec:inst-bin}

Now you have a local file {\tt setheo.solaris.tar.gz} (respectively
{\tt setheo.hp.tar.gz}, {\tt setheo.sunos.tar.gz} or {\tt
setheo.linux.tar.gz}). This is a compressed package containing
directories and files of the SETHEO system. Unpack the system with:
\begin{verbatim}
        gunzip -c setheo.tar.gz
        tar xvf setheo.solaris.tar
\end{verbatim}
You can delete the file {\tt setheo.solaris.tar}, if you want. Go into
the directory {\tt setheo.solaris} and call the installation process: 
\begin{verbatim}
        cd setheo.solaris
        make
\end{verbatim}
If you want to invoke SETHEO from any directory, you must add this
directory to your path. Now \SE\ is ready for use.
