%%%%%%%%%%%%%%%%%%%%%%%%%%%%%%%%%%%%%%%%%%%%%%%%%%%
%   SETHEO MANUAL
%	(c) J. Schumann, O. Ibens
%	TU Muenchen 1995
%
%	%W% %G%
%%%%%%%%%%%%%%%%%%%%%%%%%%%%%%%%%%%%%%%%%%%%%%%%%%%
\section{Getting the Binaries}\label{sec:get-bin}

The SETHEO system can be obtained via anonymous ftp. The ftp--server
provides a package containing the binaries of the basic SETHEO
programs for specific platforms (see Figure~\ref{fig:platforms}), the
manpages, example problems and a file with installation
hints. Binaries for other platforms will be available in the future. 

\begin{figure}[htb]

\begin{center}
\begin{tabular}{|l|l|}
\hline
System  &  Name of Package  \\ \hline
Sun Os 4.1.3  &  setheo.sunos.tar.gz  \\
Sun Solaris 2.4 (or up)  &  setheo.solaris.tar.gz  \\
HP HPUX  &  setheo.hp.tar.gz  \\
Linux  &  setheo.linux.tar.gz  \\ \hline
\end{tabular}
\end{center}

\caption{Platforms for which binaries of the SETHEO system are
         provided.}\label{fig:platforms} 
\end{figure}

To get this package you have to connect the ftp--server either by
\begin{verbatim}
        ftp ftp.informatik.tu-muenchen.de
\end{verbatim}
or by
\begin{verbatim} 
        ftp 131.159.0.198
\end{verbatim}
If the connection is established, the ftp--server will ask for your
name and your e--mail address:
\begin{verbatim}
        Name: anonymous
        Password: <your-e-mail-address>
\end{verbatim}
The next thing you have to do is change to the right directory, set
the transfer mode to {\tt binary} and get the {\tt
setheo.solaris.tar.gz} file (respectively {\tt setheo.hp.tar.gz} or
{\tt setheo.sunos.tar.gz}, or {\tt setheo.linux.tar.gz}): 
\begin{verbatim}
        ftp> cd /local/lehrstuhl/jessen/Automated_Reasoning/SETHEO
        ftp> binary
        ftp> get setheo.solaris.tar.gz
\end{verbatim}
You can get a short version of this description as well. After this
leave the ftp--server: 
\begin{verbatim}
        ftp> get README
        ftp> bye
\end{verbatim}



