%%%%%%%%%%%%%%%%%%%%%%%%%%%%%%%%%%%%%%%%%%%%%%%%%%%
%   SETHEO MANUAL
%	(c) J. Schumann, O. Ibens
%	TU Muenchen 1995
%
%	%W% %G%
%%%%%%%%%%%%%%%%%%%%%%%%%%%%%%%%%%%%%%%%%%%%%%%%%%%

\section{\SAM\ Error Messages}\label{sec:errors}

There are different types of errors which might occur during the \SAM's
runtime. The minor ones only cause a message printed on the
screen. The heavy errors additionally lead to a system interrupt. If
necessary, the hexadecimal code of the trouble causing object is
displayed, too. 
The problems can be divided into three classes: 
\begin{enumerate}
\item{user dependend problems}
\item{internal problems}
\item{implementation problems}
\end{enumerate}

{\bf User dependend problems} are either caused by the user or can
easily be avoided by the user. {\it Wrong parameter choices\/} or {\it
overflow errors\/} belong to this class. Examples for {\bf wrong
parameter choices} to the \SAM\ are: 
\begin{itemize}
\item{The name of the code file together with extension {\bf .hex}
      does not exist.
      {\it What you can do:\/} Check if you spelled the name correctly
      and if you left out the extension.}
\item{The name of the code file is missing.
      {\it What you can do:\/} Add the code file.}
\item{An option was requested which is not available in your current
      version of SETHEO.
      {\it What you can do:\/} If you need this option, this becomes a
      problem for a programmer because you have to change some
      compiler flags and compile the \SAM\ again. If possible, just
      don't request this option.}
\item{You have made a typing mistake concerning options or parameters
      to options.
      {\it What you can do:\/} Check the spelling of what you typed and
      maybe look at Section~\ref{sec:sam} again.}
\end{itemize}
Some of the built--ins also require special kinds of parameters, e.g.\
the arithmetic built--ins expect number values.
{\it What you can do:\/} You have to repair this kind of wrong
parameter choices within the {\bf .lop}--file. The error message tells
you what to change. Don't forget to call the \inw\ and the \wasm\ again
before calling the \SAM.

{\bf Overflow errors} are not caused by the user but they are
avoidable by the 
user. An overflow error occurs if one of the \SAM's memory areas is
too small. Overflow errors always cause the \SAM\ to interrupt. You
can still get statistical information if you are interested in, but
there is no way to go on proving. 
{\it What you can do:\/} Quit the program and start it again with the
overflowed memory area enlarged. Eventually read Section~\ref{sec:sam}
how to enlarge memory areas.

{\bf Internal problems} are memory allocation problems or limit
exceeding problems. {\bf Memory allocation problems} can be the
following: 
\begin{itemize}
\item{The operating system does not provide enough space to allocate
      the \SAM's memory areas or the address of one of the memory
      areas is out of range. {\it What you can do:\/} Reduce the 
      size of the memory areas. Use the same options as for enlarging
      the memory areas (see Section~\ref{sec:sam}).}
\item{The operating system does not provide enough space to allocate
      memory for storing antilemmata or the address of the allocated
      memory is out of range. {\it What you can do:\/} Reduce
      the number of generated antilemmata. Call the {\bf -anl
      [number]} option with {\it number\/} enlarged. The default value
      for {\it number\/} is 25 (see Section~\ref{sec:sam}).}
\item{The operating system does not provide enough space to open an
      output file. {\it What you can do:\/} Reduce the size occupied
      by the \SAM, i.e.\ reduce the size of the memory areas (see above).}
\end{itemize}
The following problems are {\bf limit exceeding problems}:
\begin{itemize}
\item{A number value gets bigger than 32767 or less than --32768 and is
      truncated to a short value.
      (Remember the \SAM\ works only with short arithmetic). {\it What
      you can do:\/} Try to avoid refering to this value.}
\item{A choicepoint contains too many alternatives for the random
      reordering instruction. {\it What you can do:\/} If you want to
      repair this problem you have to change a constant within the
      \SAM's source code, so this becomes an implementation problem.}
\end{itemize}

{\bf Implementation problems} arise from wrong or incomplete
changes to the SETHEO system. Messages about illegal tags, illegal
instructions or unavailable bounds belong to this kind. 
{\it What you can do:} Errors concerning illegal tags are hard to
debug. So think very carefully what you are changing. Errors
concerning illegal instructions might result from spelling mistakes or
from not having made the corresponding changes in \inw\ and \wasm. Errors
concerning unavailable bounds should be detectable using a usual
debugging tool.
