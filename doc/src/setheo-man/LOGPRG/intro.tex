%%%%%%%%%%%%%%%%%%%%%%%%%%%%%%%%%%%%%%%%%%%%%%%%%%%
%   SETHEO MANUAL
%	(c) J. Schumann, O. Ibens
%	TU Muenchen 1995
%
%	%W% %G%
%	16.12.96
%%%%%%%%%%%%%%%%%%%%%%%%%%%%%%%%%%%%%%%%%%%%%%%%%%%

The basic Model Elimination Calculus and SETHEO's way of
searching for a proof is quite similar to the way, a PROLOG program
is executed. 
Therefore, we have added a variety of techniques to SETHEO which allows
SETHEO to be used for logic-programming purposes as well as for
pure automated theorem proving. In particular, these logic programming
facilities allow to control the behavior of SETHEO.
Thus, e.g., different search strategies, generation and usage of
lemmata can be implemented easily (in a prototypical way). 

In the current version of SETHEO, two major extensions are provided:
global, backtrackable variables (see following Section~\ref{sec:globvar}, 
and PROLOG-style built-in predicates.

Syntactically, a built-in predicate is an {\tt ATOM} in the LOP-language,
corresponding to the syntax definition in Section~\ref{sec:lop-syntax}.
Except for built-ins for arithmetic, all built-in predicates
are of the form $\$p$ or $\$p(t_1,\ldots,t_n)$.
The name of the predicate symbol always starts with a $\$ $ sign in order
to be able to distinguish syntactically between ordinary predicate symbols
and built-ins. Such a built-in must alway occur in the {\em tail\/}
of a clause.
The complete list of built-ins is given in the following sections of
this Chapter. For each built-in, we give its name, synopsis, a description,
and one or more example demonstrating the usage of the built-in.

When using built-in predicates, several items must be taken
into account:
\begin{itemize}
\item
when using built-ins, SETHEO can become both unsound and incomplete.
\item
built-in predicates can have side-effects. Therefore, the order in which
these predicates are executed are important. Since, per default,
SETHEO reorders the subgoals of the clauses, it is advisable to turn
off subgoal reordering ({\tt -nosgreord} flag of {\bf inwasm}), or to
use procedural clauses (with a {\tt :-} as separator.

\item
Although all built-ins can be used in combination with non-Horn clauses
as well, no clear semantics can be defined in that case.

\item
The list, described in this manual, describes all predicates, available
with SETHEO Version V3.3. Future versions will probably extend this list.
\item
When using built-ins and global variables, it might be advisable to turn
off the search-space reduction techniques, since these might disable
the execution of clauses with procedural side-effects only.
\end{itemize}


