%%%%%%%%%%%%%%%%%%%%%%%%%%%%%%%%%%%%%%%%%%%%%%%%%%%
%   SETHEO MANUAL
%	(c) J. Schumann, O. Ibens
%	TU Muenchen 1995
%
%	%W% %G%
%
% BUILT-IN:	
% SCCS:		%W% %G%
% AUTHOR:	J. Schumann
%
%%%%%%%%%%%%%%%%%%%%%%%%%%%%%%%%%%%%%%%%%%%%%%%%%%%
\subsection{\$size/2}

\begin{description}
\item[Synopsis:]
	{\tt \$size(T,N)}
\item[Parameters:]\ \\[-0.5cm]
	\begin{description}
	\item[{\tt T}] term
	\item[{\tt N}] number or variable
	\end{description}
\item[Low Level Name:]
	{\tt size}
\end{description}

\vspace*{0.5cm}
\noindent
{\bf Description.}
This predicate calculates the size of the term {\tt T} and
unifies the result with the second parameter {\tt N}.
The size of a term is the number of its constants, variables, and
function symbols.

\vspace*{0.5cm}
\noindent
{\bf Example.}
\begin{verbatim}
p <- $size(f(a,X),2)
p <- $size(f(a,g(Y)),2)  /* will fail */
\end{verbatim}


