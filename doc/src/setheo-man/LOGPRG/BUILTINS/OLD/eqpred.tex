%%%%%%%%%%%%%%%%%%%%%%%%%%%%%%%%%%%%%%%%%%%%%%%%%%%
%   SETHEO MANUAL
%	(c) J. Schumann, O. Ibens
%	TU Muenchen 1995
%
%	%W% %G%
%
% BUILT-IN:	
% SCCS:		%W% %G%
% AUTHOR:	J. Schumann
%
%%%%%%%%%%%%%%%%%%%%%%%%%%%%%%%%%%%%%%%%%%%%%%%%%%%
\subsection{eqpred/1}

\begin{description}
\item[Synopsis:]
	{\tt eqpred(N)}
\item[Parameters:]\ \\[-0.5cm]
	\begin{description}
	\item[{\tt N}]
	N: number of literal to check. {\tt N} must be in the range
        between 1 and the length of the clause.
	\end{description}
\item[Low Level Name:]
	{\tt eqpred/4}
\item[Result:]\ \\
A built-in of this form fails, whenever the
            n-th literal in the current clause is preceeded by an
            identical one (syntactically equal) in the current tableau.
\end{description}

\vspace*{0.5cm}
\noindent
{\bf Description.}

\vspace*{0.5cm}
\noindent
{\bf Side-effects.}

\vspace*{0.5cm}
\noindent
{\bf Notes.}
This built-in can be used to implement a weak form of
regularity checks.

\vspace*{0.5cm}
\noindent
{\bf Example.}
\begin{verbatim}
L1:- eqpred(1),     /* eq pred check for the head literal L1.   */
     L3, eqpred(3), /* eq pred check for the literal ~L3.       */
     L5, eqpred(5), /* eq pred check for the literal ~L5.       */
              :
              :

\end{verbatim}


