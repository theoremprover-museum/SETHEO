-------------------------------------
Assignment and destructive assignment 
-------------------------------------

\subsection{:=}

{\tt \$V := T}

\$V is the name of a global variable.

T is an arbitrary term which is gets (destructively) assigned
to the global variable.


Destructive term assignment to global variables. 
%-------------------------------------------
The term {\tt T} is not copied during the assignment.
This means that later substitutions of variables present in {\tt T}
also alter the contents of the global variable.
Copying of the entire term can be accomplished using 
{\bf functor/3}.
%-------------------------------------------
This statement always succeeds.
%-------------------------------------------
\begin{verbatim}
p <- $G := [p_entered | $G],..
\end{verbatim}

Here, we enter a symbol to a list which is kept in the global
variable {\tt \$G}.

%%%%%%%%%%%%%%%%%%%%%%%%%%%%%%%%%%%%%%%%
\subsection{is}

{\tt V is NUMEXPR}

V: (logical) variable
NUMEXPR: term or numerical expression which must evaluate to a number
	when this statement is encountered.

%-------------------------------------------
This statement succeeds, if {\tt V} is an unbound variable or
if {\tt V} is instantiated to a number which is equal to the value
of the numerical expression {\tt NUMEXPR}.
Otherwise a fail occurrs.
If {\tt NUMEXPR} does not evaluate to a number, a non-fatal run-time
error occurrs.
%-------------------------------------------
   Note: In contrast to global variables in some Prolog Systems
   	 like Sepia or Eclipse, destructive assignments to
	 global variables in LOP are backtracked if the proof 
	 search fails at later subgoals.
%-------------------------------------------
\begin{verbatim}
p(X,Y) <- X is Y+1.
\end{verbatim}

%%%%%%%%%%%%%%%%%%%%%%%%%%%%%%%%%%%%%%%%

\subsection{$<,>,\geq$}

{\tt NUMEXPR relop NUMEXPR}

%----------------------------------
%Unification and test for identity
%---------------------------------

\subsection{unify/2}

{\tt unify(T1,T2)}

T1,T2: two terms to be unified.

%-------------------------------------------
This built-in succeeds, if both terms are unifiable.
%-------------------------------------------
As a side-effect, the substiutions of the unification are applied to
the terms {\tt T1, T2}.

%-------------------------------------------
If only a check is required, if two terms are unifiable, {\tt isunifiable/2}
should be used.
%-------------------------------------------
\begin{verbatim}
p(X) <- unify(X,f(g(a,b),c)).
\end{verbatim}

produces the same result as
\begin{verbatim}
p(f(g(a,b),c)<-.
\end{verbatim}

%%%%%%%%%%%%%%%%%%%%%%%%%%%%%%%%%%%%%%%%
\subsection{eq/2, ==}

{\tt eq(T1,T2)}
{\tt T1 == T2}

T1,T2: terms to be checked for syntactical equality.

%-------------------------------------------
This built-in succeeds, if {\tt T1} is syntactical equal to
{\tt T2}.

%-------------------------------------------
This predicate checks two terms for syntactical equality.
Two terms are syntactical equal, if
\begin{itemize}
\item
both are the same symbolic constants (os strings)
\item
both are the {\em same\/} variable,
\item
the terms are of the form
$f^1(t_1^1,\ldots,t_n^1)$ and
$f^2(t_1^2,\ldots,t_n^2)$, and the function symbols $f^1,f^2$ and all
pairs of subterms $t^1_i,t^2_i$ are syntactical equal.
\end{itemize}


%-------------------------------------------
\begin{verbatim}
p(X)<- eq(f(a),X).
p(X)<- eq(f(Y),X).
\end{verbatim}
The first clause succeeds, if $X$ is instantiated to $f(a)$.
The second clause, however, does {\em never\/} succeed, since
$Y$ only occurrs only once in the second clause.

\subsection{neq/2, =/=}

This built-in is complementary to {\bf eq/2}. For details see
Section~\ref{sec:eq/2}


%%%%%%%%%%%%%%%%%%%%%%%%%%%%%%%%%%%%%%%%
\subsection{eqpred/1}

{\tt eqpred(N)}

N: number of literal to check. {\tt N} must be in the range
	between 1 and the length of the clause.

%-------------------------------------------

Equal predecessor check in the current tableau:

A built-in of this form fails, whenever the 
            n-th literal in the current clause is preceeded by an 
            identical one (syntactically equal) in the current tableau. 

This built-in can be used to implement a weak form of
regularity checks.

%-------------------------------------------
\begin{verbatim}
L1:- eqpred(1),     /* eq pred check for the head literal L1.   */
     L3, eqpred(3), /* eq pred check for the literal ~L3.       */
     L5, eqpred(5), /* eq pred check for the literal ~L5.       */
	      :
	      :
\end{verbatim}

%%%%%%%%%%%%%%%%%%%%%%%%%%%%%%%%%%%%%%%%
\subsection{isvar/1}

{\tt isvar(T)}

T: term


%-------------------------------------------
This built-in predicate succeeds, if {\tt T} is
a variable or bound to a variable. 
%-------------------------------------------
Global variables are initially bound to ordinary variables.
Therefore, this test is also suited for global variables.
%-------------------------------------------
\begin{verbatim}
p(X) <- isvar(X),...
\end{verbatim}

%%%%%%%%%%%%%%%%%%%%%%%%%%%%%%%%%%%%%%%%
\subsection{isnonvar/1}

{\tt isnonvar(T)}

T: term


%-------------------------------------------
This built-in predicate succeeds, if {\tt T} is
{\em not\/} a variable or bound to a variable. 
%-------------------------------------------
\begin{verbatim}
p(X) <- isvar(X),...
p(X) <- isnonvar(X),...
\end{verbatim}

%%%%%%%%%%%%%%%%%%%%%%%%%%%%%%%%%%%%%%%%
\subsection{isnumber/1}

{\tt isnumber(T)}

T: term


%-------------------------------------------
This built-in predicate succeeds, if {\tt T} is
bound to a number.
%-------------------------------------------
\begin{verbatim}
p(X) <- isnumber(X),...
\end{verbatim}

%%%%%%%%%%%%%%%%%%%%%%%%%%%%%%%%%%%%%%%%
\subsection{isconst/2}

{\tt isconst(T)}

T: term


%-------------------------------------------
This built-in predicate succeeds, if {\tt T} is
bound to a symbolic constant or a string.
%-------------------------------------------
Natural numbers are no symbolic constants.
%-------------------------------------------
\begin{verbatim}
p(X) <- isconst(X),...
p(X) <- isvar(X),...
\end{verbatim}

%%%%%%%%%%%%%%%%%%%%%%%%%%%%%%%%%%%%%%%%
\subsection{iscompl/1}

{\tt iscompl(T)}

T: term


%-------------------------------------------
This built-in predicate succeeds, if {\tt T} is
bound to a complex term $f(t_1,\ldots,t_n)$, or a list.
%-------------------------------------------
\begin{verbatim}
p(X) <- isconst(X),...
p(X) <- iscompl(X),...
\end{verbatim}

%%%%%%%%%%%%%%%%%%%%%%%%%%%%%%%%%%%%%%%%
\subsection{size/2}

{\tt size(T,N)}

This predicate calculates the size of the term {\tt T} and
unifies the result with the second parameter {\tt N}.
The size of a term is the number of its constants, variables, and
function symbols.
%-------------------------------------------
\begin{verbatim}
p <- size(f(a,X),2)
p <- size(f(a,g(Y)),2)  /* will fail */
\end{verbatim}

%%%%%%%%%%%%%%%%%%%%%%%%%%%%%%%%%%%%%%%%
\subsection{tdepth/2}

{\tt tdepth(T,N)}

This predicate calculates the depth of the term {\tt T} and
unifies the result with the second parameter {\tt N}.
The depth of a term is maximal length of its paths.
%-------------------------------------------
\begin{verbatim}
p <- tdepth(f(a,X),2)
p <- size(f(a,g(Y)),3) 
\end{verbatim}


---------------------------------------
Testing and bounding tableau complexity
---------------------------------------


*) BUILT_IN ::= getdepth^( VARIABLE ) 

   Comment: Instantiates VARIABLE with the maximum tableau depth 
            (allowed for a final tableau) minus current depth.

*) BUILT_IN ::= getinf^( TERM ) 

   Comment: Instantiates VARIABLE with the current number
   	    of inferences (i.e. the number of inferences needed
	    to derive the current tableau).

*) BUILT_IN ::= getmaxinf^( TERM ) 

   Comment: Instantiates VARIABLE with the maximum number 
   	    of inferences (allowed for a final tableau).


*) BUILT_IN ::= setdepth^( NAT )
	
   Comment: Setting the still available depth to the value of NAT

*) BUILT_IN ::= setinf^( NAT )

   Comment: Setting the current inference counter to the value of NAT.

*) BUILT_IN ::= setmaxinf^( NAT )

   Comment: Setting the maximum inference counter to the value of NAT.



----------------------------------------
Other possibilities for explicit control
----------------------------------------


*) BUILT_IN ::= fail

   Comment: `fail' is an usolvable subgoal. 

*) BUILT_IN ::= stop

   Comment: `stop' is a subgoal letting the proof procedure halt.

*) BUILT_IN ::= precut 
            |   cut 

   Comment: The Prolog clause

 	p(X) :- q(X), 
	        !, 
		r(X).

             is operationally equivalent to the LOP clause

 	p(X) :- precut, 
	        q(X), 
		cut, 
		r(X).

   Comment: The Prolog Clause 

 	p(X) :- not q(X), 
		r(X).

 	     is operationally equivalent to the following sequence of
	     LOP clauses

 	p(X) :- not_q(X),
		r(X).

	not_q(X) :- q(X),
		 !,
		 fail.

        q(X).
  

------------------
Output statements
------------------

*) To open a file with filename STRING:

	BUILT_IN ::=  tell^( "STRING" )


*) To close the file with filename STRING:

	BUILT_IN ::= told

*) To write the value of variables into the file just opened 
   or onto the screen:

	BUILT_IN ::= write^( VARIABLE )

   Note: The write statement is executed immediately whenever
   	 it is met during the search. 

   Note: to obtain the full answer substitution the write statement 
         must be the last subgoal of a QUERY (rather than a GOAL). 
	 This is because goal clauses are fanned.

	 In order to output disjunctive answer substitutions 
	 global variables are needed:

Example:

   Given the clause

        p(a) ; p(b) <-.

   the `disjunctive answer' for the query

	<- p(X).

   is `X = a ; X = b'.


  A program which computes this answer needs a new `artificial query' 
  containing a subgoal to enter the original goal. In this way, the 
  goal is transformed into an axiom. The axiom however, is fanned into
  contrapositives. The program is given below.

# In order to initialize list $X:

   ?- $X := [], query, write_list($X).
   
# In order to collect the answer substitutions in the list $X:

   query <- $X := [X|$X], p(X).

# The disjunctive fact remains unchanged:

   p(a) ; p(b) <-.

# In order to output the elements of $X:

   write_list([]) :- write(".").
   write_list([H|T]) :- write(H), write(";"), write_list(T).
   

Note: It would be helpful if you try this program by your own with SETHEO 
      and then have a closer look onto the proof tree using `xptree'.





       BUILT_IN ::= GLOBAL_VARIABLE :is ARITH_EXPR
