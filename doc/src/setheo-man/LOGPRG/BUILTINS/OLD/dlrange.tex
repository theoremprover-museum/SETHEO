%%%%%%%%%%%%%%%%%%%%%%%%%%%%%%%%%%%%%%%%%%%%%%%%%%%
%   SETHEO MANUAL
%	(c) J. Schumann, O. Ibens
%	TU Muenchen 1995
%
%	%W% %G%
%
% BUILT-IN:	
% SCCS:		%W% %G%
% AUTHOR:	J. Schumann
%
%%%%%%%%%%%%%%%%%%%%%%%%%%%%%%%%%%%%%%%%%%%%%%%%%%%
\subsection{ dlrange/2}

\begin{description}
\item[Synopsis:]
	{\tt dlrange(N1,N2)}
\item[Parameters:]\ \\[-0.5cm]
	\begin{description}
	\item[{\tt N1}]
lower bound of the range to print
	\item[{\tt N2}]
upper bound of the range to print
	\end{description}
\item[Low Level Name:]
	{\tt dlrange}
\item[Result:]\ \\
%----------------------------------------------------
This built-in succeeds, if both parameters are
instantiated to numbers.

\end{description}

\vspace*{0.5cm}
\noindent
{\bf Description.}

 
This built-in dumps all lemmata in the lemma-store
with annotated values between the first and the second parameter
(inclusive; both parameters must be instantiated to numbers)
onto the current output-file (as set by {\bf tell/1}), or on
stdout.
Lemmata are printed in a LOP-like syntax. Therefore, the output can directly
be processed by {\bf inwasm(1)}.
 
\vspace*{0.5cm}
\noindent
{\bf Side-effects.}

\vspace*{0.5cm}
\noindent
{\bf Notes.}

\vspace*{0.5cm}
\noindent
{\bf Example~1.}
\begin{verbatim}
printlemma <- tell("out"),write("Lemmata:\n"),
              dlrange(3,4),told.
\end{verbatim}


