%%%%%%%%%%%%%%%%%%%%%%%%%%%%%%%%%%%%%%%%%%%%%%%%%%%
%   SETHEO MANUAL
%	(c) J. Schumann, O. Ibens
%	TU Muenchen 1995
%
%	%W% %G%
%
% BUILT-IN:	
% SCCS:		%W% %G%
% AUTHOR:	J. Schumann
%
%%%%%%%%%%%%%%%%%%%%%%%%%%%%%%%%%%%%%%%%%%%%%%%%%%%
\subsection{\$eq/2, ==}

\begin{description}
\item[Synopsis:]
	{\tt \$eq(T1,T2)}
\item[Parameters:]\ \\[-0.5cm]
	\begin{description}
	\item[{\tt T1,T2}]
erms to be checked for syntactical equality.

	\end{description}
\item[Low Level Name:]
	{\tt eq}
\item[Result:]\ \\
This built-in succeeds, if {\tt T1} is syntactical equal to
{\tt T2}.
\end{description}

\vspace*{0.5cm}
\noindent
{\bf Description.}

This predicate checks two terms for syntactical equality.
Two terms are syntactical equal, if
\begin{itemize}
\item
both are the same symbolic constants (os strings)
\item
both are the {\em same\/} variable,
\item
the terms are of the form
$f^1(t_1^1,\ldots,t_n^1)$ and
$f^2(t_1^2,\ldots,t_n^2)$, and the function symbols $f^1,f^2$ and all
pairs of subterms $t^1_i,t^2_i$ are syntactical equal.
\end{itemize}
 

\vspace*{0.5cm}
\noindent
{\bf Side-effects.}

\vspace*{0.5cm}
\noindent
{\bf Notes.}

\vspace*{0.5cm}
\noindent
{\bf Example.}
\begin{verbatim}
p(X)<- eq(f(a),X).
p(X)<- eq(f(Y),X).
\end{verbatim}
The first clause succeeds, if $X$ is instantiated to $f(a)$.
The second clause, however, does {\em never\/} succeed, since
$Y$ only occurrs only once in the second clause.
 
