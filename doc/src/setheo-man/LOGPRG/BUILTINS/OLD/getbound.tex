%%%%%%%%%%%%%%%%%%%%%%%%%%%%%%%%%%%%%%%%%%%%%%%%%%%
%   SETHEO MANUAL
%	(c) J. Schumann, O. Ibens
%	TU Muenchen 1995
%
%	%W% %G%
%
% BUILT-IN:	
% SCCS:		%W% %G%
% AUTHOR:	J. Schumann
%
%%%%%%%%%%%%%%%%%%%%%%%%%%%%%%%%%%%%%%%%%%%%%%%%%%%
\subsection{\$get{\em bound\/}/1}

This section describes a number of built-ins which allow to access the
current (or maximal) values of the search bounds.
\begin{description}
\item[Synopsis:]
	{\tt \$get{\em bound}(N)}
\item[Parameters:]\ \\[-0.5cm]
	\begin{description}
	\item[{\tt N}] number or variable
	\end{description}
\item[Low Level Name:]
	{\tt get{\em bound}}
\item[Result:]\ \\
\end{description}

\vspace*{0.5cm}
\noindent
{\bf Description.}
A built-in of this group obtains the current or maximal value
of a search-bound and
unifies this value with the parameter {\tt N}.
\begin{description}
\item[{\tt \$getdepth(N)}]
gets the maximum tableau depth
            (allowed for a final tableau) minus current depth.
\item[{\tt \$getinf(N)}]
   gets the current number
            of inferences (i.e. the number of inferences needed
            to derive the current tableau).
\item[{\tt \$getlocinf(N)}]
   gets the current value 
            of the local inference bound.
\item[{\tt \$getmaxinf(N)}]
Obtains the maximum number
            of inferences (allowed for a final tableau).
This value is set at the beginning of a search level (when iterative
deepening is used), and then kept constant.
\end{description}

\vspace*{0.5cm}
\noindent
{\bf Side-effects.}

\vspace*{0.5cm}
\noindent
{\bf Notes.}

\vspace*{0.5cm}
\noindent
{\bf Example.}
\begin{verbatim}
p(X) <- getdepth(D), D > 5, q(X).
\end{verbatim}

This clause can be used only, if the remaining allowable depth
is greater than $5$.


