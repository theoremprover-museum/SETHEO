%%%%%%%%%%%%%%%%%%%%%%%%%%%%%%%%%%%%%%%%%%%%%%%%%%%
%   SETHEO MANUAL
%	(c) J. Schumann, O. Ibens
%	TU Muenchen 1995
%
%	%W% %G%
%
% BUILT-IN:	
% SCCS:		%W% %G%
% AUTHOR:	J. Schumann
%
%%%%%%%%%%%%%%%%%%%%%%%%%%%%%%%%%%%%%%%%%%%%%%%%%%%
\subsection{uselemma/1}


\begin{description}
\item[Synopsis:]
	{\tt uselemma(N)}
\item[Parameters:]\ \\[-0.5cm]
	\begin{description}
	\item[{\tt N}]
	\item[{\tt N2}]
	\end{description}
\item[Low Level Name:]
	{\tt }
\item[Result:]\ \\
\end{description}

\vspace*{0.5cm}
\noindent
{\bf Description.}
 
during run-time, only lemmata from the index may be used which carry
a number which is smaller than the given parameter.
 
%----------------------------------------------------
 
This built-in always succeeds. If the pushed lemmata and the remaining
alternatives are to be tried, this built-in must be directly followed
by a {\bf fail}.
%----------------------------------------------------
This built-in tries to use lemmata from the lemma-store.
This built-in must be used within a clause of a specific structure
as shown below.
When called, this built-in extracts all lemmata from the index
and marks them as possible choices, which fulfill the following conditions:
\begin{enumerate}
\item
the lemma in the index must not be marked ``deleted''.
\item
the lemma must be unifiable with the head of the current clause.
\item
the number stored with the lemma must be smaller ($<$) than the
given parameter.
\end{enumerate}.
  
After adding the choices, this built-in succeeds, If the
selected lemmata are to be tried, a fail must be executed
after this built-in.
Then, the lemmata are tried and then
all other or-branches for
the current subgoal which have not yet been touched are tried.
 
 

\vspace*{0.5cm}
\noindent
{\bf Side-effects.}

\vspace*{0.5cm}
\noindent
{\bf Notes.}

\vspace*{0.5cm}
\noindent
{\bf Example}
%----------------------------------------------------
 
The following example illustrates the usage of this built-in.
\begin{verbatim}
...
p(a,b) <-..
p(b,c) <-..
 
p(_,_) <- uselemma(5),fail.
 
p(e,f) <-.
..
\end{verbatim}
 
If the clauses are not reordered, SETHEO first tries the alternatives
{\tt p(a,b)..} and {\tt p(b,c)..}.
Then it extracts all suitable lemmata from the lemma store.
 
These lemmata are pushed upon the stack as additional alternatives
which are tried (caused by the fail), before {\tt p(e,f)..} is tried.

