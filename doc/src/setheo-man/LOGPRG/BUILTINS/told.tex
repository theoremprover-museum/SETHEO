%%%%%%%%%%%%%%%%%%%%%%%%%%%%%%%%%%%%%%%%%%%%%%%%%%%
%   SETHEO MANUAL
%	(c) J. Schumann, O. Ibens
%	TU Muenchen 1995
%
%	%W% %G%
%
% BUILT-IN:	
% SCCS:		%W% %G%
% AUTHOR:	J. Schumann
%
%%%%%%%%%%%%%%%%%%%%%%%%%%%%%%%%%%%%%%%%%%%%%%%%%%%
\subsection{\$told/0}

\begin{description}
\item[Synopsis:]
	{\tt \$told}
\item[Low Level Name:]
	{\tt told}
\item[Result:]\ \\
This built-in always succeeds
\end{description}

\vspace*{0.5cm}
\noindent
{\bf Description.}
When encountered, this built-in closes a named file which have previously
been opened by {\tt \$tell(T)}.

\vspace*{0.5cm}
\noindent
{\bf Example.}
\begin{verbatim}
p(X) :- $tell("output"),$write(X),$told,$write("done").
\end{verbatim}
This clause writes the current values of {\tt X} into the file
{\tt output}. Then, the word {\tt done} is printed to stdout.


