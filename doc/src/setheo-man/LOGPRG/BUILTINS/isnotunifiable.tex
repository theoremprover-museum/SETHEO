%%%%%%%%%%%%%%%%%%%%%%%%%%%%%%%%%%%%%%%%%%%%%%%%%%%
%   SETHEO MANUAL
%	(c) J. Schumann, O. Ibens
%	TU Muenchen 1995
%
%	%W% %G%
%
% BUILT-IN:	
% SCCS:		%W% %G%
% AUTHOR:	J. Schumann
%
%%%%%%%%%%%%%%%%%%%%%%%%%%%%%%%%%%%%%%%%%%%%%%%%%%%
\subsection{\$isnotunifiable/2}

\begin{description}
\item[Synopsis:]
	{\tt \$isnotunifiable(T1,T2)}
\item[Parameters:]\ \\[-0.5cm]
	\begin{description}
	\item[{\tt T1,T2}]
two terms to be checked.
	\end{description}
\item[Low Level Name:]
	{\tt is\_notunifiable}
\item[Result:]\ \\
This built-in succeeds, if both terms are not unifiable.
\end{description}

\vspace*{0.5cm}
\noindent
{\bf Description.}
This built-in tests, if both terms are unifiable, and fails, if they are. 
The terms are not modified.

\vspace*{0.5cm}
\noindent
{\bf Example.}
\begin{verbatim}
p(X) <- $isnotunifiable(X,f(g(Y,b),c)).
\end{verbatim}
tests, if  {\tt X} is not unifiable with {\tt f(g(Y,b),c)}.
 



