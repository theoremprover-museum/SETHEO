%%%%%%%%%%%%%%%%%%%%%%%%%%%%%%%%%%%%%%%%%%%%%%%%%%%
%   SETHEO MANUAL
%	(c) J. Schumann, O. Ibens
%	TU Muenchen 1995
%
%	%W% %G%
%
% BUILT-IN:	
% SCCS:		%W% %G%
% AUTHOR:	J. Schumann
%
%%%%%%%%%%%%%%%%%%%%%%%%%%%%%%%%%%%%%%%%%%%%%%%%%%%
\subsection{\$genulemma/3}

\begin{description}
\item[Synopsis:]
	{\tt \$genulemma(N1,N2,N3)}
\item[Parameters:]\ \\[-0.5cm]
	\begin{description}
	\item[{\tt N1}]
        This number indicates the number of the literal for which
        a unit-lemma is to be generated. {\tt N1} must be in the
        range between $1$ and the length of the clause (including
        built-in literals).
        \item[{\tt N2}]
        If this number is $\neq 0$, the generation of a unit-lemma
        is considered. Otherwise, this predicate just succeeds.
        \item[{\tt N3}]
        If a unit-lemma is stored in the index, this number will be
        stored together with the lemma. This information can the
        be used when unit-lemmata are being used (e.g., by {\bf uselemma/1}).
        \end{description}
\item[Low Level Name:]
	{\tt genulemma}
\item[Result:]\ \\
        This built-in predicate always succeeds. Memory overflow and
        wronly instantiated parameters will result in a
        non-fatal run-time error.
\end{description}

\vspace*{0.5cm}
\noindent
{\bf Description.}
%----------------------------------------------------
This built-in generates a unit-lemma {\tt L<-.} for a literal
{\tt ~L} in the current clause. Its number is given as the first argument.
The lemma is generated, if all of the following
conditions hold.
The lemma will be annotated by the value of the third argument
(must be a number).
 
\begin{itemize}
\item
all arguments must be instantitated to numbers,
otherwise, a run-time error is generated.
\item
the first argument must be instantiated to a number
in the range between $1$ and the length of the clause (including
built-in literals).
\item
the value of the second argument is $\neq 0$,
\item
the current instantiation of the given literal is not subsumed
by any unit-lemma in the lemma store.
\end{itemize}

If the unit-lemma subsumes one or more unit-lemmata in the lemma-store,
these unit-lemmata are deleted.

%----------------------------------------------------
 
 
%----------------------------------------------------
\vspace*{0.5cm}
\noindent
{\bf Notes.}
\begin{itemize}
\item
the range of the first argument is NOT checked
\item
it is not checked, if the solution of the given literal required
reduction steps above the current node in the tableau.
Therefore, it is easily possible to generate unsound proofs, if
yet unsolved literals are kept as lemmata.
\item
all built-in literals (including this one) count as literals
\end{itemize}

\vspace*{0.5cm}
\noindent
{\bf Example.}
\begin{verbatim}
q(X,Y) :- p(X),$genulemma(2,1,5),r(X,Y).
\end{verbatim}
 
generates a lemma {\tt p(a)<-.}, if X is instantiated to {\tt a}.
The lemma carries the number $5$.
 


