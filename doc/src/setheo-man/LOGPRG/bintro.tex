%%%%%%%%%%%%%%%%%%%%%%%%%%%%%%%%%%%%%%%%%%%%%%%%%%%
%   SETHEO MANUAL
%	(c) J. Schumann, O. Ibens
%	TU Muenchen 1995
%
%	%W% %G%
%%%%%%%%%%%%%%%%%%%%%%%%%%%%%%%%%%%%%%%%%%%%%%%%%%%
\section{List of Built-Ins}\label{sec:builtins}

The SETHEO system provides a number of built-in predicates
for programming and control purposes.
Most of them are used in a similar way to PROLOG, but several
differences exist.

The following list describes all built-in predicates available
with SETHEO Version V3.3.
This list is likely to be extended in future versions of SETHEO.
Therefore, changes are marked clearly.
Furthermore, built-ins can be implemented by the user, using
SETHEO's C-interface as described in Section~\ref{sec:C-interface}.
However, no support is provided in this case.

Syntactically, a built-in predicate is an {\tt ATOM} 
(in the LOP-file), corresponding to the syntax definition in
Section~\ref{sec:lop-syntax}.
Whereas most built-in predicates are in prefix notation
(of the form $\$p$ or $\$p(t_1,\ldots,t_n)$), some (mostly for
arithmetic) are available in infix notation.
In contrast to PROLOG, the user cannot define if a predicate can
appear in prefix, infix or postfix notation.

%For the following description of the built-in predicates, we give
%a list of possible data-types for the parameters. Unless otherwise
%specified, all parameters are passed using (sound) unification.
%
%\begin{description}
%\item[V] at the parameter position, an unbound variable (logical or
%	%global) must be present. This variable must not be bound to any term
%	at the time when the built-in predicate is invoked. 
%\item[\$V] the name of a global variable is required at this
%	parameter position.
%\item[N] a numeric value is required at this parameter position.
%\item[C] a symbolic constant (not a number) must be present at the
%	parameter position.
%\item[NUMEXPR] a numerical expression is required which must evaluate to a
%	number during run-time (i.e., its variables (if any) must be
%	bound to numbers).
%\item[L] a list, or an empty list, denoted by {\tt []} is
%	required.
%\item[T] an arbitrary term can be used at this parameter position.
%\end{description}
%
%In all cases, variables may be present instead of the required type of
%the parameter, as long as they are getting bound to the appropriate
%type at the time when the predicate is being called.
%A number may be appended to all type specifiers in order allow for
%references to a specific parameter (e.g., {\tt genlemma(N1,N2)}).

{\tt low-level name}

