%%%%%%%%%%%%%%%%%%%%%%%%%%%%%%%%%%%%%%%%%%%%%%%%%%%
%   SETHEO MANUAL
%	(c) J. Schumann, O. Ibens
%	TU Muenchen 1995
%
%	%W% %G%
%%%%%%%%%%%%%%%%%%%%%%%%%%%%%%%%%%%%%%%%%%%%%%%%%%%
\section{Global Variables}
\label{LOP:GLOBVAR}

The concept of {\em destructive\/} and {\em multiple\/} assignment
for variables is one of the basic concepts in a procedural
programming language.
In logic programs, however, a variable has quite a different meaning:
it is a {\em logical variable\/} \cite{Kow79b}.
A logical variable can be used and modified by the {\em unification\/}
procedure only which may perform a {\em specialisation\/}.

This means that a variable which is in the beginning of the
calculation {\em unbound\/} may be given several values during
the proof, but each value is a specialisation of the previous ones
(or the same).
Otherwise unification is not possible.
This behavior of logical variables does, of course, not allow
multiple destructive assignments to a variable, e.g.\ {\tt X:=5} and
afterwards {\tt X:=7}.

Another basic principle of procedural languages is the existence of
{\em global variables\/} which are visible and valid from all parts
of the program and in all incarnations of the procedures or functions.
This is different in logic: in clausal form, a variable is valid only within
{\em one\/} clause.

For a programmer who is using procedural languages most of the time,
it would be very convenient to have 
the possibility of using (global) variables with multiple destructive
assignment in a logic program.
This inclusion also partly
bridges the gap between procedural languages and logic languages, as
the concept of global variables can be embedded cleanly into the
logic language as shown below.
This is in contrast to PROLOG which developed the concept of global
data by introducing the {\em assert\/} and {\em retract\/}
predicates. These constructs, however, have no clean and
natural denotational semantics.
%%\footnote{The operational semantics for these constructs are
%%very complicated and unnatural as well [???]}.
Note that a multiple destructive assignment in a logical language
must be {\em backtrackable\/} when the underlying execution model
needs backtracking to find a solution (e.g.\ the \SAM\ and  PROLOG).

So we introduce global backtrackable variables
(see also \cite{LS88}). This concept has a clean denotational
semantics, which has been introduced by R.~Letz and will be shown below,
and allows for a very efficient and easy integration into 
the \SAM.
The only draw-back of this construct is that we have to commit ourselves
to a fixed order of evaluation of the subgoals
({\em literal selection function}),
namely left-to-right. Only with this restriction we can make sure
that the assignment {\tt \$X:=1} will be executed
{\em prior\/} to {\tt \$X:=2} in the tail of
a clause {\tt \ldots,\$X:=1,\ldots,\$X:=2,\ldots} which is
equal to the intuitive meaning of (procedural) evaluation.

In LOP the concept of global variables and multiple destructive assignment
is employed as a basis for almost all its non-logical and
meta-programming constructs.


\subsection{Syntax and Usage of Global Variables}

In LOP global variables are denoted like variables, but their names
start with a '\$', e.g.\ {\tt \$X}, {\tt \$World}
are global variables.
Global variables can be used like ordinary variables in the terms
of the predicates. Besides this, they may appear on the left hand side
of the destructive assignment ({\tt :=}).
On the right hand side of an assignment any term may be present.
In such a term the global variable which occurs on the left hand side
of the assignment may be present as well,  e.g.\
{\tt \$X:=f(\$X,a)} or {\tt \$X:=[a,\$X]}.

Beside single global variables, {\em arrays\/} of global variables
can be used. They are one-dimensional.
These arrays must be declared before
usage
by a {\tt globvar(\$X,size)}
built-in predicate
where {\tt size} must be a positive number. After execution of this subgoal
an array of global variables {\tt \$X[0]},\ldots,{\tt \$X[size-1]}
is available to the user.
Indexed variables are used like ordinary global variables. The index
must be a number in the correct range.
Otherwise an error occurs.
%If the index is a variable which
%has not yet been bound, an error occurs or the current subgoal has
%to be delayed in its execution.

\subsection{Semantics of Global Variables}

The description of the semantics of the global variables follows the
idea of {\em paraphrasing\/} as presented by R.~Letz in \cite{LS88}.
A logic program $P$ with global variables is transformed into
a program $P^\prime$ {\em without\/} global variables, but with the same
behavior of the variables.
More formally, this transformation $\cal T$ 
can be represented by a commutative diagram shown in the
following Figure.
The execution calculus is $\cal C$, in our case model elimination.
In the case of $P$ with global variables with destructive
assignment the selection function is fixed
to left-to-right (L-R) denoted by ${\cal C}_{LR}$

\begin{figure}[htb]
\setlength{\unitlength}{1mm}
\begin{center}
\begin{picture}(100,40)
\put(0,25){\framebox(30,15){\shortstack{$P$ \\ {\footnotesize w.\ global variables}}}}
\put(70,25){\framebox(30,15){\shortstack{$P^\prime$ \\ {\footnotesize no global variables}}}}
\put(30,32.5){\vector(1,0){40}}
\put(15,25){\vector(0,-1){20}}
\put(85,25){\vector(0,-1){20}}

\put(50,33.5){\makebox(0,0)[b]{$\cal T$}}
\put(50,0.5){\makebox(0,0)[b]{\shortstack{$=$ \\ same behavior w.r.t.\ \\
variable substitutions}}}

\put(16,13){${\cal C}_{LR}$}
\put(86,13){$\cal C$}
\end{picture}
\end{center}

\caption{Transformation $\cal T$ and its semantics}
\end{figure}

The transformation $\cal T$ transforms each literal $L$ of the program
to a literal $L^\prime$ with a higher arity. The additional arguments are
used to ``pass'' the values of the global variables.
Given a clause $c = L_1,\ldots,L_n$ with literals $L_i = p_i(T_i)$, where
$T_i$ denotes the parameter terms of the literal $L_i$. Then we transform the
clause $\ldots,p_{i-1}(T_{i-1}),p_i(T_i),p_{i+1},\ldots$ as follows:

\begin{center}
\setlength{\unitlength}{1mm}
\begin{picture}(140,25)
\put(38,10){\makebox(0,0)[rb]{$p^\prime_{i-1}(T_{i-1},$}}
\put(40,10){\makebox(0,0)[b]{$\circ$}}
\put(42,10){\makebox(0,0)[b]{,}}
\put(44,10){\makebox(0,0)[b]{$\bullet$}}
\put(46,10){\makebox(0,0)[b]{$),$}}

\put(78,10){\makebox(0,0)[rb]{$p^\prime_i(T_i,$}}
\put(80,10){\makebox(0,0)[b]{$\circ$}}
\put(82,10){\makebox(0,0)[b]{,}}
\put(84,10){\makebox(0,0)[b]{$\bullet$}}
\put(86,10){\makebox(0,0)[b]{$),$}}

\put(118,10){\makebox(0,0)[rb]{$p^\prime_{i+1}(T_{i+1},$}}
\put(120,10){\makebox(0,0)[b]{$\circ$}}
\put(122,10){\makebox(0,0)[b]{,}}
\put(124,10){\makebox(0,0)[b]{$\bullet$}}
\put(126,10){\makebox(0,0)[lb]{$), \ldots$}}

\put(40,20){\vector(0,-1){8}}   % down
\put(80,20){\vector(0,-1){8}}   % down
\put(120,20){\vector(0,-1){8}}   % down


\put(44,12){\vector(0,1){8}}   % up
\put(84,12){\vector(0,1){8}}   % up
\put(124,12){\vector(0,1){8}}   % up

\put(44,20){\line(1,0){36}}   % horiz
\put(84,20){\line(1,0){36}}   % horiz

\put(40,20){\line(-1,0){16}}   % horiz
\put(124,20){\line(1,0){16}}   % horiz
\end{picture}
\end{center}

where the ``$\circ$'' denote new logical variables which hold the value of
the global variables {\em before\/} the execution of the resp.\ literal, the
variables ``$\bullet$'' hold the values afterwards.
The interdependance of the values is shown with the arrows and can be gained
using the same variables.
Formally we write down the 

{\bf Transformation $\cal T$}. Let $g$ be the number of global variables in
a program $P$. Then each clause
of $P$ is transformed as shown in the following table.
We use
new (logical) variables denoted by $V^j_i$ for some $i,j$.
For the sake of readability we abbreviate $ T := t_1,\ldots,t_n$.

\begin{center}
\begin{tabular}{|l|l|}
\hline 
$P$ & $P^\prime$ \\
\hline \hline
{$p(T) \leftarrow$. } &
   $p^\prime(T,V_1^0,V_1^0,\ldots,V_g^0,V_g^0) \leftarrow$. \\
\hline
$p(T) \leftarrow$ & $p^\prime(T^\prime,V^0_1,V^n_1,\ldots,V^0_g,V^n_g) \leftarrow$ \\
\ \ $p_1(T_1),\ldots,p_n(T_n)$ & \ \ 
  $\ldots,{p^\prime}_i(T_i^\prime,V^{i-1}_1,V^i_1,\ldots,V^{i-1}_g,V^i_g),\ldots
; 1 \leq i \leq n$ \\
\hline
$L_i \equiv {\$G}_k$ := $T$ &
$V^i_1 = V^{i-1}_1,\ldots,V^i_k = T^\prime,\ldots,V^i_g = V^{i-1}_g$\\
$1 \leq k \leq g , 1 \leq i \leq n$ & \\
\hline 
\hline
\end{tabular}
\end{center}
\vspace{2mm}

with
\[ T^\prime_i \equiv 
         \left\{
  \begin{array}{ll}
    T_i  &
    \mbox{if ${\$G}_k \not\in T_i , 1 \leq i \leq n , 1 \leq k \leq g$} \\
    \ldots V^{i-1}_k \ldots & 
    \mbox{if ${\$G}_k \in T_i , 1 \leq i \leq n , 1 \leq k \leq g$} \\ 
    \   &
    \mbox{replace ${\$G}_k$ by $V^{i-1}_k$.}
  \end{array}
         \right.
\]

The current value of ${\$G}_k$ may occur in $T$ (destructive assignment), its
value is kept in $Vin^{i-1}_k$ in $P^\prime$.
It is obvious that the execution of $P$ w.r.t.\ ${\cal C}_{LR}$ and
of $P^\prime$ on $\cal C$ has the same structure.
A short example of the transformation is given in the following.

\subsection{Examples of the Usage of Global Variables}
One short example shows how global variables, in our case {\tt \$X},
are backtracked.

\begin{center}
\begin{tabular}{lll}
& $\leftarrow$ {\em example}.& \\
 {\em example} & $\leftarrow$ {\tt \$X := a}, {\em p}.& \\
$p$ & $\leftarrow$ {\tt \$X := b}, {\em fail}.& \% must backtrack\\
$p$ & $\leftarrow$ {\em write(\/}{\tt \$X}, {\em )}.& \% print value of {\tt \$X}
\end{tabular}
\end{center}

When this program is executed, the variable {\tt \$X} is first set to
the value {\tt a}. After entering the first clause for {\tt p}, the variable
is set to {\tt b} (destructive assignment). After that, a failure occurs
which causes a backtracking step to be performed and the second clause
can be tried, printing the current value of {\tt \$X}.
If global variables were not
backtrackable, the value after the last assignment, in our case {\tt b}
would be printed. This, however, is not correct.
With the usage of {\em backtrackable\/} global variables as introduced
above, the correct value, namely {\tt a}, is printed.

We now give the transformation of the program presented above. As we only
have one global variable, namely {\tt \$X}, only one pair of new variables
is needed per predicate. The names of the augmented predicates are
marked by a ``$^\prime$''. The first variable of each predicate can be seen
as the ``input value'' of the global variable {\tt \$X}, the second one
as the ``output value'' after execution of the subgoal.

\begin{figure}[htb]
\begin{center}
\begin{tabular}{lll}
& $\leftarrow example^\prime(V_1^0,V_1^1)$. & \\
$example^\prime(V_1^0,V_1^2)$ & $\leftarrow V_1^1 = a , p^\prime(V_1^1,V_1^2)$. & \\
$p^\prime(V_1^0,V_1^2)$ & $\leftarrow V_1^1 = b , fail^\prime(V_1^1,V_1^2)$. & \\
$p^\prime(V_1^0,V_1^1)$ & $\leftarrow write(V_1^0,V_1^0,V_1^1)$. & \\
\end{tabular}
\end{center}
\caption{Transformation of a program with global variables}
\end{figure}

\subsection{Restictions for Global Variables}

Global variables are in general not compatible with SETHEO's
anti-lemma mechanism, as is shown with the following example:

\begin{verbatim}

?-p, q($X).

q(b).

p :- $X := a.
p :- $X := b.
\end{verbatim}

This logic program first solves the subgoal {\tt p}, whereby {\tt \$X}
is assigned to {\tt a}. Then the second subgoal is tried, but here
the unification fails. During backtracking, the anti-lemma mechanism
determines that {\tt p} has been solved optimally (there are no substitutions
for logical variables in {\tt p}). Hence it is decided that there exist no
more solutions, and thus the \SAM\ returns with ``total failure''.
If the generation of antilemmata is disabled, the above program works
correctly.
