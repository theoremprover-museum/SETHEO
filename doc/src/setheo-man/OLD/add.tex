\subsection{Adding Built-in Predicates}

New built-in predicates can be added easily to the SETHEO system.
Changes are necessary in the {\bf inwasm} and {\bf setheo} only.

\subsubsection{Built-ins in inwasm}

Two types of built-in predicates can be handled by {\bf inwasm}
without changing the C-sources:
\begin{enumerate}
\item
predicates which are mapped to an {\bf built} SAM-instruction, and
\item
predicates which are defined in a SAM-Assembler library.
\end{enumerate}

Both predicates are defined in the file {\em .inwasmrc\/} which is
located in the current directory (or the home-directory)
and read each time, the {\bf inwasm} is invoked. It
consists of lines with the format:

\[ : <arity> : b|l : <maj\_nu> : <name> \]
where
\begin{description}
\item[$<$arity$>$]
is the arity of the predicate,
\item[b]
predicates with that type are mapped to the SAM-assembler
statement {\bf built  $<maj\_nu>$ , argv}.
\item[l]
predicates which are defined within a library carry such a type.
The file-name of the library is $<name>$.lib and located in
the \$LIBDIR which is an environment variable of {\bf wasm}.
\item[$<maj\_nu>$]
is the number which is the first argument of the {\bf built} instruction
and must be same which is defined in {\em conf.c\/} of setheo.
This number must be unique for each predicate.
\item[$<name>$]
is the name of the predicate to be defined.
\end{description}

Note that no blanks are allowed between the colons and the data.
A typical startup file is shown in the following:

\begin{verbatim}
:2:b:18:size
:3:l:0:math
:0:b:19:precut
:1:b:20:cut
:0:l:0:prtdepth
\end{verbatim}

When a predicate in a library is to be called, the generated instruction
is {\tt call $<name>$\_\_ , argv}.
The entry-pint of the library must define the label
{\tt $<name>$\_\_}, as shown in the example, which defines
a predicate to print the current depth:

\begin{verbatim}
prtdepth__:
	nalloc	1 , 137,0	% do not add depth
	built	15 ,0           % get depth
	lconst  1               % + 1
	add
	sto	0
	out	avx002003       % print it
	ndealloc                % and return
avx002003:
	.dw	var	0
\end{verbatim}

Note that the labels which are defined and used within a library must
not be the same as used in the main program.
Also the usage of predefined constants (e.g.\ strings) will not lead
to the desired result.

\subsubsection{Changing SETHEO}

In order to implement a new built-in which can be invoked
using a {\tt built} SAM-instruction, a C-function has to be written.
It returns 0 if the predicates succeeds, 1 otherwise.
The name of the function may be arbitrary, but must not coincide with a function
name already in use (which will result in an error message during linkage).

Arguments of the predicate which are of the type {\tt WORD}
can be accessed using the macro {\tt ARGV[$<number>$]}
where $<number>$ is the argument position staring at 0.
The Macros returns a pointer to the argument.
To return a value, the macro {\tt UNIF($<number>$,WORD*)}
can be used. If the argument $<number>$ is a free variable,
it will be bound to WORD*, if it is a constant, both values are compared
and a fail is done if they are not equal.
All global variables (as defined in {\em machine.h\/})
and all SETHEO functions can be used inside the predicate.

For the following let us assume that the function has the name {\em foo()\/}
and is in the file {\em foobar.c}.

As a second step the file {\em conf.c\/} must be modified.
A Version number of SETHEO and the IDENT may be changed
in that file. Both data will be printed when setheo is invoked.
The newly added function {\em foo()\/} must be declared as
{\bf int foo()}.
At the end of the stucture {\tt built\_tab}
the function {\em foo\/} is added.
In comments the $<maj\_nu>$ is given which is incremented by one and
gives the index into that structure.
The file {\em conf.c\/} looks as follows (function {\em foo()\/} already added).

\begin{verbatim}
/* FILE: conf.c
...
/***************************************************************/
/* identification name of that SETHEO version
/***************************************************************/
#define IDENT "MY_VERSION"

/***************************************************************/
/* Version number of SETHEO
/***************************************************************/
#define VERSION "2.56"

int is_compl();
int getdepth(), getinf();
...
int foo();

int (*built_tab[])() =
         {
         isempty                /*  0 */
         issubset,              /*  1 */
         ...
         cut,                   /* 20 */
         foo,                   /* 21 */
};
\end{verbatim}


A typical file {\em foobar.c\/} must include several setheo-include files.

\begin{verbatim}
#include <stdio.h>
#include "machine.h"
#include "symbtab.h"
#include "built.h"


#define ARGV(I) deref(code + *(pc + 2)+ I, bp);

/******************************************************/
/* foo
/******************************************************/
int foo()
{
WORD *arg1;

arg1 = ARGV(0);

if (GETTAG(*arg1) != T_CONST){
	error("foo: parameter must be a constant",arg1,1);
	return 1; /* fail */
	}
...
return 0;
}
\end{verbatim}

The customised SETHEO can be made by the following makefile:

\begin{verbatim}
LIBDIR=../gsrc      # here resides the libseth.a
MACHINE = sun4
CFLAGS = -c -O -D$(MACHINE) -DGRAPH_TREE -DBUILT_INS\
         -DSTATISTICS -I../include -I./instruct
	
OBJS = foo.o


setheo: $(LIBDIR)/libseth.a $(OBJS)
        cc -o setheo $(OBJS) -L$(LIBDIR) -lseth

foo.o: foo.c  
        cc $(CFLAGS) foo.c
\end{verbatim}

