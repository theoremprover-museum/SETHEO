\subsection{Selection of Start Clauses}

Per default, SETHEO takes all clauses which contain negative
clauses only, as possible start clauses.
In LOP syntax, these clauses have the following form:

\[ \mbox{\tt <-} L_1, \ldots, L_n. \]

with atoms (non-negated literals) $L_i$.
These clauses are also referred to as {\em queries}.

The Model Elimination Calculus, however, allows to select
an {\em arbitrary\/} clause as start clause.
This can be of interest if the conjecture to be proven is of the form
$A_1,\ldots,A_n \rightarrow B$. If a true goal-oriented serach is intented,
the search should start with this clause.
On the other hand, axioms like {\tt <-equal(X,succ(X)).} should in general
not be used as a starting clause.

Cases like this can be accomplished by SETHEO without affecting
the satisfiability of the formula.
Here, we describe a way to (a) disable a query, and to
(b) convert an arbitrary clause into a query.
Please note, that at least one query must be present in the formula in order
to preserve satisfiability.

\paragraph{Disabling a Query.}
A query (as shown above) can be disabled by replacing that clause
with the following one (for any $1 \leq i \leq n$):

\[ {\tt ~}L_i \mbox{\tt <-} L_1, \ldots, L_{i-1},L_{i+1},\ldots,L_n. \]

This rule cannot be used as a starting clause by SETHEO. nevertheless,
all contrapositives are being generated.

\paragraph{Converting a Clause into a Query.}
Given a clause
\[ H_1 ; \ldots ; H_m \mbox{\tt <-} L_1, \ldots,L_n. \]
with atoms (non-negated literals) $L_i,H_i$.
Then this clause can be converted into a query by replacing this
clause by:
\[ \mbox{\tt <-}
{\tt ~}H_1, \ldots ,{\tt ~}H_m, L_1, \ldots,L_n. \]

For this clause, all contrapositives are generated.


\paragraph{Example:}

{\tt irgend ein GRP bzw. Bledsoe}
