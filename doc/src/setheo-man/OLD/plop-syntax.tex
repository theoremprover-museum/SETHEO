In\footnote{This text is from the OLD SETHEO MANUAL.}
 this part we describe the syntax of the input language, which is
accepted by the formula translator {\bf plop}.
A formula in that notation has to be in a file with the extension {\bf .pl1}.

Because the grammar of first order logic is well-known it seems
useless to repeat an inductive definition of the concept 
{\em ``formula''}. We rather represent the special PLOP
requirements i.e. the differences to the usual
standard in giving some comments on the individual
elements of the accepted language

\paragraph{Variables}
Variables are strings starting with a capital letter followed 
by arbitrary letters, digits, or underscore.

\paragraph{Constants}
Predicate-, function-, and object parameters are strings starting 
with a lower case letter followed by lower case letters, digits, or underscore. 
Predicate- and function arguments are to be included in round brackets.
Multiple arguments are to be separated by a comma ``,''.

\paragraph{Propositional Connectives}
The symbols for the propositional connectives are:

\begin{center}
\begin{tabular}{llll}
(1) &  {\tt $\sim$}  &   - &  not, \\
(2)  &	 {\tt \& } & -  & and, \\
(3)   &{\tt ; } & -  & or, \\
(4)   &{\tt -> } & -  & if ... then, \\
(5) & 	 {\tt <->}  & -  & iff. 
\end{tabular}                  
\end{center}

The binding preference is given by this order:
1 $>$ 2 $>$ 3 $>$ 4 $>$ 5 . 
Equal connectives are left associative i.e.,
\tt 
\mbox{a -> b -> c} \rm is to be read as \tt \mbox{((a -> b) -> c)} 
\rm .

\paragraph{Quantifiers}
The quantifiers are represented by
``{\tt
forall}'' and ``{\tt exists}''.
These symbols must not occur as parts of other names like constants
or variables.
The scope of a quantifier extends over the following propositionally 
complete subformula and ends there.
E.g. at the formula
\[\tt 
\mbox{forall X a -> b(X)}  \\
\rm
\]
the scope of the quantifier does not include {\tt b(X)}.
To bind {\tt X} by this quantifier one has to write as usual with parenthesis
\[\tt 
\mbox{forall X (a -> b(X)).}\\
\rm
\]
Occurences of free variables are understood by PLOP as bound
from the outside (universal closure).

\paragraph{Brackets}
Parenthesis are expressed by the usual {\em round brackets} only.
{\em Square brackets} must not be used in the whole input formula .

\paragraph{Special Characters}
All characters with ASCII value $<$ 32 are ignored and will be deleted
during the transformation.
The {\em space-character} can be omitted if the meaning is not affected and if
there cannot result any ambiguities. 
E.g. 
\[\tt 
\mbox{forallX} \\
\rm
\]
is recognized as a quantifier and a variable. 
In other cases a space character must be set, e.g.
\[\tt 
\mbox{forall X a(X) }\\
\rm
\]
cannot be written as 
\[ \tt 
\mbox{forall Xa(X).}\\
\rm
\]


Other characters as mathematical function symbols (especially in
infix notation) are not recognized, which perhaps could affect the
correctness of the transformation in some cases and are therefore to 
be treated carefully. 
If you cannot avoid 
them at any rate suppress the optimization concerning redundancies
and tautologies (option -nopti). 

\paragraph{Comments}
Comments are enclosed in C--like convention in ``{\tt /*}'' and ``{\tt */}''
and can extend over several lines. Nested comments are not allowed.
All comments are removed during the conversion.

%\paragraph{Size}
%The admissible size of the input file is limited and can be displayed
%by the option -defgro (``groesse1\_\,c'').
%If it is not sufficient for your example (e.g. a large program) 
%divide the input in several parts or change the define-statement for
%groesse1\_\,c in the source-code.

