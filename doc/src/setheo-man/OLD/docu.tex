\documentstyle[12pt,a4wide,showlab]{report}


\def\SAM{{\sc S\kern-.13em\lower.45ex\hbox{A}\kern-.13emM}}

\begin{document}

\title{SETHEO V3.3\\
Reference Manual}

\author{Johann Schumann \and Ortrun Ibens}

\maketitle

\tableofcontents

\section*{\bf History of SETHEO}

\section*{\bf CopyRight and Licence for SETHEO}

setheo mailing-list

\section*{\bf What's New?}

\section*{Notation}



\chapter{Introduction}

\section{Getting Started}

--- from old manual ---

\input{/home/schumann/papers/setheo-man/getting-started.tex}

\section{SETHEO System Overview}

The SETHEO system consists of a set of (individual) programs,
as shown in Figure~\ref{fig.1}. Data are exchanged between the
modules using files. Parameters for each program are given in the
command line.

\begin{figure}[htb]
\caption{Parts of the SETHEO system}
\label{fig.1}
\end{figure}

The indivual programs will be described in detail below. They are:

\begin{description}
\item[plop]
This program\footnote{This program is distributed as an ``as is'' version
without support. For details see Section~\ref{sec:plop}.} converts
formulae in first order notation (with quantifiers and standard
(infix-) operators (e.g., $\wedge, \vee$) into a set of clauses.
\item[inwasm]
This program is the {\em compiler\/} for SETHEO. It takes a formula
(as a set of clauses) or a logic program as its input. Details of the
input language (LOP) and its syntax are described in detail in
Section~\ref{sec:lop}.
{\bf Inwasm} performs several preprocessing steps which try
to reduce the search space spanned by the formula. Typical preprocessing
steps are the generation of syntactical constraints or the elimination
of pure literals.
Then, assembler instructions for the SETHEO Abstract Machine \SAM\ 
are generated. Optionally, {\bf inwasm} can output 
the preprocessed formula in LOP-syntax.
This option is helpful, if the preprocessed formula is to be
modified by hand (or some other filter), before it should be compiled
for the \SAM.
Details are discussed in Section~\ref{sec:inwasm}.
\item[wasm]
The assembler {\bf wasm} converts assembler instructions (as normally
produced by {\bf inwasm} and converts them into a binary
(currently actually hexadecimal) representation which can be directly
read into the SETHEO Abstract Machine.
Symbolic labels and constants are resolved. Furthermore, the
code is somewhat optimized to reduce the size of the produced code file.

\item[sam]
This program is the interpreter of the SETHEO Abstract Machine
\SAM. After loading the code-file as prepared by {\bf wasm}, the
abstract machine is started with given parameters.
If a proof could be found, the given time-limit has been exceeded,
or upon request, statistical information is printed on the screen
and into a log-file. In case, a proof could be found, the tableau
with instantiated literals is generated and written into a file.

\item[xptree]
This program is X-based graphical viewer for tableaux, generated
by {\bf sam}. The tableau is displayed as a tree with literals
(and additional information) as its nodes. Scrolling, selection and
hinding of subtableaux can be accomplished using the mouse.
\end{description}

Around these basic programs, additional modules are located which
use the basic programs. They increase the functionality of the
SETHEO system and facilitates its usage.

New modules are likely to be added in further distribution versions
of SETHEO. Therefore, modules which are new for the current version of
SETHEO are marked clearly.
Where appropriate, these modules are described in detail in separate
subsections.
The modules are listed in alphabetical order.

\begin{description}
\item[clop]
This script combines the compiler {\bf inwasm} and the assembler
{\bf wasm}. All command-line parameters are passed to the compiler
{\bf inwasm}.
\item[delta]
is a preprocessor which takes a formula in LOP syntax, and produces
one in the same syntax. {\bf Delta} generates unit-lemmata in a bottom-up
way which are added to the original formula. In many cases, this combination
of bottom-up preprocessing (using {\bf Delta}) and subsequent top-down
processing by the \SAM\ results in a dramatic increase of efficiency.
{\bf Delta} uses {\bf inwasm, wasm,} and {\bf sam}.
\item[rctheo]
\item[idtheo]
\item[setheo]
This script represents the top-level script of the SETHEO system.
Given a formula in LOP-notation, the compiler, assembler, and the
prover are activated automatically, using {\em default parameters}.
\item[xvdelta]
This module is a primitve graphical user interface (GUI) for {\bf delta}
preprocessor.
It is based on XVIEW toolkit of X. {\bf xvdelta} facilitates the selection
of parameters of {\bf delta} and to control its operation.
\item[xvtheo]
This module is a primitve graphical user interface (GUI) for SETHEO.
It is based on XVIEW toolkit of X. {\bf xvtheo} allows to edit
a formula, to select approriate parameters (using pull-down menues),
to start the prover and to view the results.
{\bf xvtheo} calls all the basic progams.
\end{description}


%/home/schumann/papers/setheo-man
\subsection{Filename Conventions}

All files used within the SETHEO system carry specific file name
extensions. These extensions are mandatory and are {\em not} given
when a SETHEO command (see Section~\ref{chap:2},\ref{chap:3})
is issued.
In the following we list all possible file name extensions and describe
their contents. Formal definitions of the syntax is given in 
Chapter~\ref{ch:syntax}.

\begin{description}
\item[{\em file}.pl1]
Files with this extension contain formulae in first order predicate
logic in mathematical syntax. Formulae can contain the standard
operators (like $\vee, \wedge,\rightarrow,\leftrightarrow$) and
existential and universal quantifiers. An ASCII representation
of the quantors is given, e.g., {\tt forall} for $\forall$.
A detailed description and formal definition is given in
Section~\ref{sec:pl1-syntax}.
In the current version of SETHEO, this notation is not supported.
Note, however, that this syntax is very similar to the input syntax
of OTTER and can be converted easily (see Section~\ref{sec:plop2otter}).

\item[{\em file}.lop]
Files with this extension contain formulae and/or logic programs
written in LOP syntax. This syntax comprises the default for the
SETHEO system.
A detailed description and the syntax definition is given in
Section~\ref{sec:lop-syntax}. Built-in predicates which can be used
for logic programming are listed in Chapter~\ref{chap:builtins}.
Files with the extension {\bf .lop} comprise the input of the compiler
{\bf inwasm}.

\item[{\em file}\_pp.lop]
Files with this extension are produced by the compiler {\bf inwasm}
when it is invoked with the command-line option {\tt -lop}.
This file contains the formula after preprocessing. Clauses are fanned
into contra-positives, constraints are added to the clauses
(if selected), and the weak-unification graph is given.
Such a file can be read again by the compiler {\bf inwasm}.
Thus, manual (or automatic) modifications of the formula between
preprocessing and the proof attempt can be accomplished.

\item[{\em file}.out.lop]
Files with this extension are produced by the {\sc Delta} iterator
when given {\em file\/} as input.
This file contains (in LOP syntax) the original formula (as found
in {\em file\/}{\bf .lop}) with the newly generated unit clauses
appended to it.

\item[{\em file}.s]
Files with this extension contain \SAM\ assembler statements and
are the input of the assembler {\bf wasm}.
Assembler files are generally generated by the compiler {\bf inwasm}.
Nevertheless, assembler files are (to some extend) human-readable
and can be modified (for prototypical implementation issues).

\item[{\em file}.hex]
Such files are generated by the assembler {\bf wasm} and contain the
binary (i.e., hexadecimal) representation of \SAM-instructions and
\SAM's symbol table. {\bf .hex} files are directly loaded by the
SETHEO Abstract Machine \SAM (command {\tt sam}.

\item[{\em file}.tree]
The prover {\tt sam} generates files with that extension,
if a proof has been found or upon request. Such files contain
one or more Model Elimination Tableaux which then can be displayed
graphically by the proof tree viewer {\bf xptree}.
The syntax of each tableau corresponds to a PROLOG term. Therefore,
tableaux can be read in by a PROLOG system or (after adding
predicate symbols to it) by the compiler {\bf inwasm}.
The syntax of the tree-files are described in Section~\ref{sec:tree-syntax}.
\item[{\em file}.log]
This file is generated by the prover {\tt sam}. It contains useful
information about the current run and comprises
a copy of the data displayed on stdout.
Besides command-line switches and a log of the iterative deepening, this
file contains statistical information which is printed when the
\SAM\ stops (successfully or not).
\end{description}


Two further types of files are used within the SETHEO-system which
do not carry specific extensions.
The \SAM-output file is opened by the built-in predicate {\tt tell/1}
and contains all subsequent output as produced by other built-ins
(e.g., {\tt write} and {\tt dumplemma}).

The {\bf xptree} operator file contains  a translation table
of all operators together with their binding power. Its syntax is
defined in Section~\ref{sec:xtab-syntax}.


\chapter{The Basic Programs}

\section{plop}\label{sec:plop}   % ORTRUN: label ergaenzt
von altem setheo-manual



\ifcase \@ptsize\relax
   \font\wasy=wasy10 scaled 1000
   \font\deut=eufm10 scaled 1000
   \font\deuts=eufm8 scaled 1000
\or
   \font\wasy=wasy10 scaled 1095
   \font\deut=eufm10 scaled 1095
   \font\deuts=eufm9 scaled 1000
\or
   \font\wasy=wasy10 scaled 1200
   \font\deut=eufm10 scaled 1200
   \font\deuts=eufm10 scaled 1000
\fi



            % ORTRUN
\input{/home/ibens/Setheo/Users-Manual/sec-inwasm}       % ORTRUN
\input{/home/ibens/Setheo/Users-Manual/sec-wasm}         % ORTRUN
\input{/home/ibens/Setheo/Users-Manual/sec-sam}          % ORTRUN
\input{/home/ibens/Setheo/Users-Manual/sec-xptree}       % ORTRUN

\chapter{Additional Modules}
\section{The Environment}

SETHEOHOME

\section{clop}
{\bf Clop} combines the SETHEO compiler {\bf inwasm} with the
assembler {\bf wasm}. Given a {\em file\/}, first {\bf inwasm}
is invoked.
If the compilation phase terminated successfully, the
assembler {\bf wasm} is called with default parameters.
All parameters of {\bf clop} are passed to {\bf inwasm}.

\begin{description}
\item[Synopsis:]
clop [-$par_1$]\ldots[-$par_n$] {\em file}
\item[Input:] {\em file}.lop
\item[Output:] {\em file}.hex
\item[Temp:] {\em file}.s (not removed)
\item[Return Values:]
	{\bf Clop} returns $0$ is case of successful compilation.
	Otherwise the non-zero return value of the module in which
	the error occurred, is returned.
\item[Parameters:]
	All given parameters are passed to the {\bf inwasm}.
\item[Environment:]
	The variable {\tt SETHEOHOME} indicates the place where
	{\bf inwasm} and {\bf wasm} reside. If this variable is
	not defined, a default value is taken.
\end{description}

\section{setheo}
\input{/home/ibens/Setheo/Users-Manual/sec-setheo}       % ORTRUN
\section{xvtheo}
\section{xvdelta}
\section{delta}

\chapter{The Interactive SETHEO}

\chapter{File Formats}
\section{The FOL Syntax}
In\footnote{This text is from the OLD SETHEO MANUAL.}
 this part we describe the syntax of the input language, which is
accepted by the formula translator {\bf plop}.
A formula in that notation has to be in a file with the extension {\bf .pl1}.

Because the grammar of first order logic is well-known it seems
useless to repeat an inductive definition of the concept 
{\em ``formula''}. We rather represent the special PLOP
requirements i.e. the differences to the usual
standard in giving some comments on the individual
elements of the accepted language

\paragraph{Variables}
Variables are strings starting with a capital letter followed 
by arbitrary letters, digits, or underscore.

\paragraph{Constants}
Predicate-, function-, and object parameters are strings starting 
with a lower case letter followed by lower case letters, digits, or underscore. 
Predicate- and function arguments are to be included in round brackets.
Multiple arguments are to be separated by a comma ``,''.

\paragraph{Propositional Connectives}
The symbols for the propositional connectives are:

\begin{center}
\begin{tabular}{llll}
(1) &  {\tt $\sim$}  &   - &  not, \\
(2)  &	 {\tt \& } & -  & and, \\
(3)   &{\tt ; } & -  & or, \\
(4)   &{\tt -> } & -  & if ... then, \\
(5) & 	 {\tt <->}  & -  & iff. 
\end{tabular}                  
\end{center}

The binding preference is given by this order:
1 $>$ 2 $>$ 3 $>$ 4 $>$ 5 . 
Equal connectives are left associative i.e.,
\tt 
\mbox{a -> b -> c} \rm is to be read as \tt \mbox{((a -> b) -> c)} 
\rm .

\paragraph{Quantifiers}
The quantifiers are represented by
``{\tt
forall}'' and ``{\tt exists}''.
These symbols must not occur as parts of other names like constants
or variables.
The scope of a quantifier extends over the following propositionally 
complete subformula and ends there.
E.g. at the formula
\[\tt 
\mbox{forall X a -> b(X)}  \\
\rm
\]
the scope of the quantifier does not include {\tt b(X)}.
To bind {\tt X} by this quantifier one has to write as usual with parenthesis
\[\tt 
\mbox{forall X (a -> b(X)).}\\
\rm
\]
Occurences of free variables are understood by PLOP as bound
from the outside (universal closure).

\paragraph{Brackets}
Parenthesis are expressed by the usual {\em round brackets} only.
{\em Square brackets} must not be used in the whole input formula .

\paragraph{Special Characters}
All characters with ASCII value $<$ 32 are ignored and will be deleted
during the transformation.
The {\em space-character} can be omitted if the meaning is not affected and if
there cannot result any ambiguities. 
E.g. 
\[\tt 
\mbox{forallX} \\
\rm
\]
is recognized as a quantifier and a variable. 
In other cases a space character must be set, e.g.
\[\tt 
\mbox{forall X a(X) }\\
\rm
\]
cannot be written as 
\[ \tt 
\mbox{forall Xa(X).}\\
\rm
\]


Other characters as mathematical function symbols (especially in
infix notation) are not recognized, which perhaps could affect the
correctness of the transformation in some cases and are therefore to 
be treated carefully. 
If you cannot avoid 
them at any rate suppress the optimization concerning redundancies
and tautologies (option -nopti). 

\paragraph{Comments}
Comments are enclosed in C--like convention in ``{\tt /*}'' and ``{\tt */}''
and can extend over several lines. Nested comments are not allowed.
All comments are removed during the conversion.

%\paragraph{Size}
%The admissible size of the input file is limited and can be displayed
%by the option -defgro (``groesse1\_\,c'').
%If it is not sufficient for your example (e.g. a large program) 
%divide the input in several parts or change the define-statement for
%groesse1\_\,c in the source-code.


\section{The LOP Syntax}
\label{sec:lop}   % ORTRUN: label ergaenzt
von reinhold unter setheo
\begin{verbatim}
		-------------------------------------------
		LOP-SYNTAX FOR THEOREM PROVING APPLICATIONS 
		-------------------------------------------

The syntax fragment of LOP relevant for theorem proving applications is 
inductively defined in a BNF-type format. We are using quasi-quotation 
technique with the following conventions. 

1. `::=' is the definition symbol, and `|' is the logical `or' of the 
   meta-language. 
   
2. Strings of capital letters in the meta-language denote arbitrary strings 
   of object-level symbols. 
   
3. All other symbols in the definienda denote the respective 
   object-level symbols.

4. Concatenation of strings `s1' and `s2' is expressed by writing `s1^s2'.
   Accordingly, the empty string in the object language is encoded with 
   the symbol `^' in the meta-language. 
   
5. An empty space in the meta-language between strings designates arbitrary 
   many concatenations of empty spaces and comments in the object language,   
   including the empty word; for the definition of comments see `COMMENT' 
   below. Here, `\N' denotes `newline' and `\T' denotes `tabulator'.

6. The symbol `&' in the meta-language denotes an arbitrary string in the 
   object language not containing `\', including the empty string.


Definiens	Definiendum		

CLAUSE		::= GOAL     
		|   AXIOM  
		|   QUERY    
		|   RULE     

GOAL		::= <- TAIL .
		|   <- TAIL : CONSTRAINTS .

AXIOM		::= HEAD <- TAIL . 
		|   HEAD <- TAIL : CONSTRAINTS .
		|   HEAD <- .  
		|   HEAD <- : CONSTRAINTS .


QUERY		::= ?- TAIL .
		|   ?- TAIL : CONSTRAINTS .

RULE		::= LITERAL .
		|   LITERAL : CONSTRAINTS .
		|   LITERAL :- TAIL .
		|   LITERAL :- TAIL : CONSTRAINTS .

HEAD		::= LITERAL 
		|   LITERAL ; HEAD

TAIL		::= SUBGOAL
		|   SUBGOAL , TAIL

LITERAL		::= ATOM
		|   ~ ATOM

SUBGOAL		::= LITERAL
		|   BUILT_IN

ATOM 		::= CONSTANT 
		|   CONSTANT^TERMLIST

CONSTRAINTS	::= CONSTRAINT
		|   CONSTRAINTS , CONSTRAINT

CONSTRAINT	::= [ CTERMS ] =/= [ CTERMS ]

TERMLIST	::= ( TERMS )

TERMS	 	::= TERM
		|   TERM , TERMS

TERM		::= CONSTANT
		|   VARIABLE
		|   GLOBAL_VARIABLE
		|   LIST
		|   CONSTANT^TERMLIST

CTERMLIST	::= ( CTERMS )

CTERMS	 	::= CTERM
		|   CTERM , CTERMS

CTERM		::= CONSTANT
		|   VARIABLE
		|   STRUCT_VARIABLE
		|   LIST
		|   CONSTANT^CTERMLIST

LIST		::= [ ]
		|   [ TERM | LIST ]
		|   [ TERMS ]

CONSTANT	::= SMALLCHARACTER^WORD
		|   NAT
		|   "^&^"

VARIABLE	::= BIGCHARACTER^WORD
		|   _^WORD

STRUCT_VARIABLE	::= #^VARIABLE

GLOBAL_VARIABLE ::= $^VARIABLE

WORD		::= ^
		|   WORD^SMALLCHARACTER
		|   WORD^BIGCHARACTER
		|   WORD^CIPHER
		|   WORD^_

SMALLCHARACTER	::= a | b | c | ... | x | y | z

BIGCHARACTER	::= A | B | C | ... | X | Y | Z 

NAT 		::= CIPHER
		|   CIPHER^NAT

CIPHER		::= 0 | 1 | 2 | ... | 7 | 8 | 9

COMMENT		::= /*^&^*/
		| \N^#^&
		| \T^%^&


Additinally, LOP inputs must fulfill the following consistency condition, 
which is different from PROLOG. No predicate symbol must occur as a function 
symbol, and conversely. No predicate or function symbol must occur with more 
than one arity in the input.

GOALS are fanned and taken as top clauses
AXIOMS are fanned but not taken as top clauses
QUERIES are not fanned but taken as top clauses
RULES are not fanned and not taken as top clauses

CONSTRAINS are syntactical disequations of terms. They can be considered as 
additional subgoals of the clauses which are permanently checked. 

STRUCT_VARIABLES in constraints are treated as universally quantified 
variables of the whole clause. The constraint [X] =/= [f(#Y)] is
violated, when X is istantiated to a term of the form f(TERM).
	            

\end{verbatim}



\section{\SAM\ Assembler Syntax}
Assembler code for the SETHEO Abstract machine \SAM\ is typically
generated by the compiler {\bf inwasm}. Per default, \SAM-assembler
code is kept in files with the extension ``.s''. The assembler code
can be processed by {\bf wasm} in order to generate binary code for the
\SAM.
The assembler language of SETHEO is defined in a very straight forward
way. Each line of the input file can be empty, may contain
a {\em directive\/}, a {\em label}, or an {\em assembler instruction}.

\begin{verbatim}
assembler_code ::= line '\n'
	| line '\n' assembler_code

line ::= 
         | LABEL ':' 
         | '.' directive
         | statement 

LABEL ::= [A-Za-z][A-Za-z0-9_]*
\end{verbatim}

Comments are C-like and enclosed in {\tt /* \ldots */}.

\subsection{Directives}
Each directive starts with a '.', followed by an identifier and
arguments. Directives control the operation of the assembler and do
not generate any code.

\begin{verbatim}
directive ::=
            'include' STRING
          | 'equ' LABEL numexpr
          | 'dw'  exprlist
          | 'ds'  numexpr
          | 'start' numexpr
          | 'org' numexpr
          | 'symb' STRING ',' symbtype ',' numexpr
          | 'clause' numexpr ',' numexpr
          | 'red'   numexpr
          | 'optim'
          | 'noopt'

symbtype ::=
          'const' | 'var' | 'pred' | 'global' | 'gterm' | 'ngterm'

exprlist ::= expr
          | exprlist ',' expr

exprlist ::= numexpr
          | TAG numexpr

TAG ::=
        'const' | 'cref' | 'eostr' | 'var' | 'gterm'
        | 'ngterm' | 'crterm' | 'cstrvar'
\end{verbatim}

A numeric expression {\tt numexpr} can be a number, a label, or a
sum or difference of numeric expressions. Labels need not be defined
when they are used. However, if a label occurrs in a sum or difference,
it must be defined in order to yield a defined result.
Both {\tt +} and {\tt --} are left associative.

\begin{verbatim}
numexpr ::=
	LABEL
        | NUMBER
        | numexpr '+' numexpr
        | numexpr '-' numexpr

NUMBER  ::= [0-9][0-9]*
\end{verbatim}

\noindent{\bf include}
includes the named file {\tt STRING}. Nested includes are possible up
to 8 levels.

\noindent{\bf equ}
\noindent{\bf dw}
\noindent{\bf ds}
\noindent{\bf start}
\noindent{\bf org}
\noindent{\bf symb}
\noindent{\bf clause}
\noindent{\bf red}
\noindent{\bf optim, noopt}

\subsection{Statements}

\begin{verbatim}
statement ::=
	instruction0
        | instruction1 expr
        | instruction2 expr ',' expr
        | instruction3 expr ',' expr ',' expr

instruction0 ::=
        'stop' | 'told' | ...

instruction1 ::=
        'alloc' | 'isunifiable' | ...

instruction2 ::=
        'assign' | 'call' | ...

instruction3 ::=
        'eqpred' | 'porbranch' | ...

expr ::=
        NUMBER
        | LABEL
\end{verbatim}

\section{\SAM\ Machine Code Syntax}
Machine code for the SETHEO Abstract Machine \SAM\ is always located
in files with the extension ``.hex''. 
Each entry in this file occupies one line.
A one-character identifier is used
to select the appropriate type of data, contained in the current
line.
Blank lines, comments or extra spaces are not allowed.

The following grammar shows the definition of the \SAM\ Machine
Code:

\begin{verbatim}
file    ::= lines

lines   ::= line
          | lines line

line    ::= 
       ':' ident ':' address ':' data ':' string '\n'

ident   ::= 'C' | 'Y' | 'E' | 'S' | 'M' | 'N'


address ::= [0-9A-F]{8}
data    ::= [0-9A-F]{8}
string  ::= [A-Za-z_0-9]
\end{verbatim}

The following table shows the meaning of the fields, corresponding to
the given identifier:

\begin{center}
\begin{tabular}{lllll}
C & word in code area of \SAM &
	\SAM-code-address & contents & --- \\
Y & symbol &
	type of symbol & arity & printname \\
S & set start-address &
	\SAM-code-address & --- & --- \\
M & highest address in code area of \SAM &
	 0 & highest \SAM-code-address & --- \\
N & number of symbols &
	 0 & number of symbols & --- \\
\end{tabular}
\end{center}

\noindent{\bf Notes:}
\begin{enumerate}
\item
A ``highest memory address'' (identifier: ``M'')
directive must be placed prior to any code words (``C'') in order
to be effective. Otherwise, the size of the code-area of the \SAM\ is
determined by a default value (or the {\bf sam} parameter -code).
\item
The relative line number of a ``symbol'' directive determines the index
of that symbol in the symbol table. Therefore, the order of these
directives must not be changed and their number must not be 
increased or decreased.
\item
Special characters in a symbol are encoded by: \verb+\OOO+
where {\tt OOO} is the 3-digit octal value of the ASCII code of that
character. E.g. \verb+\007+ is for {\tt BELL}.
\item
The maximal length of a symbol is limited to 42 on the 
generation side ({\bf wasm}), and to 200 in the reader of
the abstract machine.
This limit includes special characters (see above). Each such character
contributes 4 to the total length of the symbol.
\item
all other directives may be intermixed and arbitrarily changed in order.
\item
A file with machine code must at least contain one line.
\end{enumerate}

\section{Syntax Definition of a Proof Tree}
xptree .y file

\section{The Operator Translation Table}

\section{The \SAM\ Logfile}

\chapter{Logic Programming with SETHEO}
\section{Introduction}
\section{Global Variables}
von diss
\section{Built-in Predicates}
\section{List of Built-Ins}

The SETHEO system provides a number of built-in predicates
for programming and control purposes.
Most of them are used in a similar way to PROLOG, but several
differences exist.

The following list describes all built-in predicates, available
with SETHEO Version V3.3
This list is likely to be extended in future versions of SETHEO.
Therefore, changes are marked clearly.
Furthermore, built-ins can be implemented by the user, using
SETHEO's C-interface as described in Section~\ref{sec:C-interface}.
However, no support is provided in this case.

Syntactically, a built-in predicate is an {\tt ATOM} 
(in the LOP-file), corresponding to the syntax definition in
Section~\ref{sec:lop-syntax}.
Whereas most built-in predicates are in prefix notation
(of the form $p$ or $p(t_1,\ldots,t_n)$), some are available
in infix notation. These built-ins (mostly arithmetic) can be
used 
A built-in predicate must always 

Notes:
\begin{itemize}
\item
currently all references to the index are done via the LCB,
i.e., by giving the literal-number, and NOT the term itself.
\end{itemize}

\subsection{genlemma/2}

\begin{description}
\item[Synopsis:]
	{\tt genlemma(Nu1, Nu2)}
\item[Parameters:]\ \\
	\begin{description}
	\item[{\tt Nu1}: Number]\ \\
	If this number is $\neq 0$, the generation of a unit-lemma
	is considered. Otherwise, this predicate just succeeds.
	\item[{\tt Nu2}: Number]\ \\
	If a unit-lemma is stored in the index, this number will be
	stored together with the lemma. This information can the
	be used when unit-lemmata are being used (e.g., by {\bf uselemma/1}).
	\end{description}
\item[Result:]\ \\
	This built-in predicate always succeeds. Memory overflow
	will result in a run-time error.
\item[Low Level Name:]
	{\tt genlemma argvector}

\item[Description:]\ \\
This built-in generates a unit-lemma {\tt H<-.} for the
head {\tt H} of the current clause, if all of the following
conditions hold:
\begin{itemize}
\item
both arguments are instantitated to a number,
\item
the value of the first argument is $\neq 0$,
\item
The solution of the head of the current subgoal does not involve
any reduction steps above the current node in the tableau.
(This means, that we really have a unit-lemma).
\item
The current head (with its current instantiation) is not subsumed
by any unit-lemma in the lemma store.
\end{itemize}

The lemma will be annotated by the value of the second argument
(must be a number).

If the unit-lemma subsumes one or more unit-lemmata in the lemma-store,
these unit-lemmata are deleted.

\item[Side-effects:]\ \\
If the {\bf sam(1)} is called with the inline-command flag {\tt -lemmatree},
the proof tree for the current lemma (if it gets stored in the index)
is appended to the proof-tree file.

\item[Notes:]\ \\
The {\bf genlemma/1} built-in must be the last subgoal of a given
clause or contrapositive.
Otherwise, correctness of SETHEO is not ensured.

\item[Example1:]\ \\
\begin{verbatim}
q(X,Y) <- p(X),r(X,a),genlemma(1,5).
\end{verbatim}

In this example, a unitlemma {\tt q(X,Y)<-.} --- with the current substitutions of {\tt X} and {\tt Y} --- is generated and labeled with the number $5$.

\item[Example2:]\ \\
\begin{verbatim}
q(X,Y) <- p(X),r(X,a),getdepth(D),genlemma(1,D).
\end{verbatim}

In this example, the generated unit-lemma is labeled with the current
depth. When lemmata are selected for usage ({\bf uselemma/1}), one
can select only those unit-lemmata which have been generated on a lower
depth.
\end{description}

\subsection{genlemma/2}

This built-in generates a unit-lemma {\tt H<-.} for the
head {\tt H} of the current clause, if all of the following
conditions hold.
The lemma will be annotated by the value of the second argument
(must be a number).

\begin{itemize}
\item
both arguments must be instantitated to a number,
otherwise, a run-time error is generated.
\item
the value of the first argument is $\neq 0$,
\item
The solution of the head of the current subgoal does not involve
any reduction steps above the current node in the tableau.
(This means, that we really have a unit-lemma).
\item
The current head (with its current instantiation) is not subsumed
by any unit-lemma in the lemma store.
\end{itemize}

In all cases, this built-in predicate succeeds or generates a run-time
error.

If the unit-lemma subsumes one or more unit-lemmata in the lemma-store,
these unit-lemmata are deleted.

The {\bf genlemma/1} built-in must be the last subgoal of a given
clause or contrapositive.

Example:
\begin{verbatim}
q(X,Y) <- p(X),r(X,a),genlemma(1,5).
\end{verbatim}


\subsection{genulemma/3}

This built-in generates a unit-lemma {\tt L<-.} for a literal
{\tt ~L} in the current clause. Its number is given as the first argument.
The lemma is generated, if all of the following
conditions hold.
The lemma will be annotated by the value of the third argument
(must be a number).

\begin{itemize}
\item
all arguments must be instantitated to a number,
otherwise, a run-time error is generated.
\item
the value of the first argument is $\neq 0$,
\item
The current instantiation of the given literal is not subsumed
by any unit-lemma in the lemma store.
\end{itemize}

In all cases, this built-in predicate succeeds or generates a run-time
error.

If the unit-lemma subsumes one or more unit-lemmata in the lemma-store,
these unit-lemmata are deleted.

Notes:
\begin{itemize}
\item
The range of the first argument is NOT checked
\item
it is not checked, if the solution of the given literal required
reduction steps above the current node in the tableau.
\item
All built-in literals (including this one) count as literals
\end{itemize}



Example:
\begin{verbatim}
q(X,Y) <- p(X),genulemma(2,1,5),r(X,Y).
\end{verbatim}

generates a lemma {\tt p(a)<-.}, if X is instantiated to {\tt a}.


\subsection{uselemma/1}

This built-in tries to use lemmata from the lemma-store.
This built-in must be given within a special clause, as shown
in the following example:

\begin{verbatim}
...
p(a,b) <-..
p(b,c) <-..

p(_,_) <- uselemma(5),fail.

p(e,f) <-.
..
\end{verbatim}

If the clauses are not reordered, SETHEO first tries the alternatives
{\tt p(a,b)..} and {\tt p(b,c)..}.
Then it extracts all lemmata from the lemma store, which
\begin{itemize}
\item
are unifiable with the current subgoal $\neg p(\ldots)$, and
\item
are annotated by a numeric value which is {\em smaller} than the
argument of {\bf uselemma}.
\end{itemize}

These lemmata are pushed upon the stack as additional alternatives
which are tried, before {\tt p(e,f)..} are tried.

This built-in always succeeds. If the pushed lemmata and the remaining
alternatives are to be tried, this built-in must be directly followed
by a {\bf fail}.

\subsection{getnrlemma/1}

This built-in unifies the given argument with the number
of lemmata which are currently stored (and not deleted) in the
lemma store.

Example:
\begin{verbatim}
p(X) <- getnrlemmata(Y), X > Y.
\end{verbatim}

\subsection{dumplemma/0}

This built-in dumps all lemmata in the lemma-store
onto the current output-file (as set by {\bf tell/1}), or on
stdout.
Lemmata are printed in a LOP-like syntax. Therefore, the output can directly
be processed by {\bf inwasm(1)}.
Lemmata which have been marked deleted are preceeded by a {\bf \#}.

Example:
\begin{verbatim}
printlemma <- tell("out"),write("Lemmata:\n"),
              dumplemma,told.
\end{verbatim}

\subsection{dlrange/2}

This built-in dumps all lemmata in the lemma-store
with annotated values between the first and the second parameter
(inclusive; both parameters must be instantiated to numbers)
onto the current output-file (as set by {\bf tell/1}), or on
stdout.
Lemmata are printed in a LOP-like syntax. Therefore, the output can directly
be processed by {\bf inwasm(1)}.

Example:
\begin{verbatim}
printlemma <- tell("out"),write("Lemmata:\n"),
              dlrange(3,4),told.
\end{verbatim}

\subsection{delrange/2}
This built-in deletes all lemmata in the lemma store
with annotated values between the first and the second parameter
(inclusive).
Both parameters must be instantiated to a number.

\subsection{checklemma/1}
check, if the lemma would be stored at all:
argument unified with 0 if not.
\subsection{checkulemma/2}
similar to genulemma
check, if the lemma would be stored at all:
argument unified with 0 if not.

\subsection{addtoindex/2}
add literal unconditionally to index

\begin{verbatim}
<-... checklemma(X),addtoindex(1,X,5).

same as 

<-... genlemma(1,5).
\end{verbatim}

\subsection{clearindex/0}
clears the entire index
\subsection{clearlemma/2}
delete all entries, which are unifiable with the given literal

\section{Non-Backtrackable Counters}

\subsection{counters/1}
This built-in allocates $N$ non-backtrackable counters. They are
initialized to the numeric value $0$.
Note, that this built-in must be used prior to any other computation
(but after the {\bf galloc}).

\subsection{setcounter/2}

The built-in {\bf setcounter(I,N)} destructively sets the counter $I$
to the numeric or constant value $N$. A run-time error occurs, if
the parameters are not of the required type.
The validity of $I$ is {\em not\/} checked.

Example:
\begin{verbatim}
p(U) <- X is $D + U, setcounter(3,X).
\end{verbatim}

\subsection{getcounter/2}

The built-in {\bf getcounter(I,N)} unifies the second parameter $N$
with the value of the $I$'s counter.

Example:
\begin{verbatim}
p(U) <- getcounter(3,X), U is X + 1.
\end{verbatim}



\section{How to \ldots}
\subsection{Selection of Start Clauses}

Per default, SETHEO takes all clauses which contain negative
clauses only, as possible start clauses.
In LOP syntax, these clauses have the following form:

\[ \mbox{\tt <-} L_1, \ldots, L_n. \]

with atoms (non-negated literals) $L_i$.
These clauses are also referred to as {\em queries}.

The Model Elimination Calculus, however, allows to select
an {\em arbitrary\/} clause as start clause.
This can be of interest if the conjecture to be proven is of the form
$A_1,\ldots,A_n \rightarrow B$. If a true goal-oriented serach is intented,
the search should start with this clause.
On the other hand, axioms like {\tt <-equal(X,succ(X)).} should in general
not be used as a starting clause.

Cases like this can be accomplished by SETHEO without affecting
the satisfiability of the formula.
Here, we describe a way to (a) disable a query, and to
(b) convert an arbitrary clause into a query.
Please note, that at least one query must be present in the formula in order
to preserve satisfiability.

\paragraph{Disabling a Query.}
A query (as shown above) can be disabled by replacing that clause
with the following one (for any $1 \leq i \leq n$):

\[ {\tt ~}L_i \mbox{\tt <-} L_1, \ldots, L_{i-1},L_{i+1},\ldots,L_n. \]

This rule cannot be used as a starting clause by SETHEO. nevertheless,
all contrapositives are being generated.

\paragraph{Converting a Clause into a Query.}
Given a clause
\[ H_1 ; \ldots ; H_m \mbox{\tt <-} L_1, \ldots,L_n. \]
with atoms (non-negated literals) $L_i,H_i$.
Then this clause can be converted into a query by replacing this
clause by:
\[ \mbox{\tt <-}
{\tt ~}H_1, \ldots ,{\tt ~}H_m, L_1, \ldots,L_n. \]

For this clause, all contrapositives are generated.


\paragraph{Example:}

{\tt irgend ein GRP bzw. Bledsoe}

\subsection{Answer Substitutions}
\subsection{Bounds and iterative deepening}
\subsection{Restart Model Elimination}

\subsection{The SETHEO C-Interface}
%%%%%%%%%%%%%%%%%%%%%%%%%%%%%%%%%%%%%%%%%%%%%%%%%%%
%   SETHEO MANUAL
%	(c) J. Schumann, O. Ibens
%	TU Muenchen 1995
%
%	%W% %G%
%%%%%%%%%%%%%%%%%%%%%%%%%%%%%%%%%%%%%%%%%%%%%%%%%%%
\section{Additional Modules}

The additional modules are shell scripts which call the basic
programs. The environment variable {\tt SETHEOHOME} should indicate
the place where \plop, \inw, \wasm\ and \SAM\ reside\footnote{See
Section~\ref{sec:environ}.}. If this variable is not defined, a
default value is taken. 

\subsection{clop}

\Clop\ combines the SETHEO compiler \inw\ with the
assembler \wasm. Given a file, first \inw\
is invoked. All of the given parameters are passed to the \inw.
If the compilation phase terminated successfully, the
assembler \wasm\ is called with default parameters.
The following is the synopsis of \clop:

{\tt clop} [-$\mbox{\tt par}_1$]\ldots[-$\mbox{\tt par}_n$] {\tt file}

The input file for \clop\ has to be given. This must be a file with
extension {\bf .lop}. \Clop\ will produce a temporary file with
extension {\bf .s} and generate the output file with extension {\bf
.hex}. The temporary file is not removed automatically. The {\bf .hex}
file can be used as an input for the \SAM.

\Clop\ returns $0$ in case of successful compilation.
Otherwise the non--zero return value of the module in which
the error occurred, is returned.



\subsection{setheo}\label{sec:setheo}

\Se\ calls the whole SETHEO theorem prover to prove a formula
contained in a  file  {\it file.lop\/} using default options. \Se\ in turn
calls \inw, \wasm\ and \SAM. 
%In case, a proof is found, a proof tree is 
%generated which can be displayed using \xp. 

This is the synopsis:
\begin{verbatim}
setheo file
\end{verbatim}

The name of the input file must have the extension {\bf .lop}. During
computation temporary files with the extensions {\bf .s} and {\bf
.hex} are generated. These are not removed automatically. Two output
files are generated. The output file with the extension {\bf .tree}
can be used as an input for \xp\ to visualize a found
proof. The output file with the extension {\bf .lop} contains the
output of the prover \SAM, the same as written onto the screen.

The following options are used (for details concerning these options
see Sections~\ref{sec:inwasm}, \ref{sec:wasm} and \ref{sec:sam}): 
\begin{description}
      \item[Compiler:]
           {Constraints are generated  with {\bf --cons}.   The  output
            code is optimized with {\bf --opt}.}
      \item[Prover:]
           {Iterative deepening over the depth of  the  tableau  is
            performed   ({\bf --dr}).   Constraints  are  generated  and
            checked with {\bf --cons}.}
\end{description}

If the user wants to modify this default set of options, the
user  should  refer  to the commands \inw\ and \wasm\ for compilation
and to the \SAM\ for the prover itself.



\subsection{delta}
The {\sc Delta}-iterator is a preprocessing module which generates
unit-lemmata from the input formula in a bottom up way. These unit-lemmata
are then added to the input formula and then main top-down search for the
proof is started.
These additional unit-lemmata often abbreviate long (and difficult to find)
sub-tableaux and thus often leads to a dramatic decrease of search-time
for the overall system.

{\sc Delta} has a number of optional parameters and is invoked:

\begin{verbatim}
delta [-all] [-horn] [-nonhorn] [-noentry] [-debug] [-limit number] [-termsize number]\\
    [-query] [-nosubs] [-large] [-cputime number] [-lastlevels number] [-termdepth number]\\
    [-delta number] [-level number] [-pred predsymb arity] ... file
\end{verbatim}

{\sc Delta}
is a preprocessing tool for \Se. It generates new single-literal
clauses (facts) in a bottom-up fashion, using \Se.
A delta iteration method is used (see below). 
 The newly generated facts are appended to the original formula in the file
{\em file}.lop to obtain the output file {\em file}.out.lop.

The orginal file {\em file}.lop remains unchanged.
The number of newly generated clauses per level and the total
amount of time needed for compilation and execution of SETHEO
is printed at the end of the run.

The options for {\sc Delta} are:

\begin{description}
\item[-all]
With this option, unit-lemmata for all predicate symbols of the
formula are generated. Furthermore, it is automatically determined,
if the formula is Horn or Non-Horn (see {\bf -[non]horn}).

\item[-horn]
generate positive single-literal clauses only. (default)
\item[-nonhorn]
positive and negative facts are generated. These are
added as additional queries to
the formula unless {\bf -noentry} is given.
\item[-noentry]
all newly generated negative facts are added as such
to the formula.
\item[-level {\em number}]
the given number of iterations are performed, or the given limit
of newly generated clauses has been reached. (default: level = 1)
\item[-limit {\em number}]
generate only less or equal than 
{\em number\/}
new clauses. (default: limit = 999)
\item[-termsize {\em number}]
generate (and keep) newly generated clauses only, if the total
number of symbols in the terms of each clause is
less than or equal than {\em number\/} symbols.
Default: do not restrict the size.
\item[-termdepth {\em number}]
generate (and keep) newly generated clauses only, if the maximal
nesting level of symbols in the terms of each clause is less than or equal
{\em number\/}. Default: do not restrict the depth of the terms.
\item[-level {\em number}]
use the given number (default: 3) as the depth-bound for the
\SAM\ during the delta-iteration.
\item[-pred name arity]
this option specifies the predicates, for which
new clauses are to be generated. The arity of the predicate must be given.
If the option {\bf -all} is not set,
at least one predictate must be specified.
\item[-debug] causes
{\sc delta}
not to remove the temporary files after processing. Instead, their name
is given.
\end{description}

For {\sc Delta} the following algorithm is used:
\begin{enumerate}
\item
For each predicate given in the argument list or for all predicate
symbols, special code
is generated and added to the original formula (``special query'').
Furthermore, code for the meta-level control is generated.
\item
This formula is compiled and executed by the \SAM.

\item
The formula is entierly searched with a given low depth bound (which can be
set using {\bf -delta {\em number\/}}, default: 3).
During the run, the newly generated single-literal clauses are stored
in the Unit-lemma index.
The unit-clauses are kept only, if their term size and term depth do
not viaolate the given bounds, and as long the limit of generated lemmas
is reached.
\item
If a stopping condition is met (number of levels or number of
newly generated clauses is exceeded), the remaining clauses
are added to the original formula and written into the file
{\em file}.out.lop.
Then
{\sc Delta}
prints the execution and compilation times and exits.
\item
Otherwise, the search starts again (Step~3), whereby all unit-lemmata
generated in the previous levels can be used during the search.
\end{enumerate}

\subsection{xvtheo}
{\tt johann}
\subsection{xvdelta}
{\tt johann}


\chapter{Odds and Ends}
\section{Default Values}
von specs/limits nehmen
\section{Limitations of the SETHEO System}
\documentstyle[12pt,a4]{article}


\title{Limiations of the SETHEO system}
\author{Johann Schumann}
\date{9.1.95}

\begin{document}

\maketitle

\begin{abstract}
-abstract-
\end{abstract}

\noindent{\bf Keywords:}  

\noindent{\bf Action: } {\tt BUGFIX  CHANGE EXTENSION}

\section{Limitations in the WASM}

\begin{tabular}{|l|r|l|l|c|}
Name & value & Description & Type & Action \\
\hline
M\_SYMB\_LENG & 42 & Maximal length of a symbol & D & T\\
MAXID & 15000 & Max.~number of symbols & D & E \\
LABSTART & 200 & Max.~number of ``equ''s & D & E\\
NAMELENGTH & 30 & Max.~length of a label & D & T\\
PATHLENGTH & 100 & length of path+filename (incl .s) & D & F\\
MAX\_INCL & 8 & max.~level of includes & D & E\\
MAXWORDS & 2000 & nr.~data in one readlist-part & D & F\\
MAXNAMES & 200 & max.~length of one orbranch (optimizer) & D & F\\
MAXDTREE & 10000 & number of elements allocated in one step & D & -\\
TOKENLENGTH & 255 & max. length of a token (=YYLMAX) & D & T \\
YYMAXDEPTH & 150 & stack-limit of yacc & D & E \\
\hline
\end{tabular}

\section{Limitations in the SAM}

\begin{tabular}{|l|r|l|l|c|}
Name & value & Description & Type & Action \\
\hline
int &$-2^{15},\ldots,2^{15}-1$ & numeric values & F & T\\
IDMASK & $2^24-1$ & Max.~number of symbols & F & F \\
M\_SYMB\_LENG & 42 & Maximal length of a symbol & D & T\\
DEF\_CODE & 100000B & default size of code area & C & E \\
DEF\_STACK & 100000B & default size of stack area & C & E \\
DEF\_HEAP & 300000B & default size of heap area & C & E \\
DEF\_TRAIL & 100000B & default size of trail area & C & E \\
DEF\_C\_STACK & 100000B & default size of constraint stack area & C & E \\
DEF\_PU\_TRAIL & 4000B & default size of trail for constraint unification & D & E \\
DEF\_SYMBSIZE & 1000 & max.~nr.~of symbols & D & E \\
DEF\_DEPTH & 2147483647 & default depth limit & C & - \\
DEF\_MIN\_DEPTH & 3 & default starting depth & D & - \\
DEF\_DEPTH\_INC & 1 & default increment for depth & D & - \\
DEF\_INF & 2147483647 & default inference limit & C & - \\
DEF\_MIN\_INF & 3 & default starting inference & D & - \\
DEF\_INF\_INC & 5 & default increment for inference & D & - \\
DEF\_LOCINF & 2147483647 & default local inference bound limit & C & - \\
DEF\_MIN\_LOCINF & 3 & default starting local inference bound & D & - \\
DEF\_LOCINF\_INC & 1 & default increment for local inference bound & D & - \\
CLNU\_LEMMA & 20000 & start of clause-numbers for unit-lemmata & D & - \\
MAXPREDSYMB & 10000 & max.~number of predicate symbols+1  & D & F \\
\hline
\end{tabular}
\end{document}



\section{Known Bugs}
\section{How to Report a Bug}

\chapter{Installations Guide}
\section{Introduction}
\section{Getting the Binaries}
\section{Getting the Sources}
\section{Installing the SETHEO System (Binaries)}
\section{Installing the SETHEO System (Sources)}

\appendix
\chapter*{Annotated Bibliography}

\chapter*{Glossary}

\chapter*{Index}

\end{document}
