%%%%%%%%%%%%%%%%%%%%%%%%%%%%%%%%%%%%%%%%%%%%%%%%%%%
%   SETHEO MANUAL
%	(c) J. Schumann, O. Ibens
%	TU Muenchen 1995
%
%	%W% %G%
%%%%%%%%%%%%%%%%%%%%%%%%%%%%%%%%%%%%%%%%%%%%%%%%%%%

\newcommand\bild[5]{{\tt #1 #2 #3 #4 }\label{#5}}

\section{Introduction} 
XSETHEO is a graphic user interface for the SETHEO theorem prover.
This document introduces you to the user--interface by giving a 
guided tour through all stages of automatically proving an actual theorem.
To get an impression of the interface have a glimpse at Figure~\ref{fig:whole-thing}.
%\begin{figure}[htbp]
%\begin{picture}(100,170)
%\put(0,0){\framebox(180,170)}
%\end{picture}
%\caption{Full fledged XSETHEO}\label{fig:whole-thing}
%\end{figure}

\bild{xsetheo01.ps}{12cm}{17cm}{Full fledged XSTHEO}{fig:whole-thing}

\section{The Example--Theorem} 
The theorem we are about to prove is a classic example\footnote{Chang and Lee, 
Symbolic Logic and Mechanical Theorem Proving, Academic Press, Orlando, 1973.} from ATP.
It alleges that from the axioms of group theory it follows that any
left--identity element is also a right--identity element. 
To recollect the fundamentals of group theory, cast a look at the definition coming up.
\\ \\
{\bf Definition}
\begin{quote}
A {\it group\/} consists of 
\begin{quote}
a set $\cal G$ 
and 
a function $* : S \times S \rightarrow S; (s1,s2) \mapsto s1 * s2$. 
\end{quote}
Additionally, two axioms must hold.
\begin{enumerate}
\item There exists an element $e$ in $\cal G$ such that: \\
\begin{itemize}
\item[a.]$e * x = x$ for any element $x$ in $\cal G$. 
(We call $e$ a left--identity element.) 
\item[b.] For any element $x$ in $\cal G$ there exists an element $x'$ in $\cal G$ such that 
$x * x' = e.$ \\ 
\end{itemize}
\item For all $x, y, z$ from $\cal G$ it holds that $(x * y) * z = x * (y * z)$.
\end{enumerate}
\end{quote}

An example of a group is $(Q\setminus \{0\},*)$, the set of rational numbers with
standard multiplication. However, $(Q,*)$ is not a group, since axiom 1.b. cannot
be satisfied\footnote{1 is the only candidate for a (left--) identity but
$0 * x' = 1$ holds for no $x'$ in $\cal G$.}.
It also follows from the group axioms that  for any group there is exactly
one identity element, but we are not to take advantage of this knowledge.
As far as we are concerned, there
exist one or more left--identity elements, which may or may not be right--identities.
Now, rendering axiom 1 in logical notation results in:
\begin{center}
	$\exists y (\forall x  (y * x = x) \:\wedge\: \forall x  \exists x' (x' * x = y))$
\end{center} 
Axiom 2 reads in logical notation as follows:
\begin{center}
	$\forall x \forall y \forall z  ((x * y) * z) = (x * (y * z))$
\end{center} 
To prove that indeed every left--identity is a right--identity we proceed
indirectly, i.e. we assume that there is an element, which is a left--identity 
but not a right--identity.
So what we want to do is to state that there is an element that is characterised by axiom 1
but is not a right--identity.
In logic notation this can be expressed as:
\begin{center}
	$\exists y (\forall x  (y * x = x) \:\wedge\: \forall x  \exists x' (x' * x = y) \:\wedge\:  \sim \forall x  (x * y = x))$
\end{center} 
which is equivalent to: There is a left-identity which is not a right--identity.
These formulae are almost, but not quite, what you can feed the theorem--prover with.
For one, since there are no provisions for equality in SETHEO's underlying calculus,
we have to paraphrase equality by introducing the 3--place predicate
{\it product\/}. The intended
interpretation for {\it product\/}$(x,y,z)$ is obviously $x * y = z$. 
Introducing the modification to our input formulae they
turn out like this:
\begin{center}
$\exists y (\forall x  \mbox{ {\it product\/}}(y,x,x) \:\wedge\: \forall x
\exists x' \mbox{ {\it product\/}}(x',x,y) \:\wedge\:  \sim \forall x
\mbox{ {\it product\/}}(x,y,x))$
\end{center} 
Also, the formulae must be translated from predicate logic to LOP,
which is a language not unlike Prolog but with some extensions. In
LOP--syntax\footnote{See Section~\ref{sec:lop-syntax} for a description of
the LOP--syntax.} the input formulae look like this:
\begin{center}
\begin{verbatim}
product(inverse(X),X,identity) <- .
product(identity,X,X) <- .
product(X,V,W) <- product(X,Y,U) , product(Y,Z,V) , product(U,Z,W).
product(U,Z,W) <- product(X,Y,U) , product(Y,Z,V) , product(X,V,W).
<- product(a,identity,a).
\end{verbatim}
\end{center}
Before you enter the formula, read on!



\section{XSETHEO--Window}\label{sec:xsetheo}

ORTRUN: Diese Section beschreibt die XSETHEO-Version aus
/usr/local/jessen/bin/xsetheo. xsetheo muss noch an die neuen
Parameter fuer inwasm und sam angepasst werden, und dann muss diese
Beschreibung geaendert werden.

XSETHEO is a X--window--based interface for the SETHEO
theorem--prover. To call the main window into existence, change to
SETHEO's example subdirectory and type {\it xsetheo\/}. XSETHEO's main
window should now appear on the screen. It looks similar to
Figure~\ref{fig:xsetheo-main-window}. 
%\begin{figure}[htb]
%\begin{center}
%\begin{picture}(131,95)
%\input{/u1/johann/papers/setheo/man/xsetheo_main_pic}
%\end{picture}
%jfjfj
%\caption{XSETHEO's Main Window}\label{fig:xsetheo-main-window}
%\end{center}
%\end{figure}

\bild{win01.ps}{9cm}{15cm}{XSETHEO's Main Window}{fig:xsetheo-main-window}

The main--window is subdivided into two parts; the upper part is
called a {\it panel\/}, the lower part a {\it tty--window\/}. On the
panel you notice a couple of small ovals; these are called {\it
buttons\/} and this is quite literally what they are: moving the
cursor to these buttons and clicking the left mouse--button will call
forward the action selected just as pushing buttons of a vending
machine does.  

The buttons on the panel labelled {\it File\/}, {\it Show\/}, {\it
Compile\/}, {\it Prove\/} and {\it Proof--tree\/} represent, applied
from left to right, the seqence of steps from the input formula in
input syntax to a graphical representation of the proof--tree,
provided  there is a proof and SETHEO finds it.  

\paragraph{File}
The {\it File\/}--button is used to load an input formula. Clicking
the {\it File\/}--button you get a small window, the {\it Formula
Files\/} window. At top of this window the name of the current directory
is printed, followed by the list of filenames and
subdirectories. Using the scroll--bar on the right you can see the
whole contents. Below there is a line where you can enter
filenames.\\ 
In SETHEO's examples subdirectory there should be a file
{\it GRP003-1.lop\/}. This is identical to the input formula we
developed above; so you don't need to type it again.
Look for the file {\it GRP003-1.lop\/} by using the scroll--bar and
click on it with the left mouse--button. The filename will appear on
the line. Either you could type the filename {\it GRP003-1.lop\/} on
the line. Now click on the {\it LOAD\/}--button. There are two
possibilities: If anything went wrong, e.g. you misspelled {\it
GRP003-1.lop\/} or you have not been in SETHEO's examples subdirectory
when calling XSETHEO, you might see an {\it alert--popup\/} on your
screen. An alert--popup is a subwindow with a huge black arrow
pointing into the window. Usually an alert tells you that you made a
mistake or that you are about to invoke an action that you better
think about twice. There is no way just to ignore an alert. You cannot
go on unless having chosen one of the options in the alert--
subwindow. \\  
If, on the other hand, things are going smoothly, the file {\it
GRP003-1.lop\/} is loaded into XSETHEO and the {\it Formula Files\/}
window disappears.

\paragraph{Show}
You can change the loaded formula before starting XSETHEO's
prove mechanism. Click on the {\it Show\/}--button. A small window
will appear on the screen, the {\it Formula\/} window. In this
window the input file {\it GRP003-1.lop\/} is shown. On the right side of the
window there is a scroll--bar. Move the cursor into the {\it Formula\/}
window. Now you can type or delete parts of the input formula. The
original file {\it GRP003-1.lop\/} will not be changed, but XSETHEO will
use the changed formula. 

\paragraph{Compile}
Before starting the prove mechanism, the formula must be translated
into a code which can be interpreted by SETHEO. Also some
simplifications of the input formula can be done. This is the job of
the compiler. SETHEO's compiler is divided into two parts, {\bf
inwasm} and {\bf wasm\/}.\\ 
First {\bf inwasm\/} is called to make some simplifications: Often a
formula to  be proven conveys a lot of information that will not be
needed for a particular proof. But proving a theorem is essentially a
search process and surplus clauses will generally result in a vast
amount of futile search. Thus there has been some research into
limiting the set of clauses to those that stand a chance to be needed
for the proof. In theorem--proving this effort is called
pre--processing. The best possible kind
of preprocessing would be to first find a proof and then throw away
the surplus clauses, but this obviously misses the underlying idea
completely. Preprocessing is meant to effectively reduce the number of
clauses without engaging in a search--process.\\
After the preprocessing {\bf wasm\/} translates the simplified
input formula into machine code. If you are interested in further
details on {\bf inwasm\/} and {\bf wasm\/} look at
Sections~\ref{sec:inwasm} and \ref{sec:wasm}.\\
Now click on the {\it Compile\/}--button. The {\it Formula Compile\/}
window will appear on the screen. At the top the name of the formula
to be compiled is given. Also there is a list of switches which
represent options for the compilation. If you look at the list you
will notice that one of the options is already ticked\footnote{To
tick or untick an option, clicking a mouse button after having moved
the arrow into the respective box will do the job.}. Standardly {\bf
inwasm} will be called with the {\it -cons\/}--option for generating
all possible constraints. Now click on the {\it GO!\/}--button. The {\it
Formula Compile} window will disappear and the compiler will be
started. The result of the compilation phase will be displayed in the
{\it tty--subwindow\/} below the panel. 

\paragraph{Prove}
SETHEO is a first order predicate logic theorem prover featuring
completeness, consistency, speed, and a variety of proof
modi. Clicking the {\it Proof\/}--botton the {\it Prover\/} window
will appear on the screen. Again the name of the input formula and
some switches are given. Additional three bounds can be chosen, the
{\it Depth Bound\/}, the {\it Inference Bound\/} and the {\it Local
Inference Bound\/} (to get more information on SETHEO's switches and
bounds read Section~\ref{sec:sam}). Two of the options are already
ticked and the bounds are initialized with their default values. This
is SETHEO's standard proof mode. The prover is activated by clicking
at the button marked {\it PROVE!\/}, the result will be displayed in
the {\it tty--subwindow\/} below the panel. In our case, i.e. the
group--example {\it GRP003-1\/}, the result looks like
Figure~\ref{fig:GRP003-1}.

\begin{figure}

\begin{verbatim}
SAM V3.3 Copyright TU Munich (April 7, 1995)

Options : -dr -cons GRP003-1 

using antilemma-constraints
using regularity-constraints
using tautology-constraints
using subsumption-constraints

Start proving...

-d:    3  time     < 0.01 sec  inferences =       56  fails =       54

                    ******** SUCCESS ************

Number of inferences in proof      :       10
        - E/R/F/L                  :       10/       0/       0/       0
Intermediate free variables        :        6
Intermediate inferences            :       10
Intermediate open subgoals         :        5
Generated antilemmata              :        0
Number of unifications             :       56
        - E/R/F/L                          56/       0/       0/       0
Total number of fails              :       54
    - unification                  :       29
    - depth bound                  :       24
    - constraints                  :        1
        - anl/reg/ts               :        0/       0/       1
Number of generated constraints    :       53
        - anl/reg/ts               :        0/      20/      33
Instructions executed              :      340
Abstract machine time (seconds)    :   < 0.01
Overall time          (seconds)    :     0.04

Genlemma entered/no-unitlemma      :        0/       0
    - GEN/w.match/INST/w.match     :        0/       0/       0/       0
    - stored/Nr. in Index          :        0/       0
Uselemma entered/no/Nr. pushed     :        0/       0/       0

\end{verbatim}
\caption{SETHEO's output after proving the group--example {\it
         GRP003-1\/}.}
\end{figure}\label{fig:GRP003-1}

First SETHEO's defaults, the chosen options and the name of the proven
file are given. Is is also printed which types of constraints are
used. The line 
\begin{verbatim}
-d:    3  time =     0.01 sec  inferences =       56  fails =      54
\end{verbatim}
is caused by the iterative deepening and means that at depth 3 in 0.01
seconds 56 inferences have been executed which led to 54 fails. This
was sufficient to find a proof so depth 4 was not tried. The following
statistics give an overview on the various counters within SETHEO.

\paragraph{Proof--tree}
If you are interested in the proof you can have a look at the actual
proof the prover came up with. Click at the {\it Proof--tree\/} button
to have a canvas displayed with the graphical representation of the
prooftree on it. For your convenience the proof tree is reproduced in
Figure~\ref{fig:proof-tree}. 
%\begin{figure}[htb]
%\begin{center}
%\begin{picture}(110,65)
%\input{ /u1/johann/papers/setheo/man/proof-pic}
%\end{picture}
%jfjfj
%\caption{Proof tree}\label{fig:proof-tree}
%\end{center}
%\end{figure}

\bild{tree01.ps}{9cm}{15cm}{Proof tree}{fig:proof-tree}

A canvas is yet another type of window in the X--windows menagerie. As
the name suggests a canvas is a window type for drawing objects in
like, e.g. prooftrees. The interesting thing, however, is that a
canvas can be much larger than the actual window. So with a large
prooftree you might only see a part of the tree at a given time, but
you can have the canvas moved under your window like you can move a
probe under the microscobe. The way to do this is to move the little
square in the top left corner of the canvas.\\
Intuitively the root of the prooftree is what you want to show and the
leaves of the tree are the hard facts. Every node in the tree stands
for a literal. You might be surprised because the rootnode is a
literal called {\it $\sim$query\_\/} which is not part of 
the input file. The reason is the following: Like in prolog notation a
clause without a head is a query. LOP syntax allowes more than one
query in an input file, and all of them must be taken into account
when building up a prooftree. To give the prooftree nevertheless a
clear rootnode SETHEO adds an artificial query. That means the clauses
without head literal get the head literal {\it query\_\/} and the clause
{\it $\sim$query\_\/} is added to the input file.\\
If you select a literal by clicking on its node you will
get further information on the usage of this literal within the
prooftree. In this prooftree you will either find the information {\it
entry literal\/} or something like {\it InfNr 2}, {\it ext(3,1)}. {\it
entry literal\/} means that this literal was unified with the negation
of its parent node\footnote{In the model elimination calculus a clause
can only be used if one literal of the clause can be unified with the
negation an open subgoal.}. So this path is closed. SETHEO gives a
number to each clause of the input file and to each subgoal of a
clause. The information {\it ext(3,1)} at a literal means that the
negation of this literal was unified with the first subgoal of the
third clause. {\it InfNr 2} means that this was the second inference
step.\\
Clicking again unselects the literal, i.e.\ the further
information disappears. At the top left corner there are four
buttons. The {\it Tree\/}--button provides a pull--down menu. Here you
can select or unselect literals, too.

SETHEO first showed that $a^{-1^{-1}} * e = a$ using the facts $a^{-1^{-1}} * a^{-1} =e$ and $a^{-1} * a = e$
and $ e * a = a$ and the instantiation of rule 5: if $a^{-1^{-1}} * a^{-1} =e$ and $a^{-1} * a = e$ and $e * a = a$ then
$a^{-1^{-1}} * e = a$.\\
SETHEO actually showed this twice.\\
It than showed that $a * e = a$ using $a^{-1^{-1}} *e = a$ twice, the fact $e * e = e$, and the instantiation of rule 4:
if $a^{-1^{-1}} * e = a$ and $e * e = e$ and $a^{-1^{-1}} * e = a$, then $a * e = a$.
So what did we show? We presumed that there is (at least one) left--identity
which is not a right--identity. Under this assumption we could show that
any left--identity picked at random is a right--identity~\footnote{We actually
picked $e$ but there is nothing special about $e$ to make it behave
differently from any other left--identity}. So we come up with a 
contradiction. Since there is nothing wrong with the axioms, our assumption 
that there exists a left--identity which is not a right--identity
must be a mistake. This concludes the proof.

Proving this theorem obviously was a very easy task for SETHEO and it only used a tiny fraction of its 
reasoning power. So we did not need built-in predicates and there did not occur any reduction steps in the proof.
Reduction steps are needed, e.g. for proving that Sue will be at your party from knowing that:
(1) if Martin does not come, Sue will be there, and
(2) if Sue does not come, Martin won't come.
%\ssubsection{A more advanced example}
%Let's do one more example and concentrate on the proof rather than on SUN's
%windowing environment~\footnote{Actually, the environment is called SUNVIEW,
%being a more advanced version of X--windows, the standard of the Unix
%X--OPEN group.}.
%The theorem we want to prove is $$There is no largest prime.$$
%Or, which is the same thing, there exist infinetly many primes.
%The proof is attributed to Euclid (300 B.C.). Again, the proof is
%indirect: Assume there is a largest prime and call it $P$. Then, since
%the property of being prime is decidable for any number you can
%collect all primes smaller than $P$, form the product of all these
%primes including $P$.






%
%
%
%
%
%
%
%
%
%
%
%
%
%
%

\paragraph{Further hints}
\begin{itemize}
\item X--windows can be moved and/or enlarged. Clicking the {\it
right\/} button on the black bar atop each window, holding the button
and moving the mouse will do the job.
\item The canvas settings allow also for a generally more dense
display of the prooftree possibly using a smaller font. This might be
useful if you want to get a birds--eye view on the whole proof. You
can also open a second canvas with the the same proof on it using one
to get an overview and the other to look at details.
\item If you want to prove theorems the files of which are not in the
current directory, you can use the {\it ..(Go up a level)\/} entry on
the {\it Formula Files\/} window to enter the paths.
\item Be sure to set your path variable {\tt SETHEOHOME}\footnote{See
Section~\ref{sec:environ}.}so you can access the XSETHEO modules after
installation.
\end{itemize}
%\ssubsection{BUGS}
%There are problems with the prooftree output  by SETHEO. Most of the times the information is not
%sufficient for the construction of a tree. If a prooftree can be constructed there are frequent
%mistakes in the succession of literals and their instantiation. Often variables are not fully
%instantiated as this would afford an additional run trough the prooftree. $\not\models$
