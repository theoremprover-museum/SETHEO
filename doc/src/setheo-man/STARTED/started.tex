% new getting started
%
After you have fetched and installed SETHEO on your system,
you are ready to go.
In this chapter, we present three tutorials which shall enable you to
successfully use the SETHEO system.

The tutorials are suited for different levels of experience in
automated theorem proving and thus have  different prerequisits.

The first tutorial starts with a problem, given in mathematical
notation. In a step by step fashion, all actions necessary from entering
the formula to understanding SETHEO's proof will be described.
The only prerequisits needed to understand this tutorial is a general
knowledge of logic and mathematics.

The second tutorial aims at readers who are somewhat familiar 
with the Model Elimination calculus and the basics of automated theorem
proving. Given a formula in first order logic (or as a set of clauses)
(in our tutorial, it will be the same formula as in the first tutorial),
the user is guided to explore the different possibilities to set
parameters and watch the influence the setting has on the search
for the proof.

The third tutorial assumes familiarity with the basics of SETHEO
and logic. It describes the steps which are necessary to go from
given proof tasks from an application towards the usage of SETHEO.
For each step, practical hints will be given and problems which might
(or will) occur will be discussed.

A number of exercises for the user of SETHEO concludes this chapter.

\section{Tutorial 1}

Assume you have been given the following problem to prove\footnote{
	PELLAAR: citation}:
\begin{quote}
Let $p$ and $q$ be two binary relations.
For the relations we know the following properties:
$p$ and $q$ are transitive. Furthermore, $q$ is symmentric.
Additionally, we know that for each pair of elements, this pair is
either in relation $p$ or in relation $q$.

Now, we have to show that at least one of thr relations $p$ or $q$
is total.
\end{quote}

As a first step, we have to write down this problem in a formal
notation, expanding the definitions of ``transitive'', ``symmetric'',
and ``total''. 
A relation $r$\footnote{
	With $XrY$ we denote that $X$ is in relation $r$ with $Y$.
	}
 is transitive, if
\[ \forall X,Y,Z  XrY \wedge YrZ \rightarrow XrZ,\]
it is symmetric, if
\[ \forall X,Y  XrY \rightarrow YrX,\]
and total (i.e., all elements are in the relation $r$), if
\[ \forall X,Y  XrY.\]

Now, given the formal definitions of all properties, you are invited
to take a sheet of paper and try to prove the theorem.

Did you make it?
If not, don't be desparate. There is a proof in Table~\ref{tab:started:nonob-proof} at the end of this Chapter and furthermore, there is SETHEO to
support you.

In order to use SETHEO to automatically find a proof, you have to write
down the problem in first order predicate logic.
In general, such a formula is of the form
\[ 
{\cal A}_1 \wedge \ldots  \wedge {\cal A}_1 \rightarrow {\cal T}
\]
where ${\cal A}_i$ are Axioms (in our case, the properties of the relations
$p$ and $q$, and ${\cal T}$ is the theorem we want to show.
When we insert the definitions, we obtain:

\vspace*{4cm}

Before starting SETHEO, two further steps must be done:
the formula must be converted into Conjunctive Normal Form (CNF)
and the syntax must be changed.
Although the first step can be performed automatically, we will describe
how our formula is converted into CNF.
If you want to do this manually, proceed to Section~\ref{sec:started:2cnf}.

\subsection{Manual Preparation of the formula}
Since SETHEO does not understand \LaTeX\ and infix mathematical
notation, we have to modify the syntax according to the following
rules.
\begin{itemize}
\item
SETHEO does not understand infix notation (well, there are some exceptions).
Therefore, instead of $XrY$ we have to write our binary relation $r$
as a predicate of arity two: {\tt r(X,Y)}\footnote{
	Actually, there is quite a number of different ways to
	convert the syntax. E.g., one could have written instead:
	{\tt isInRelation(r,X,Y)}. For further details see our third
	tutorial.}.
This predicate has the meaning: $\mbox{{\tt r(X,Y)}} \equiv {\rm\bf T}$
iff $X$ and $Y$ are in the relation $r$.

\item
All variables must start with an uppercase letter or an underscore
e.g.,{\tt X, Y, \_99, \_}, all
predicate-, function-, and constant symbols must start with a lowercase
letter (e.g., {\tt r, f(X)}.
\item
The connectives and qunatifiers must be changed into an ASCII representation
(e.g., $\forall$ must be written as {\tt forall}).
\end{itemize}

For details on the syntax see Chapter~\ref{chap:file-formats}.


\subsection{Manual Preparation of the formula}
The transformation of an arbitrary formula into CNF is described in detail
in many textbooks. Here, we refer to \cite{Lov78,Chang75}.
The transformation consists of essentially two activitites: 
removing all logical operators which do not belong to CNF, and
elimination of the quantifiers by skolemization.

Furthermore,

