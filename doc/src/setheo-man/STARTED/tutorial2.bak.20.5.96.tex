\documentstyle[12pt]{article}
\def\SAM{SAM}
\begin{document}
\newtheorem{definition}{Definition}

For this tutorial, we assume the reader to be familiar with the
basics of the Model Elimination Calculus and SETHEO. The topic
of this tutorial is the influence of refinements and extensions
of the basic Model Elimiantion Calculus on the search behavior.
Again, we use an example to illustrate the techniques implemented in
SETHEO. Here, we use the same example as in the previous tutorial.
For reference, its clauses (in LOP syntax) are given below in 
Figure~\ref{tab:tut2:pellaar.lop}. This example, is of course, no
challenge problem for Automated Theorem Provers
(it used to be one about 10 years ago), but it has some nice 
properties.


\begin{table}[htb]
\begin{center}
\begin{verbatim}
(1)  p(X,Z) <- p(X,Y),p(Y,Z).
(2)  q(X,Z) <- q(X,Y),q(Y,Z).
(3)  q(Y,X) <- q(X,Y).
(4)  p(X,Y) ; q(X,Y)<-.
(5)  <- p(c_1,c_2).
(6)  <- q(c_3,c_4).
\end{verbatim}
\end{center}
\caption{LOP clauses of the Example (file: MSCXXXX.lop)}
\label{tab:tut2:pellaar.lop}
\end{table}

For our first experiment, we run SETHEO, using the basic Model Elimination
Calculus, and perfomring depth-first iterative deepening over the
depth of the tabelau (A-literal depth). Using different bounds
will be discussed below in Section~\ref{sec:tut2:bounds}.
Such a run will be accomplished using the two commands:

\begin{center}
\begin{verbatim}
inwasm MSCXXXX
sam -dr MSCXXXX
\end{verbatim}
\end{center}

\begin{figure}[htb]
\begin{center}
\begin{verbatim}
\end{verbatim}
\end{center}
\caption{SETHEO-output for MSCXXXX.lop)}
\label{fig:tut2:pellaar.pure.log}
\end{figure}

After some time, SETHEO finds a proof, consisting of 18 Inferences,
two of which are Model Elimination Reduction steps
(see Figure~\ref{fig:tut2:pellaar.pure.log}  for SETHEO's output).
A graphical representation
of the tree can be seen in Figure~\ref{fig:tut1:pellaar} in Tutorial~1.
For us of interest now, however, is not the proof itself, but the amount
of search involved to find the proof.
This is reflected in the following sizes:

\begin{description}
\item[Abstract Machine Time] This number gives the amount of time (CPU user 
time),
the \SAM, SETHEO's Abstract Machine needed to find the proof. 
However, on different machines, and even on different runs of the same
problem, other values can be returned. Therefore, this figure should be
taken as a first approximation only.
\item[Total number of inferences $n_i$] gives the overall number of times,
a unification was tried during search (by trying to perform an
Extension or Reduction step). This number directly reflects the amount
of search performed, but does not take into account, how long each
attempted unification takes.

\item[Number of Fails $n_f$] is the number of times, a unification failed,
or a bound has been reached. This number is again splitted into those
two values.
\end{description}

The output SETHEO produces at the end of the run gives these values
for the overall search. Additionally, these values are also
printed, after the entire search space has been exhausted with a given
bound. Figure~\ref{fig:tut2:pellaar.pure.log} shows these figures
for the increasing depth-bound ({\tt -d}). As can be seen clearly,
the size of the search space grows substantially with each iteration
(at exponentially).

In the following, we present several improvements of the Model
Elimination Calculus and SETHEO's search procedure and have a look
at these figures.
The list of improvements discussed in this tutorial
is only a small selections of techniques integrated into SETHEO.
For a list of all techniques and the corresponding command-line
switches see Chapter~\ref{chap:basic-modules}.
Furthermore, we do {\em not\/} present the theoretical background
nor any formal definitions or theorems. For such issues see
e.g.\ \cite{LSBB89,LMG94,Letzdiss,Mayrdiss}.

\section{Refinements}
\subsection{Regularity}

One of the most powerful refinement of the pure Model Elimination
Calculus is its restriction to {\em regular\/} tableaux.

\begin{definition}[regular tableau]
A Model Elimination Tableau $\cal T$ is regular, if and only if on
each path from the root to a leaf, no literal occurs more than once.
\end{definition}

One can show (cf. \cite{LSBB89,LMG94}) that for each closed tableau
there also exists a closed regular tableau, i.e., we don't loose any
proofs if we are searching for regular tableaux only.

Figure~\ref{fig:tut2:reg-tab} contains two tableaux for the same
subgoal (taken from our example). The left tableau is not regular,
because the literal $\neg p(a,b)$ occurs twice in it; the right tableau
is regular. In our case, it is easy to see, why the left tableau is
not regular: in attempt to solve the goal $\neg p(a,b)$, an extension
step into the symmetry clause (clause number (3)) is made, yielding
a new subgoal $\neg p(b,a)$. Then this clause is used again to
obtain $\neg p(a,b)$. Comparing this subgoal to the original one,
we have gained nothing! Therefore, we can leave these two steps out,
yielding a regular tableau. Restricting the search to certain kinds
of tableaux thus reduces the search space considerably.

\begin{figure}[htb]
-figure-
\caption{Model Elimination Tableau and Regular Model Elimination Tableau}
\label{fig:tut2:reg-tab}
\end{figure}

Within SETHEO, there are two different ways to enforce regularity:
a {\em direct\/} check which is performed, as soon as an Extension
or Reduction step is tried, and the generation of regularity constraints.

\noindent{\bf Direct Regularity Check.} This check can be
activated by calling {\bf inwasm} with the option {\tt -eqpred},
yielding the following sequence of commands:

\begin{center}
\begin{verbatim}
inwasm -eqpred MSCXXXX
sam -dr MSCXXXX
\end{verbatim}
\end{center}

When we look at the result SETHEO produces, a considerable 
reduction in the amount of search necessary to find the proof
can be seen.
Table~\ref{tab:tut2:results.regularity} shows typical figures, compared
to the basic calculus.

\begin{table}[htb]
\begin{center}
\begin{tabular}{|l|r|r||r|r||r|r|r|r|r|}
\hline
Method & $t_{\SAM}$ & $n_i$ & $n_f$ & 
	$n_i^3$ & $n_i^4$ & $n_i^5$ & $n_i^6$ & $n_i^7$ \\
\hline\hline
basic & 2.34 & 160606 & 25252 &
	25 & 145 & 2525 & 26262 & 282828 \\
\hline
{\tt -eqpred} & 2.34 & 160606 & 25252 &
	25 & 145 & 2525 & 26262 & 282828 \\
\hline
{\tt -reg} & 2.34 & 160606 & 25252 &
	25 & 145 & 2525 & 26262 & 282828 \\
\hline\hline
\end{tabular}
\end{center}
\caption{Search Space for Basic ME, and implementations of
regularity}
\label{tab:tut2:results.regularity}
\end{table}


\noindent{\bf Regularity Constraints.} 
A more elegant and powerful method is realized, using syntactical
inequality constraints. These constraints of the form
$ [ X \not = \{t_1,\ldots,t_n\}] $  are attached to variables (here $X$)
and checked permanently. As soon, as $X$ gets bound to a term,
its constraints are evaluated. If a violation occurs (i.e.,
$X$ gets instantiated to one of the $t_i$), backtracking
occurs within the \SAM\ Abstract Machine.
Looking at our above example, the regularity-checker would
generate the constraints $[ X,Y ] \not = [ b,a ]$

{\tt to be done}

The results for running SETHEO with regularity constraints, which
is performed by the commands:
\begin{center}
\begin{verbatim}
inwasm -reg MSCXXXX
sam -dr -reg MSCXXXX
\end{verbatim}
\end{center}
is shown in Table~\ref{tab:tut2:results.regularity}.
Although the figures for this example are quite similar for these
two methods of enforcing regularity, it is advisable to use the
latter method, because it is (a) more powerful, but costs a little
more overhead, and (b) neatly fits into other Calculus Refinements
described below. As a general hint, it is {\em always\/} advisable to
activate the enforcement of regular tableaux. Note, that the
default parameters for {\bf setheo} incorporates this option.

\subsection{Tautology and Subsumption Constraints}

A Model Elimination Tableau can be further restricted, namely that
no instance of a clause in the tableau is a {\em tautology\/}, and
that no instance of a clause is subsumed by another clause.
Again, when these restrictions are enforced, no proofs are lost.
In SETHEO, these conditions are checked permanently using
the constraint mechanism. Constraints for checking for tautology and
sumbsumption are generated during the run of the {\bf inwasm} compiler,
if it is invoked with the {\bf -cons} or {\bf -taut -subs} options.
Looking at our example, we can easily detect instantiations of clauses
which will result in tautological clauses or clauses which are subsumed
by others. Let's have a look at clause (4) in 
Figure~\ref{tab:tut2:pellaar.lop} which expresses the symmetry of $p$.
If we instantiate $X$ to the same value as $Y$ (e.g., by calling this
clause with a subgoal $\neg p(a,a)$), we obtain the following
instatiation of that clause in the tableau
{\tt p(X,X) <- p(X,X)}. Obviously, this clause is tautologial and
its application does not lead us anywhere. Therefore, we can {\em forbid\/}
such an instantiation by adding the constraint {\tt [X] =/= [Y]}
to the clause.

In the case of the symmetry clauses, we obtain a similar sitation:
If two of the variables in that clause $X,Y,Z$ get instantiated to the
same value, the resulting clause is tautological, namely if
$X = Y$ or $Y = Z$. This can be forbidden, using two constraints.

Additionally, there exist instantiations, where this clause is subsumed
by clause (5) {\verb+<- p(c_1,c_2)+}, namely $X$ is instantiated
to $c\_1$ and $Y$ to $c\_2$, or $Y$ to $c\_1$ and $Z$ to $c\_2$.
In both cases, the resulting instance of the clause is subsumed by
clause (5).
Therefore, the following constraints are automatically added to
clause (1):

\begin{center}
\begin{verbatim}
p(X,Z) <-  p(X,Y), p(Y,Z)
	: [Y] =/= [Z]	/*  tau  */, 
	  [Y] =/= [X]	/*  tau  */, 
	  [X,Y] =/= [c_1,c_2]	/* sub by 5 */, 
	  [Y,Z] =/= [c_1,c_2]	/* sub by 5 */.
\end{verbatim}
\end{center}

A list of all generated constraints can be obtained, when the {\bf inwasm}
is called with the option {\tt -lop}. In that case, a file 
{\tt {\em file}\_pp.lop} is generated which contains all contrapositives
of all clauses, the generated constraints as well as other information.

A variety of other refinements, working with constraints have been
developed and integrated into SETHEO, but these will not be discussed
here. For details see Chapter~\ref{chap:manpages} and the Glossary.
Comparative run-times with our example is shown in 
Table~\ref{tab:tut2:results.constr}.

\begin{table}[htb]
\begin{center}
\begin{tabular}{|l|r|r||r|r||r|r|r|r|r|}
\hline
Method & $t_{\SAM}$ & $n_i$ & $n_f$ & 
	$n_i^3$ & $n_i^4$ & $n_i^5$ & $n_i^6$ & $n_i^7$ \\
\hline\hline
basic & 2.34 & 160606 & 25252 &
	25 & 145 & 2525 & 26262 & 282828 \\
\hline
{\tt -taut} & 2.34 & 160606 & 25252 &
	25 & 145 & 2525 & 26262 & 282828 \\
\hline
{\tt -subs} & 2.34 & 160606 & 25252 &
	25 & 145 & 2525 & 26262 & 282828 \\
\hline
{\tt -subs -taut -reg} & 2.34 & 160606 & 25252 &
	25 & 145 & 2525 & 26262 & 282828 \\
\hline\hline
\end{tabular}
\end{center}
\caption{Search Space for Basic ME, and several comile-time
refinement}
\label{tab:tut2:results.constr}
\end{table}

\subsection{Deletion of Links}

The search space for finding a Model Elimination proof is spanned
by the {\em connections\/} between the literals of clauses. The more
connections, the larger the search-space (which increases at least
exponential with their number). Therefore, any method which can
remove connections without sacrificing completeness of the search procedure
can lead to considerable decrease in run-time needed to find a proof.

The method built in into SETHEO tries to remove connections which
are of no use. This option can be activated by calling the {\bf inwasm}
with {\tt -linksubs} or {\tt -rlinksubs}.
In this tutorial, wo won't describe in detail, how this preprocessing
step works. Rather, we explain with selected clauses of our example
what happens.
Let us consider clause (3) (Symmetry of $q$). The second literal
(the tail literal) has several connections: some going to positive
$q$ literals in other clauses, and a connection to the head-literal
of our clause (3). The latter connection --- called a {\em back-connection\/} ---
is of interest. If clause (3) has been called from a subgoal
$\neg p(a,b)$, then our tail literal gets $\neg p(b,a)$. If we now follow
the back-connection, we end up with a new subgoal $\neg p(a,b)$ which
is exactly our original one. In this case, following the back-connection
does not make any sense\footnote{
	In this case, our situation can be detected by the
	regularity-contraints as well, but in general, these
	two methods are independent from each other.
	}.
Thus this connection can be removed without
harming completeness.

Within the rules of transitivity (clauses (1) and (2)), one
back-connection in each clause can be removed as well. 
Transitivity clauses are in particular harmful, because they have two
subgoals (the longer a clause, the more search space it induces), and
these subgoals always introduce new variables (in our case variable $Y$).

Results of experiments with this option are shown in
Table~\ref{tab:tut2:results.linksubs}.

\begin{table}[htb]
\begin{center}
\begin{tabular}{|l|r|r||r|r||r|r|r|r|r|}
\hline
Method & $t_{\SAM}$ & $n_i$ & $n_f$ & 
	$n_i^3$ & $n_i^4$ & $n_i^5$ & $n_i^6$ & $n_i^7$ \\
\hline\hline
basic & 2.34 & 160606 & 25252 &
	25 & 145 & 2525 & 26262 & 282828 \\
\hline
{\tt -taut} & 2.34 & 160606 & 25252 &
	25 & 145 & 2525 & 26262 & 282828 \\
\hline
{\tt -subs} & 2.34 & 160606 & 25252 &
	25 & 145 & 2525 & 26262 & 282828 \\
\hline
{\tt -subs -taut -reg} & 2.34 & 160606 & 25252 &
	25 & 145 & 2525 & 26262 & 282828 \\
\hline\hline
\end{tabular}
\end{center}
\caption{Search Space for Basic ME, and removal of connections}
\label{tab:tut2:results.linksubs}
\end{table}


\subsection{Dynamic Constraints: Antilemmata}

-Ortrun

\section{Additional Inference Rules}

folding up and down

\section{Reordering}
\subsection{Static Clause and Subgoal Reordering}

\subsection{Dynamic Subgoal Reordering}

-Ortrun

\section{Bounds}
\label{sec:tut2:bounds}.

Until now, we only have considered the depth-bound (A-literal depth)
and iterative deepening over that bound as a means to obtain completeness.
SETHEO, however, features a large variety of bounds.
In general, the bounds in SETHEO follow the rules:
\begin{itemize}
\item
Each bound can be used separatedly or combined with other bounds.
\item
Iterative deepening is performed with one bound at each time only.
\item
Several bounds are not complete when used alone (e.g., the term complexity,
or the maximal number of open subgoals).
\item
Bounds can be obtained and set using specific LOP-builtin predicates.
These predicates are described in Chapter~\ref{chap:logprg}.
\end{itemize}

For those bounds which allow for iterative deepening, a fixed start-value
is used. The increment can be set using teh command-line options.
The following table~\ref{tab:tut2:bounds:list} shows a short
defintion of all bounds available in the current version and their main
features.

\begin{table}[htb]
\begin{center}
\small
\begin{tabular}{|l|l|c|c|l|c|}
\hline
Name & option & complete? & iterate? & iterate option & type \\
\hline\hline
depth & {\tt -d} & Y & Y & {\tt -dr} & L \\
inference & {\tt -i} & Y & Y & {\tt -ir} & G \\
local inference & {\tt -li} & Y & Y & {\tt -lir} & L \\
weighted depth & {\tt -wd} & Y & Y & {\tt -wdr} & L \\
\hline
term complexity & {\tt -tc} & N & N & - & - \\
free variables & {\tt -fvars} & N & N & - & - \\
open subgoals & {\tt -??} & N & N & - & - \\
\hline
\end{tabular}
\end{center}
\caption{Search Bounds for SETHEO}
\label{tab:tut2:bounds:list}
\end{table}

Most commonly used are the depth bound, the inference bound, and the
weighted depth bound which has been designed to combine the advantages
of the depth and inference bound. The choice of a bound dramatically
influences the beaviour of SETHEO, as is depicted in
Table~\ref{tab:tut2:bounds.results}. However, there does not seem
to be a ``universally good'' iteration bound for all possible problems.

\begin{table}[htb]
\begin{center}
\begin{tabular}{|l|r|r||r|r||r|r|r|r|r|}
\hline
Method & $t_{\SAM}$ & $n_i$ & $n_f$ & 
	$n_i (d=3)$ & $n_i (d=4)$ & $n_i (d=5)$ & $n_i (d=6)$ & $n_i (d=7)$ \\
\hline\hline
{-tt -dr} & 2.34 & 160606 & 25252 &
	25 & 145 & 2525 & 26262 & 282828 \\
\hline
{\tt -ir} & 2.34 & 160606 & 25252 &
	25 & 145 & 2525 & 26262 & 282828 \\
\hline
{\tt -locir} & 2.34 & 160606 & 25252 &
	25 & 145 & 2525 & 26262 & 282828 \\
\hline
{\tt -wdr} & 2.34 & 160606 & 25252 &
	25 & 145 & 2525 & 26262 & 282828 \\
\hline\hline
\end{tabular}
\end{center}
\caption{Search Space for different completeness bounds}
\label{tab:tut2:bounds.results}
\end{table}

Therefore, an appropriate completeness bound must be selected upon information
about similar proof tasks within the same domain, and possible additional
information about the proof tasks.

Another approach to deal with this problem is to explore several different
completeness bounds in parallel in a competitive way (as SiCoTHEO does):
on each processor, SETHEO tries to solve the problem, using a different
bound (or a combination of bounds). The processor which finds a solution
first, wins and stops the other processors.
For details on this approach see \cite{Sch96ppai}.

\end{document}
