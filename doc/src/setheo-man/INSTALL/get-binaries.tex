%%%%%%%%%%%%%%%%%%%%%%%%%%%%%%%%%%%%%%%%%%%%%%%%%%%
%   SETHEO MANUAL
%	(c) J. Schumann, O. Ibens
%	TU Muenchen 1995
%
%	%W% %G%
%%%%%%%%%%%%%%%%%%%%%%%%%%%%%%%%%%%%%%%%%%%%%%%%%%%
\section{Getting the Binaries}\label{sec:get-bin}

The SETHEO system can be obtained via an anonymous ftp--server. The
ftp--server provides a 
package containing the binaries of the basic SETHEO programs for Sun
Sparc computers, the available shell scripts, manpages, example
problems, a users registration form and a file with
installation hints. 

To get this package you have to connect the ftp--server either by
\begin{verbatim}
        ftp 131.159.8.35
\end{verbatim}
or by
\begin{verbatim}
        ftp flop.informatik.tu-muenchen.de
\end{verbatim}
If the connection is established the ftp--server will ask for your
name and your e--mail address:
\begin{verbatim}
        Name: anonymous
        Password: <your-e-mail-address>
\end{verbatim}
The next thing you have to do is change to the right directory, set
the transfer mode to {\tt binary} and get the {\tt setheo.tar.gz}
file: 
\begin{verbatim}
        ftp> cd pub/fki
        ftp> binary
        ftp> get setheo.tar.gz
\end{verbatim}
You can get a short SETHEO info as well. After this leave the
ftp--server: 
\begin{verbatim}
        ftp> get setheo.info
        ftp> bye
\end{verbatim}

Now you have a file {\tt setheo.tar.gz}. First uncompress it:
\begin{verbatim}
        gunzip setheo.tar.gz
\end{verbatim}
After this the name of the file has changed to {\tt setheo.tar}. This
is a package containing directories and files. Before unpacking it
create a SETHEO directory and move the package into:
\begin{verbatim}
        mkdir setheo
        mv setheo.tar setheo/setheo.tar
\end{verbatim}
Now unpack the {\tt setheo.tar} file:
\begin{verbatim}
        tar xf setheo.tar
\end{verbatim}
You can delete the file {\tt setheo.tar} if you want. Before you try
to start the SETHEO theorem prover read
Section~\ref{sec:inst-bin}. 
