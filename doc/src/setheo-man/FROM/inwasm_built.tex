\documentstyle[12pt,a4]{article}


\title{New handling of built-ins: inwasm}
\author{Johann Schumann}
\date{21.5.93,7.6.93,23.6.93}

\begin{document}

\maketitle

\begin{abstract}
This document specifies the changes necessary for a new and unified
handling of built-in predicates in the inwasm.

\end{abstract}

\noindent{\bf Keywords:}  built-ins, inwasm, code-generation

\noindent{\bf Action: } {\tt BUGFIX, EXTENSION}

\section{Solution Provided}

In the current version of inwasm, for most built-ins,
``built nr, argv'' has been generated, where {\tt nr} specified the number
of the built-in predicate.


\subsection{Problems with Solution Provided}
	
In Version V3.0, all built-in instructions of the SAM are addressed
via their names. Therefore the old type of code generation is obsolete.

\section{Detailed Description}
\subsection{Informal Description}

The table, containing all built-ins are expanded, containing now:
\begin{enumerate}
\item
the LOP-name of the built-in
\item
its arity
\item
the SAM-instruction name of the built-in
\item
the function for generating code for this particular instruction,
or for this group of instructions
\item
the type of built-in, like deterministic, non-deteriminstic
\item
visibility, i.e., if the predicate can be used directly from LOP
\end{enumerate}

All built-ins which are coded into the inwasm are entered into the
symbol table in function {\tt enter\_built\_ins()}. For each built-in,
the function {\tt enter\_built} is called to enter that built-in.

Furthermore, a built-in predicate can be defined on the LOP
level.
Similar to PROLOG, one should be able to issue (within the
compiling directives of a module):

{\tt ?- export(name/arity,lowlevel\_name,type).}

The latter type of entering is performed within the scanner and parser.
Further work is necessary to specify the exact operation of the feature
wrt.\ the modularity and control concept of MPLOP.

Reading in built-ins from the file ``.inwasmrc'' and the usage of
libraries have become obsolete now.


\section{Specification of Algorithms and Data Structures}

\subsection{Data Structures}

The symbol table element has been modified slightly to:

\begin{verbatim}
typedef struct sytabel {
	char name[NAMELENGTH];	/* symbol name (string) 	*/
	int symbnr;             /* index in symboltable 	*/
	int modunr;             /* number of modul		*/
	int type;               /* type of symbol 		*/
	int arity;              /* store arity  		*/
	int index;		/* index  different usage 	*/
	int codenr;		/* index in setheo symbtab, code*/
	int (*bfun)();		/* pt to built-in function 	*/
	char *b_instr_name;	/* name of sam-instruction for
				/* built-in			*/
	built_in_type	b_type;	/* type of built-in		*/
	struct preli *loclist;  /* list of occurrencies 	*/
	struct sytabel * next;  /* pt to next elem 		*/
	} syel;
\end{verbatim}

\section{Required Changes}
\subsection{Required Changes in inwasm}

Each built-in is entered using the function:
\begin{verbatim}
int enter_built(lop_name,general_type,built_type,arity,cg_fun,sam_name)
char *lop_name;         // name of the built-in as seen from LOP
int general_type;       // CONSTANT or BUILTIN
built_in_type built_type;   // Tyep of built-in:
                               * NONE
                               * DETERMINISTIC
                               * NONDETERMINISTIC
                               * INTERNAL
\end{verbatim}

Several constants and standard built-ins are entered into the
symbol table at the beginning. This is done via the function
{\tt enter\_built\_ins()} in {\bf built\_ins.c}.
For other purposes in the code-generation, it is essential, that
the first built-ins and constants ALWAYS appear in the same order,
since their index is defined in {\bf defines.h}. This index is
used in several places.

When code is generated for the subgoals, it is checked, if
it is an built-in. If yes, the code-generation function ``b\_fun'' as
entered in the symbol table is called with a pointer to that predicate.
Default code generation functions will be described in the following section.

\subsection{Default functions for code-generation}
\subsection{b\_generic}
This code-generation function is defined as {\tt b\_generic(predtype *)}.
For a deterministic built in, it generates the following code:
({\tt sam\_name} is the name of the SAM-instruction for that built-in.

\begin{verbatim}
...
sam_name      avXXXXXX       % if its arity is greater 0
...
sam_name                     % otherwise
\end{verbatim}

\subsection{b\_error}
This function causes an error message to be printed and is used for
internal constant symbols, for which no code must be generated.

\subsection{b\_assign}
This function generates code for the assignment to global
variables (:=).

\subsection{b\_{\em relop}}
The functions generate code for the relational operators
$<, >, \leq, \geq$.

\subsection{b\_globeval}
This function generates code for the evaluating assignment
(:is) to global variables.

\subsection{b\_eval}
This function generates code for the evaluating assignment
to ordinary variables (is).

\subsection{b\_eqpred}
This function generates code for the {\bf eqpred} built in.
The following code is generated for {\bf eqpred($i$)}:

\verb+eqpred avXXX,predicate_number, arity+

where {\tt avXXX}, and the arity and number of the predicate
belong to predicate $i$ of the current clause.


For each literal, the literal-control-block is generated prior to the
generation of the argument vector.

\subsection{Changed Files}
The files {\tt symb.h, built\_ins.c, symb.c, codegen/*.c} are
involved.

\section{TO BE DONE}
The following items are still to be specified and implemented:
\begin{enumerate}
\item
Entering a built-in on LOP-level
\item
Generation of code for non-deterministic built-ins
\item
Generation of code for additional inference rules
(like random reordering, usage of lemmata,...)
\item
Access and usage of arrays of global variables
\item
The global evaluating assignment.
\end{enumerate}


\section{Testing}

\section{Proposed Time Schedule}

\begin{center}
\begin{tabular}{|l|r|r|r|}
\hline\hline
Action & mplop & wasm & setheo \\
\hline
Specification & May/93& & \\
Implementation &Jun/93 & & \\
Test &Jun/93 & & \\
Documentation &May/93 & & \\
Review & & & \\
\hline
Total & & & \\
\hline\hline
\end{tabular}
\end{center}


\section{Design and Implementation Procedure}

\begin{center}
\begin{tabular}{|l||c|c||c|c||c|c||}
\hline\hline
Action & \multicolumn{2}{|c|}{mplop} &
\multicolumn{2}{|c|}{wasm} & \multicolumn{2}{|c|}{setheo}  \\
\hline
& Name & Date & Name & Date & Name & Date \\
\hline\hline
 Proposed:& jsc & May/93 & & & & \\
 Specified:&jsc &May/93  & & & & \\
 Approval of Specification:& & & & & & \\
 Start of Implementation:& jsc& 4.6.93& & & & \\
 End of Implementation:`& & & & & & \\
 Start of Tests:& & & & & & \\
 End of Tests:& & & & & & \\
 Approval of Code:& & & & & & \\
 Integration into Version: & & & & & & \\
 Start of Tests (of new Version):& & & & & & \\
 End of Tests (of new Version):& & & & & & \\
 Approval of Change:& & & & & & \\
\hline\hline
\end{tabular}
\end{center}

\end{document}
