\documentstyle[12pt,a4]{article}


\title{The built-in predicate ``ancestor/1''}
\author{Johann Schumann}
\date{27.3.93}

\begin{document}

\maketitle

\begin{abstract}
This document specifies the behavior and implementation of the built-in
{\bf ancestor/1} which allows to access the current ancestor list within
the SAM.
\end{abstract}

\noindent{\bf Keywords:}  built-in 

\noindent{\bf Action: } {\tt EXTENSION}

\section{Solution Provided}

none

\subsection{Problems with Solution Provided}
	
No means for access of the ancestor list on the LOP level
have been provided.


\section{Detailed Description}

\subsection{Informal Description}

This built-in allows to extract individual elements from the
ancestor list, and unifies them with the given term.
The behavior of this predicate is the same as of the standard predicate
member: if {\bf ancestor/1} is used with an unbound variable, the
first element of the ancestor-list will be converted into a term
with the predicate symbol as the functor. Then the variable will be
bound to this term.
If {\bf ancestor} is called with a term with a predicate symbol
(or its negation) as the root-functor, {\bf ancestor} tries to
find a member in the ancestor list which is unifiable with that term
and performs a unification.

The predicate fails, if no such member could be found, or the end of
the list is reached.

When a backtracking occurs over that built-in, the next element in the
ancestor list is tried (or returned). If no more members exist, the predicate
will fail.

The following pseudo-LOP code will illustrate the behavior of {\bf ancestor}.
We assume, that the ancestor-list is present in the global variable
{\tt \$Ancestors}.

\begin{verbatim}
begin module ancestor.
  export ancestor/1.
  begin control.
  $Depth := 999999.
  $Inf := 999999.
  $Copies := 999999.
  end control.

ancestor(T) :- member(T,$Ancestors).
member(T,[T|_]).
member(T,[_|R]) :- member(T,R).
end module.
\end{verbatim}

\noindent {\bf Example1:}
the following example shows, how the reduction steps
can be implemented on the lop-level. Please note, that the {\bf inwasm}
must be called with {\em -noreduct}

\begin{verbatim}
p(X) :- ancestor(~p(X)).
~p(X) :- ancestor(p(X)).
q(X) :- ancestor(~q(X)).
~q(X) :- ancestor(q(X)).

<-p(a),q(b).
p(X) <- ...
\end{verbatim}

\noindent {\bf Example2:}
the following example shows, how a special purpose unification, implemented
as a built-in {\bf my\_unify(X,Y)} can be used within SETHEO for
extension and reduction steps.

\begin{verbatim}
# reduction steps.
p(X,Y) :- ancestor(~p(U,V)),my_unify(X,U),my_unify(Y,V).
~p(X,Y) :- ancestor(p(U,V)),my_unify(X,U),my_unify(Y,V).
...
# original clause: p(a,f(X,b)) <-q(...
p(U,V) :- my_unify(U,a),my_unify(V,f(X,b)),q(...
...
\end{verbatim}

\section{SAM-Based Specification}

The realisation of the built-in {\bf ancestor} within the SAM is
more complex than that of other built-in predicates, since
it must be backtrackable. The built-in must carry an internal pointer
which points to the environment of the element of the ancestor list
currently being tried. This pointer must be allocated on the stack and
must be backtracked. Additionally, a new choice-point must be generated.

In order to accomplish this, two approaches are possible:
(1) a high-level approach, where the predicate {\bf ancestor} is
defined in lop using low-level built-in predicates, and (2) a
direct implementation of such a backtrackable predicate.

\subsection{The high-level approach}

In this approach, the predicate {\bf ancestor} is defined by 3
low-level built-in predicates as shown below. This predicate definition,
however is hidden in an inwasm-libary, so the user only sees one
predicate.

\begin{verbatim}
ancestor(T) :- init_ancestor(I), my_ancestor(T,I).

my_ancestor(T,I) :- getancestor(T,I).
my_ancestor(T,I) :- incr_ancestor(I,I1), my_ancestor(T,I1).

\end{verbatim}

The low-level predicates work are defined as follows:

\begin{description}
\item[init\_ancestor(I)] initializes the variable $I$ to the beginning
of the ancestor list. In our case, $I$ is just a pointer
(with tag T\_BVAR) to the current environment.
\item[getancestor(T,I)] tries to unify the ancestor with the environment
$I$ with $T$. If the ancestor list is empty (i.e.\ $I == NULL$), or
the unification fails, that predicate fails.

\item[incr\_ancestor(I,I1)] just performs: $I1$ = $I${\tt ->dyn\_link}.
This causes the check for the ancestor to proceed one level higher.
\end{description}

\subsection{The low-level approach}

\subsection{Discussion of Approaches}


\subsection{Data Structures}

none

\subsection{New SAM instructions}

\subsubsection{The functor-instruction}

A new SAM instruction {\bf functor} is being defined. Its argument
points to a correponding argument vector of length 3.

The pseudo-code of {\bf functor} is as follows:

\begin{verbatim}
---
\end{verbatim}

\subsubsection{The negate-instruction}

The pseudo-code for this built-in is as follows. Again, the argument
of this built-in is a pointer to a one-element argument vector.

\begin{verbatim}
---
\end{verbatim}

\subsection{Additional C-functions within SETHEO}
\subsubsection{The function heapgen}

as specified in Document ``functor''.

\section{Required Changes}

\subsection{Required Changes in sam}

Files to be modified: conf.c

New files: i\_functor.c, i\_negate.c

\subsection{Required Changes in inwasm}

The instructions {\bf ancestor} must be added to the set of
built-in predicates as a library.

\section{Required Changes in wasm}

Must be compiled after making setheo.

\section{The Testing}
\subsection{Test-examples}
{\tt Beispiele hier angeben}

\subsection{Results of Tests}
{\tt Resultate: Ausgabe, Zeiten,... angeben}

\section{Proposed Time Schedule}

\begin{center}
\begin{tabular}{|l|r|r|r|}
\hline\hline
Action & inwasm & wasm & sam \\
\hline
Specification & & & \\
Implementation & & & \\
Test & & & \\
Documentation & & & \\
Review & & & \\
\hline
Total & & & \\
\hline\hline
\end{tabular}
\end{center}


\section{Design and Implementation Procedure}

\begin{center}
\begin{tabular}{|l||c|c||c|c||c|c||}
\hline\hline
Action & \multicolumn{2}{|c|}{inwasm} &
\multicolumn{2}{|c|}{wasm} & \multicolumn{2}{|c|}{sam}  \\
\hline
& Name & Date & Name & Date & Name & Date \\
\hline\hline
 Proposed:& & & & & & \\
 Specified:& & & & & & \\
 Approval of Specification:& & & & & & \\
 Start of Implementation:& & & & & & \\
 End of Implementation:`& & & & & & \\
 Start of Tests:& & & & & & \\
 End of Tests:& & & & & & \\
 Approval of Code:& & & & & & \\
 Integration into Version: & & & & & & \\
 Start of Tests (of new Version):& & & & & & \\
 End of Tests (of new Version):& & & & & & \\
 Approval of Change:& & & & & & \\
\hline\hline
\end{tabular}
\end{center}

\end{document}
