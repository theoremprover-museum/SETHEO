%details
\ssection{Implementations Guide}
\ssubsection{Layout of the System}
The SETHEO--system consists of the programs
{\bf plop},
{\bf preproc},
{\bf inwasm},
{\bf wasm},
{\bf setheo},
{\bf xtheo} and
{\bf xtheo--graphics}.
Some fonts for the xtheo and libraries for wasm are needed as well.
\ssubsection{System Requirements}
To run the entire system including xtheo, you have to have a sun
with SUN/OS 3.2 or later.
If you do not need any graphical output, any UNIX system will do well.
%Also some other hardware systems may work, e.g.\ atari ST, IBM/PC.
Successful installations have been carried through on sun2, sun3, sun4,
sun386i, IBM PC/RT, Targon 35, AT\&T/Olivetti 3b2, 386--PC under UNIX,
Transputer (T800 and T414 with 3L--parallel C and IBM/PC), and
$\mu$VAX, VAX/750.
%Minor problems may occur for atari and PC under MS--DOS.

In any case you need:
\begin{itemize}
\item
Setheo sources (of course)
\item
C compiler with a 32-bit data type (preferable int) and a
UNIX--lookalike run--time library.
\item
The hash(3) functions. If they are not available, a home--made
hash function hsearch for inwasm and wasm must be written.
\item
The sun-view libraries and include files for xtheo and xtheo--graphics ONLY.
\item
Memory requirements: 0.5 MB of RAM should be sufficient for most examples.
\end{itemize}

\ssubsection{Installing SETHEO}
\ssubsubsection{Loading the sources}

Prior to loading SETHEO we recommend to create a new user (for example 
`setheo'), the home directory of which will hold the respective files 
and directories of SETHEO.
In this home directory SETHEO can be loaded on a UNIX machine using the

\vspace{-2mm}
\begin{center}
              tar xvf /dev/rst0
\end{center}
\vspace{-2mm}
command. It loads the source files under the current
directory.
They are:
\\[3mm]

\begin{tabular}{|l|l|}
\hline
SETHEO.users.registration.form & users registration form \\
make.conf      & configuration file for make \\
README        & README file containing important information \\
SOURCES/plop  & directory containing formula to LOP translator \\
SOURCES/preproc & directory containing preprocessing modules \\
SOURCES/inwasm  & directory containing formula compiler \\
SOURCES/wasm & directory containing setheo assembler \\
SOURCES/setheo & directory containing setheo theorem prover \\
SOURCES/xtheo & directory containing user interface for sun \\
SOURCES/include & directory containing include files for SETHEO \\
man     & directory containing manual pages and auxiliary documentation\\
examples& directory containing examples \\
bin	& directory for compiled programs \\
\hline
\end{tabular}\\[3mm]
Please ensure for the following installation that the environment variable
\$SETHEOHOME is set to the directory containing the files and directories above.



\ssubsubsection{Installation of Manual-Pages}

To install the manual pages of SETHEO you have to copy the files
in the directory {\tt man} ending with \verb-.1- to your local manpage 
directory.

Note: You must be superuser to do this. 

\ssubsubsection{Installation of the SETHEO Programs}

In order to install the SETHEO-programs 
{\bf plop},
{\bf preproc},
{\bf inwasm},
{\bf wasm},
{\bf setheo},
{\bf xtheo} and
{\bf xtheo--graphics} you might type

\begin{center}
                cd SOURCES\\
                make \\
		cd ..
\end{center} \vspace{-2mm}

This will start the compilation of the several programs and copy the 
executables to the directory \$SETHEOHOME/bin.
Please ensure that this directory is included in your path, or else copy the
contents of this directory to an appropriate directory on your machine.

%\ssubsubsection{Installation of the Preprocessor}
%
%To install the SETHEO Preprocessor preproc you have to be in the directory
%./preproc, so you might type \vspace{-2mm}
%
%\begin{center}
                %cd SOURCES/preproc\\
                %make install\\
		%cd ../..
%\end{center} \vspace{-2mm}
%
%
%\ssubsubsection{Installation of the Translator plop}
%
%To install the SETHEO Formula Translator {\it plop\/} you have to be in the directory
%./setheo/plop, so you might type \vspace{-2mm}
%
%\begin{center}
%                cd SOURCES/plop\\
%                make install\\
%		cd ../..
%\end{center} \vspace{-2mm}
%
%
%\ssubsubsection{Installation of the SETHEO-Compiler}
%
%To install the SETHEO Compiler inwasm: \vspace{-2mm}
% 
%\begin{center}
%                cd SOURCES/inwasm\\
%                make install\\
%		cd ../..
%\end{center} \vspace{-2mm}
%
%
%This will create the program {\tt inwasm} in the directory bin.
%
%Note: If You are running strange Unix-systems (e.g. Targon35),
%you have to type
%  
% \vspace{-2mm}
%\begin{center}
%                att make install
%\end{center}
% \vspace{-2mm}
%      
%      in order to get the hsearch(3C) routines. For very old bsd
%      systems which do not have the hsearch(3C), see 'Internals'
%
%\ssubsubsection{Installation of the SETHEO Assembler}
%
%To install the SETHEO Assembler wasm you might type
%\vspace{-2mm}
%\begin{center}
%                cd SOURCES/wasm\\
%                make install\\
%		cd ../..
%\end{center}\vspace{-2mm}
%
%
%Note:
% wasm uses hsearch(3C) as well as inwasm, so the note above
%      holds as well.
%
%\ssubsubsection{Installation of SETHEO}
%
%To install setheo you have to say:\vspace{-2mm}
%\begin{center}
%                cd SOURCES/setheo
%                make install\\
%		cd ../..
%\end{center}
% \vspace{-2mm}
%
%%\ssubsubsection{Installation of PARTHEO}
%%PARTHEO consists of 3 parts: the worker, the master and the configuration
%%files. The worker itsself consists of a modified SAM (in the directory
%%./setheo/mainpart/gsrc) and a parallel communications part in ./setheo/mainpart/partheo/worker.
%%To make the worker, go into the SETHEO directory and edit the makefile:
%%change the {\tt MACHINE =} to:
%%\begin{center}
%%MACHINE = TRANSPUTER
%%\end{center}
%%
%%After that make the setheo library with (from the MS-DOS window):
%%\vspace{-2mm}
%%\begin{center}
%%make partheo.lib
%%\end{center}
%%
%%After that change to the directory worker with (MS-DOS):
%%\vspace{-2mm}
%%\begin{center}
%%cd ..$\backslash$partheo$\backslash$worker \\
%%make worker
%%\end{center}
%%
%%To make the master say: \vspace{-2mm}
%%\begin{center}
%cd ..$\backslash$master \\
%make master
%\end{center}
%\vspace{-2mm}
%and configure all parts with (MS-DOS):
%\vspace{-2mm}
%\begin{center}
%cd ..$\backslash$conf \\
%pt
%\end{center}
%\vspace{-2mm}
%This will configure PARTHEO for a 16-Transputer Torus topology.
%Partheo itself can be started from the MS-DOS window saying:
%\vspace{-2mm}
%\begin{center}
%partheo
%\end{center}

%\ssubsubsection{Installation of the User Interface Xtheo}
%
%To install the SETHEO User Interface {\it Xtheo\/} 
%you might type \vspace{-2mm}
%
%\begin{center}
%                cd SOURCES/xtheo\\
%		make install\\
%		cd ../..
%\end{center} \vspace{-2mm}
%
%Note: You have to be superuser to install the fonts at the proper
%places. \vspace{-2mm}
%

\ssubsubsection{Testing the theorem prover}

After installation you can test the operation of the theorem-prover:
\begin{description}
\item[manuals]  Say: \vspace{-2mm}

\begin{center}
                man setheo \\
                man inwasm \\
                man wasm \\
                man preproc \\
                man plop \\
\end{center} \vspace{-2mm}

  A UNIX-lookalike manual page should appear on the screen.

\item[theorem prover] Go to the directory  examples
  and say: \vspace{-2mm}
           
\begin{center}
                proofall
\end{center} \vspace{-2mm}

  All examples contained in this directory are run. No error message
  should occur and the LOG file should be empty.
  To test a special example, say chang6.lop just say
 \vspace{-2mm}
\begin{center}
                chang6.pr
\end{center}
 \vspace{-2mm}
  This invokes the inwasm, the wasm and setheo with a correct
  parameter setting.
\end{description}

Note: You might have to 'rehash' before if you are using the csh(1).

\ssubsubsection{Customisation SETHEO to special machines}

If you have a machine not listed above you have to determine some
parameters before installing setheo. To do this say:
Go into the directory \vspace{-2mm}
\begin{center}
                SOURCES/setheo
\end{center} \vspace{-2mm}

Type: \vspace{-2mm}
\begin{center}
                cc byte.c \\
                ./a.out
\end{center} \vspace{-2mm}

An output like the following may be produced: \vspace{-2mm}

\begin{verbatim}
      Size of int in bytes: 4 pointer in hex: 4108
      Mask to be used for setheo: 0 Hex
\end{verbatim}

If the size of an int is not 4 bytes you will get in trouble.
The displayed mask must be entered as
 {\tt PTR\_MASK}
in the file {\it tags.h}.

If the size of int is 2 bytes, make the follwoing entry: \vspace{-2mm}
\begin{center}
               typedef long WORD
\end{center} \vspace{-2mm}

In the current version there is no guarantee that this will run.

After changing the file tags.h, proceed with an ordinary installation.


\ssubsubsection{Cleaning up}

If you also want to remove the standard examples
you may as well remove the entire setheo directory.

To clean up the .o files and core--files
just go to the appropriate directories
and say: \vspace{-2mm}
\begin{center}
                make clean
\end{center}


\begin{center}
{\bf Have fun !!}
\end{center}
