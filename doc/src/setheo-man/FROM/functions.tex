\documentstyle[12pt,a4]{article}


\title{Interpretation of SAM Instructions\\
--- CHANGE DESCRIPTION ---}
\author{Johann Schumann}
%\date{23.7.92}
\date{26.10.92}

\begin{document}

\maketitle

\begin{abstract}
This document specifies the changes necessary for a new style
of interpreting the SAM instructions.

\end{abstract}

\section{Description of Change}

\subsection{Solution Provided}

In the current version of setheo (V3.01 or less), the main command-loop
(function {\tt instr\_cycle}) mainly consists of an endless for loop.
The decoding of instructions is done via two nested {\bf switch} commands.
as shown in the following:

\begin{verbatim}
instr_cycle()
{
...
// check bounds
switch (* pc & INSTR_MASK){

case CONTROL:
     switch (*pc){
     case ALLOC:
             ... // code for alloc
             break;
     case CALL:
             ...
          }
case UNIFSIMPLE:
     ...

}
\end{verbatim}

If an instruction succeeded, just a {\bf break} is executed, otherwise,
a {\bf goto} to the label {\bf fail} is made.

In order not to have the entire code of the SAM interpreter in one file,
a file with the extension {\tt .ins} is being used for each instruction.
Additionally there exist include files for each group of instructions,
as shown in the following piece of code:

\begin{verbatim}
instr_cycle()
{
...
// check bounds
switch (* pc & INSTR_MASK){

case CONTROL:
     switch (*pc){
#      include "contrins.h"   ------>   #include "instruct/alloc.ins"
                                        #include "instruct/call.ins"
case UNIFSIMPLE:
     ...

}
\end{verbatim}

\subsection{The Problem}
This kind of decoding the SAM instructions is very efficient,
however, it also exhibits several drawbacks:
\begin{itemize}
\item
poor readability
\item
one change causes a recompilation of the entire function {\tt instr\_cycle}
\item
a larger number of instructions in one group needs a longer time for
decoding (linear search within the list of case-labels)
\end{itemize}

\section{Requirements for new Solution}

\section{Description of New Solution}
The solution proposed is similar to that which is being used already
for the built-ins.

Each SAM-instruction is being encoded in a function with zero
parameters. The outcome of the instruction (success, failure, error)
is returned as the return value of the function.

Furthermore, a global table which contains all function names,
the print-names of the instructions, and the length of each instruction
exists.
The OP-code for each instruction is the index of this instruction
in this function table.

This kind of decoding allows for a constant-time decoding, however with the
additional cost of a function call with no parameters.

As a second restriction, the OP-codes for the instructions must be integers
in the range of $0\ldots n_{instr}$ without any holes.

As described below, this global function table is used to construct
the OP-code table for the {\bf wasm} as well.

\section{Data Structures}

\subsection{The Instruction Table}

The instruction table is located in the file {\em conf.c}.
It replaces the old file, since all built-ins are handled in the same way
as ordinary SAM instructions.

It contains the following definitions\footnote{The declaration of the instruction
table itsself is in the file {\em machine.h}.}:

\begin{verbatim}
...
// machine.h
typedef struct _i_tab {
        int     (*instruction)();/* function for exec.       */
                                /* the instruction           */
        char    *name;          /* name of instruction       */
        int     length;         /* size of instr. in WORDS   */
        } instr_tab;
 
extern instr_tab        instr_table[];
extern int              nr_instr;
\end{verbatim}

The main table of instructions is located in the file {\tt conf.c}:

\begin{verbatim}
/***************************************************************/
/* identification name of that SETHEO version
/***************************************************************/
char           *ident = "XXX";
char           *version = "3.0";
int             magic = 123;

/***************************************************************/
/* global list of all SETHEO instructions
/***************************************************************/
instr_tab       instr_table[] =    {

/*------------------------------------------------------------*/
/* function             name            length
/*------------------------------------------------------------*/
{  i_alloc              ,"alloc"        ,3              },
{  i_nalloc             ,"nalloc"       ,3              },
{  i_dealloc            ,"dealloc"      ,1              },
{  i_ndealloc           ,"ndealloc"     ,1              },
...
\end{verbatim}


Each SAM-instruction is located in a separate file and contains one
function:

\begin{verbatim}
/******************************************************/
/*    S E T H E O                                     */
/*                                                    */
/* FILE: i_cut.c
/* VERSION:
/* DATE:
/* AUTHOR:
/* NAME OF FILE:
/* DESCR:
/* MOD:
/* BUGS:
/******************************************************/

#include "machine.h"

#define INSTR_LENGTH            2


instr_result i_cut()
{
/* your program text comes here */
fprintf(stderr,"sorry, %s is not implemented\n","cut");
pc +=INSTR_LENGTH;
return error;
}
\end{verbatim}

The main instruction cycle now has the following form:

\begin{verbatim}
// FILE: instr.c

inst_cycle()
{
...
for(...
	if ((res = (*instr_table[*pc].instruction)()) & stop_cycle){
                /* we break the instruction cycle and   */
                /* return the cause of the break        */
                return res & ~stop_cycle;
                }
        if (res == failure){
        if ((res =i_fail()) & stop_cycle){
                /* we have to break the stuff */
                return res & ~stop_cycle;
                }
        }
}
\end{verbatim}

\section{SAM instructions involved}

All SAM instructions are involved, since their structure is changed.
Note, that all variables which are now declared as automatic in
{\tt instr\_cycle} have to be declared globally in {\em machine.h}.

For the access of the argments of each instruction, the following
macro replaces {\tt farg,farg1,...}:

\begin{verbatim}
/******************************************************************/
/*** MACROS for argument access
/******************************************************************/
#define ARG(I) (*(pc + I))
#define ARGPTR(I)  (code + *(pc +I))
\end{verbatim}


\section{Required Changes in mplop}

The code-generation of MPLOP must be changed in that way, that for
built-in instructions the correct name of the instruction is being
used (in the .s file) instead of the construct {\tt built  number,argptr}

\section{Required Changes in wasm}

The scanner and parser of the wasm must be adapted accordingly to
the contents of the file {\tt conf.c}.
This is done by the script {\tt mkstubs}.

\section{Creation of a New SETHEO}

The global basis for all instructions is the file {\tt conf.c}
containing the print-names and function-names for all instructions.
From there, the script {\tt mkstubs} generates all necessary files
for setheo and wasm:
\begin{itemize}
\item
{\tt conf.h} contains external declarations
\item
{\tt i\_xxx.c} where xxx is the name of the instruction.
Such a file which contains an empty frame for that instruction
is created for each instruction in conf.c (unless such a file already
exists).
\item
{\tt make.conf}
contains the list of all instruction source (and object) files.
\item
{\tt w\_scan.h}
contains the lex-scanner for the wasm
\item
{\tt w\_parse1.h}
contains parts of the parser for the wasm (token definitions).
\item
{\tt w\_parse2.h}
contains parts of the parser for the wasm (grammar productions).
\end{itemize}

After that, the wasm must be created newly with a ``make wasm''.

\section{Proposed Time Schedule}

\begin{center}
\begin{tabular}{|l|r|r|r|}
\hline\hline
Action & mplop & wasm & setheo \\
\hline
Specification & & & \\
Implementation & & & \\
Test & & & \\
Documentation & & & \\
Review & & & \\
\hline
Total & & & \\
\hline\hline
\end{tabular}
\end{center}
\section{History}

\begin{center}
\begin{tabular}{|l||c|c||c|c||c|c||}
\hline\hline
Action & \multicolumn{2}{|c|}{mplop} &
\multicolumn{2}{|c|}{wasm} & \multicolumn{2}{|c|}{setheo}  \\
\hline
& Name & Date & Name & Date & Name & Date \\
\hline\hline
 Proposed:& & & & & & \\
 Specified:& & & & & & \\
 Approval of Specification:& & & & & & \\
 Start of Implementation:& & & & & & \\
 End of Implementation:`& & & & & & \\
 Start of Tests:& & & & & & \\
 End of Tests:& & & & & & \\
 Approval of Code:& & & & & & \\
 Integration into Version: & & & & & & \\
 Start of Tests (of new Version):& & & & & & \\
 End of Tests (of new Version):& & & & & & \\
 Approval of Change:& & & & & & \\
\hline\hline
\end{tabular}
\end{center}

\end{document}
