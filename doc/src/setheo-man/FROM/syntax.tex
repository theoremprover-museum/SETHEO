%syntax
\ssection{Syntax of the Input Language}
\ssubsection{Input Language of PLOP}

In this part we describe the syntax of the input language, which is
accepted by the formula translator {\bf plop}.
A formula in that notation has to be in a file with the extension {\bf .pl1}.

Because the grammar of first order logic is well-known it seems
useless to repeat an inductive definition of the concept 
{\em ``formula''}. We rather represent the special PLOP
requirements i.e. the differences to the usual
standard in giving some comments on the individual
elements of the accepted language

\paragraph{Variables}
Variables are strings starting with a capital letter followed 
by arbitrary letters, digits, or underscore.

\paragraph{Constants}
Predicate-, function-, and object parameters are strings starting 
with a lower case letter followed by lower case letters, digits, or underscore. 
Predicate- and function arguments are to be included in round brackets.
Multiple arguments are to be separated by a comma ``,''.

\paragraph{Propositional Connectives}
The symbols for the propositional connectives are:

\begin{center}
\begin{tabular}{llll}
(1) &  {\tt $\sim$}  &   - &  not, \\
(2)  &	 {\tt \& } & -  & and, \\
(3)   &{\tt ; } & -  & or, \\
(4)   &{\tt -> } & -  & if ... then, \\
(5) & 	 {\tt <->}  & -  & iff. 
\end{tabular}                  
\end{center}

The binding preference is given by this order:
1 $>$ 2 $>$ 3 $>$ 4 $>$ 5 . 
Equal connectives are left associative i.e.,
\tt 
\mbox{a -> b -> c} \rm is to be read as \tt \mbox{((a -> b) -> c)} 
\rm .

\paragraph{Quantifiers}
The quantifiers are represented by
``{\tt
forall}'' and ``{\tt exists}''.
These symbols must not occur as parts of other names like constants
or variables.
The scope of a quantifier extends over the following propositionally 
complete subformula and ends there.
E.g. at the formula
\[\tt 
\mbox{forall X a -> b(X)}  \\
\rm
\]
the scope of the quantifier does not include {\tt b(X)}.
To bind {\tt X} by this quantifier one has to write as usual with parenthesis
\[\tt 
\mbox{forall X (a -> b(X)).}\\
\rm
\]
Occurences of free variables are understood by PLOP as bound
from the outside (universal closure).

\paragraph{Brackets}
Parenthesis are expressed by the usual {\em round brackets} only.
{\em Square brackets} must not be used in the whole input formula .

\paragraph{Special Characters}
All characters with ASCII value $<$ 32 are ignored and will be deleted
during the transformation.
The {\em space-character} can be omitted if the meaning is not affected and if
there cannot result any ambiguities. 
E.g. 
\[\tt 
\mbox{forallX} \\
\rm
\]
is recognized as a quantifier and a variable. 
In other cases a space character must be set, e.g.
\[\tt 
\mbox{forall X a(X) }\\
\rm
\]
cannot be written as 
\[ \tt 
\mbox{forall Xa(X).}\\
\rm
\]


Other characters as mathematical function symbols (especially in
infix notation) are not recognized, which perhaps could affect the
correctness of the transformation in some cases and are therefore to 
be treated carefully. 
If you cannot avoid 
them at any rate suppress the optimization concerning redundancies
and tautologies (option -nopti). 

\paragraph{Comments}
Comments are enclosed in C--like convention in ``{\tt /*}'' and ``{\tt */}''
and can extend over several lines. Nested comments are not allowed.
All comments are removed during the conversion.

%\paragraph{Size}
%The admissible size of the input file is limited and can be displayed
%by the option -defgro (``groesse1\_\,c'').
%If it is not sufficient for your example (e.g. a large program) 
%divide the input in several parts or change the define-statement for
%groesse1\_\,c in the source-code.

\ssubsection{Input Language LOP}

The language LOP is accepted by the setheo--compiler {\bf inwasm}, and
parts of it also by the preprocessor. It represents a formula as (conjunctive)
normal form (clausal form).
Two different syntax styles may be used: the logic-style and a prolog-like
style.
Clauses in logic style have a {\tt <- } as seperator between
positive and negative literals.
These clauses are fanned automatically. The subgoals of the fanned
clauses are reordered, unless the -R option was specified during the
compilation with inwasm.

Prolog-like clauses have as seperator {\tt ?- } or {\tt :- }.
The first denotes a query, i.e.\ an entry point for proving,
the other separates negative and positive literals in one clause.
Note that there may be more than one queries. In that case, use the
-q option of inwasm.
Prolog-style clauses can also contain negated literals, i.e. {\tt ~literal}.

Prolog-style clauses are {\it not\/} fanned. So in the case of non-horn
formulae completeness may be sacrificed. Again reordering is
controlled with the -R option of inwasm.
 
Both syntax styles can be mixed.
If Horn--clauses
are written in the logic-style, i.e.\ the {\tt headlist} contains a maximum
of one literal, there is no difference between this notation and prolog-style
notation. 
 We give a top-down grammar--specification
(extended bnf)
of the syntax:

{
\begin{verbatim}
LOP_program ::= clause *

clause      ::= type1_cl .       /* prolog-like                      */
              | type2_cl .       /* logic-like                       */

type1_cl    ::= ?- tail          /* query                            */
              | nliteral         /* fact                             */
              | nliteral :- tail /* rule                             */

tail        ::= nliteral         /* list of literals:     'subgoals' */
              | nliteral ',' tail

nliteral    ::= literal         /* literal                           */
              | ~ literal       /* negated literal                   */

literal     ::= <PROLOG-like literal>

type2_cl    ::= headlist <- taillist

headlist    ::= <empty>         /* list of (default) positive        */
              | nliteral        /*           literal(s)              */
              | nliteral ';' headlist

taillist    ::= <empty>
              | nliteral         /* list of (default) neg.           */
                                 /* literals "subgoals"              */
              | nliteral ',' taillist
\end{verbatim}
}

A literal is written in a PROLOG--like syntax: constants and predicate
symbols are starting with a lower--case letter, variables with an
upper--case letter or an underscore \_, e.g.\ {\tt p(X1,\_Yabc,cb12)}.
Lists have a notation
like in PROLOG with [ ]. First element and rest
of a list is separated by a '|'.
A special data type, global variable, is
written like an ordinary variable, but starts with a '\$'.
For details see "Programming in LOP".
Built--in predicates count as one literal
and must occur in the tail of a rule or in the query only , i.e.\ only negated
built--in predicates are allowed.
A list of all built--in predicates appears in "Programming in LOP".

Some examples show the syntax of LOP and its usage,
where f denotes fanning, r reordering, and 1 and 2 the clause types 1,2.
\vspace{5mm}

\begin{center}
\begin{tabular}{|l|l|l|}
\hline
LOP formula & logic negation & remarks \\
\hline
\hline
{\tt p(X) $<$-.} & p(X) & 2,f \\
\hline
{\tt p(X).} & p(X) & 1 \\
\hline
{\tt p;q $<$- r,s.} &
 p $\vee$ q $\vee \neg$ r $\vee \neg$ s 
& 2, f, r, non--horn \\
\hline
{\tt p :- ~q,r,s.} & p $\vee$ q $\vee \neg$ r $\vee \neg$ s &
1, non--horn \\ 
\hline
{\tt ?- p,q,r.} &
 $\neg$ p $\vee \neg$ q $\vee \neg$ r 
& 1 \\
\hline
{\tt $<$- p,q,r.} &
 $\neg$ p $\vee \neg$ q $\vee \neg$ r 
& 1, same as above, if entire formula is horn \\
\hline
\end{tabular}
\end{center}
\vspace{8mm}

For examples in LOP see "SETHEO as a Theorem Prover" and "Programming in LOP".


                
