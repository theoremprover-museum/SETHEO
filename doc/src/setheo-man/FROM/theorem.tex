% theorem
\ssection{SETHEO as a Theorem Prover}
To use SETHEO as a theorem prover, we give a table of
advisable parameters for all modules, which should be used
with different types of formulae.
A detailed description of the parameters can be found in the section
"Description of the Modules" or in the manual pages.
The abbrevation "dfs" means that the formula can be proven with a
depth--first--search mode (PROLOG--like).

\begin{center}
\begin{tabular}{|l|l|l|l|l|}
\hline
Type of formula & inwasm & wasm & setheo  & remarks\\
\hline \hline
logic program, horn, dfs & & & & no parameters needed\\
\hline
large program, horn, dfs & & --O & --S 1000000 & optimise, large stack \\
\hline
horn formula, one query & --T & & --d {\it nu\/} & set bound to {\it nu\/}\\
&&& or: --r & iterative deepening\\
&&& or: --i {\it nu\/} & set inf. bound to {\it nu\/}\\
\hline
Horn, arbitrary & --T --q & & --r & set bound, artif. query \\
&&& or: --d nu &\\
&&& or: --i nu &\\
\hline
non--horn & --T --r --q & & --r & need reduction steps \\
&&& or: --d nu &\\
&&& or: --i nu &\\
\hline
\hline
additionally (all types)& --p & & & equal predecessor fail\\
\hline
additionally (all types)& --c {\it nu\/}& & & set copy bound to {\it nu\/}\\
\hline
\end{tabular}
\end{center}
\vspace{3mm}

The following example is from [AAR--newsletter]. In its clausal representation
it consists of 6 clauses, one being non--horn. In this case the inwasm
is called with the parameters {\bf --q --t --r} and optionally {\bf --p}
to reduce the complexity of the search. Setheo can do an iterative deepening
search {\bf --r} or the valid depth can be given directly, if known {\bf --d 9}.

\begin{center}
\begin{quote}
{\tt
<- p(a,b). \\
<- q(c,d). \\
q(X,Z) <- q(X,Y), q(Y,Z). \\
p(X,Z) <- p(X,Y), p(Y,Z). \\
q(X,Y) <- q(Y,X). \\
p(X,Y);q(X,Y)<-. \\
}
\end{quote}
\end{center}
