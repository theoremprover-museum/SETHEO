\documentstyle[12pt,a4]{article}


\title{Time-limits for SETHEO}
\author{Johann Schumann}
\date{21.9.94}

\begin{document}

\maketitle

\begin{abstract}
This document specifies the changes necessary for introducing
real-time and CPU-time limits for the SAM.
tree.

\end{abstract}

\noindent{\bf Keywords:}  time-limits

\noindent{\bf Action: } {\tt EXTENSION}

\section{Detailed Description}

For the SETHEO Abstract Machines, two kinds of run-time limits
are introduced:
\begin{description}
\item[realtimelimit]
This limit aborts the run of SETHEO after a given number of seconds
(real, ``wall-clock'' time).
This limit is useful in combination with interactive systems which
require a limitiation of the run-time of SETHEO.
The timer starts, as soon as the parameters of {\bf sam} have been parsed.

\item[cputimelimit]
This limit aborts the run of SETHEO after the abstract machine
has consumed a number of seconds user (CPU) time.
\end{itemize}

Whereas the {\bf realtimelimit} is realised, using the {\bf alarm(2)}
timer and the signal SIGALRM is used, the implementation of
the cpu-time limit is more complicated.
Since a signal which is raised on CPU-time limit (SIGXCPU) is not available
on each architecture, a portable solution is provided.

\subsection{Informal Description}

For the CPU-time limit we propose the following (portable)
algorithm:

Let {\tt cputime\_limit} be the given time-limit in seconds.
Then, a global variable {\tt cputime} is initialized using the
times(2) system call during start-up of the system.
Then, an alarm is set to ring after {\tt cputime\_limit} seconds.
After that time, the current elapsed cpu-time is estimated by
{\tt now - cputime}, where {\tt now} is the current values as returned
by times(2). If the difference {\tt now - cputime - cputime\_limit}
(which is {\em always $\geq 0$}) is greater than 1 second, a new alarm
is set up with that difference.
Otherwise, the run of the abstract machine is aborted with
``CPU time failure''.

If the new alarm rings off, the same procedure applies again.

Therefore we obtain:

\[ {\tt cputime\_limit}-1 \leq t_{CPU} \leq {\tt cputime\_limit} \]

where $t_{CPU}$ is the really consumed CPU time. All times
are given in seconds. A finer resolution of the limit is not possible.
\section{Specification of Algorithms and Data Structures}
\section{SAM-Based Specification}
\subsection{Data Structures}
\subsection{SAM instructions involved}
\subsection{New Instructions}
\section{Required Changes}
\subsection{Required Changes in setheo}
Files to be modified:
alloc.c, call.c, reduct.c, output*.c

\section{Manual Pages}

\section{Discussion of Problems}
\section{Testing}
\section{Proposed Time Schedule}

\begin{center}
\begin{tabular}{|l|r|r|r|}
\hline\hline
Action & mplop & wasm & setheo \\
\hline
Specification & & & \\
Implementation & & & \\
Test & & & \\
Documentation & & & \\
Review & & & \\
\hline
Total & & & \\
\hline\hline
\end{tabular}
\end{center}


\section{Design and Implementation Procedure}

\begin{center}
\begin{tabular}{|l||c|c||c|c||c|c||}
\hline\hline
Action & \multicolumn{2}{|c|}{mplop} &
\multicolumn{2}{|c|}{wasm} & \multicolumn{2}{|c|}{setheo}  \\
\hline
& Name & Date & Name & Date & Name & Date \\
\hline\hline
 Proposed:& & & & & & \\
 Specified:& & & & & & \\
 Approval of Specification:& & & & & & \\
 Start of Implementation:& & & & & & \\
 End of Implementation:`& & & & & & \\
 Start of Tests:& & & & & & \\
 End of Tests:& & & & & & \\
 Approval of Code:& & & & & & \\
 Integration into Version: & & & & & & \\
 Start of Tests (of new Version):& & & & & & \\
 End of Tests (of new Version):& & & & & & \\
 Approval of Change:& & & & & & \\
\hline\hline
\end{tabular}
\end{center}

\end{document}
