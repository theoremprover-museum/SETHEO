\documentstyle[12pt,a4]{article}


\title{Dynamic Reading and Executing Queries in SETHEO}
\author{Johann Schumann}
\date{16.9.93}

\begin{document}

\maketitle

\begin{abstract}
This document describes the enhancements to SETHEO in order to allow for
a compilation of a formula without a query and a dynamic reading and
executing of a query.
\end{abstract}

\noindent{\bf Keywords:}  preprocessing, read, call

\noindent{\bf Action: } {\tt EXTENSION}

\section{Detailed informal Description}

For a compilation of a formula without the query and the execution
of a query dunring the run-time, extensions must be made in the compiler
inwasm and the abstract machine.

In the following, we describe a solution which uses a ``read'' predicate
to read in the query during the run-time, and the built-in ``lop-call''
which operates in a way similar to the PROLOG call.

On a high level, we add additional predicates to the formula:

\begin{verbatim}
?- write("\n>>"), read(X), call(X).
...
// here the original formula starts
//
\end{verbatim}

This simple solution, however, imposes the following restrictions:

\begin{itemize}
\item
the query must consist of ONE literal only
\item
this kind of reading in and executing a query is restricted to Horn-formulae
or to formulae which do not require an extension step into the
negated query.
\end{itemize}

For the Horn case, the first restriction can be overcome easily with
the help of the built-in predicates. In this case, a small parser
must be written in LOP.
For the Non-Horn case, however, the handling of multi-literal queries
cannot be accomplished easily, since the query must be fanned
into contrapositives. This would lead to non-unit clauses to
be added dynamically to the original formula.

The second restriction can be overcome by using the lemma storage.
After reading in the single-literal query, we store its negation
in the lemma store. From there, it can be automatically used for
extension steps into the negated query.
 
\subsection{Extensions in inwasm}

When the inwasm is asked to compile a formula without a query, using
the command line flag ``-noquery'', the following compilation steps
must be performed differently:

\begin{enumerate}
\item
No purity-reduction must be performed.
\item
The artificial query must be generated in a specific way,
\item
additional entry-points in the weak unification graph must be
generated, and
\item
special SAM code must be generated for reading in the query, and for
executing it.
\end{enumerate}

\subsection{Extensions in sam}

read(X), call(X)


\section{SAM-Based Specification}

{\tt ein absolutes MUSS}

\subsection{Data Structures}

{\tt immer im Hinblick auf SETHEO-register + Datenstrukturen}

\subsection{New Instructions}

\subsubsection{The lop\_call-instruction}

\subsection{Additional C-functions within SETHEO}


\section{Required Changes}

{\tt wichtig anzugeben: welche files werden geandert, was wird
geandert, soweit nicht oben beschrieben.}

\subsection{Required Changes in sam}

Files to be modified:

\subsection{Required Changes in inwasm}


\section{Manual Pages}
{\tt hier kommen die neuen Manual pages hin, sofern sich was geandert hat.}

\section{Discussion of Problems}

{\tt noch bestehende Probleme, moegliche Verbesserungen hier eintragen}

\section{Testing}
\subsection{Test-examples}
{\tt schwierige/spezielle Beispiele fuer die Erweiterung/Aenderung hier angeben}

\subsection{Results of Tests}
{\tt Resultate: Ausgabe, Zeiten,... angeben}

\section{Proposed Time Schedule}

\begin{center}
\begin{tabular}{|l|r|r|r|}
\hline\hline
Action & mplop & wasm & setheo \\
\hline
Specification & & & \\
Implementation & & & \\
Test & & & \\
Documentation & & & \\
Review & & & \\
\hline
Total & & & \\
\hline\hline
\end{tabular}
\end{center}


\section{Design and Implementation Procedure}

\begin{center}
\begin{tabular}{|l||c|c||c|c||c|c||}
\hline\hline
Action & \multicolumn{2}{|c|}{mplop} &
\multicolumn{2}{|c|}{wasm} & \multicolumn{2}{|c|}{setheo}  \\
\hline
& Name & Date & Name & Date & Name & Date \\
\hline\hline
 Proposed:& & & & & & \\
 Specified:& & & & & & \\
 Approval of Specification:& & & & & & \\
 Start of Implementation:& & & & & & \\
 End of Implementation:`& & & & & & \\
 Start of Tests:& & & & & & \\
 End of Tests:& & & & & & \\
 Approval of Code:& & & & & & \\
 Integration into Version: & & & & & & \\
 Start of Tests (of new Version):& & & & & & \\
 End of Tests (of new Version):& & & & & & \\
 Approval of Change:& & & & & & \\
\hline\hline
\end{tabular}
\end{center}

\end{document}
