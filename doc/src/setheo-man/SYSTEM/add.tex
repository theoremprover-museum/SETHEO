%%%%%%%%%%%%%%%%%%%%%%%%%%%%%%%%%%%%%%%%%%%%%%%%%%%
%   SETHEO MANUAL
%	(c) J. Schumann, O. Ibens
%	TU Muenchen 1995
%
%	%W% %G%
%%%%%%%%%%%%%%%%%%%%%%%%%%%%%%%%%%%%%%%%%%%%%%%%%%%
\section{Additional Modules}

The additional modules are shell scripts which call the basic
programs. The environment variable {\tt SETHEOHOME} should indicate
the place where \plop, \inw, \wasm\ and \SAM\ reside\footnote{See
Section~\ref{sec:environ}.}. If this variable is not defined, a
default value is taken. 

\subsection{clop}

\Clop\ combines the SETHEO compiler \inw\ with the
assembler \wasm. Given a file, first \inw\
is invoked. All of the given parameters are passed to the \inw.
If the compilation phase terminated successfully, the
assembler \wasm\ is called with default parameters.
The following is the synopsis of \clop:

{\tt clop} [-$\mbox{\tt par}_1$]\ldots[-$\mbox{\tt par}_n$] {\tt file}

The input file for \clop\ has to be given. This must be a file with
extension {\bf .lop}. \Clop\ will produce a temporary file with
extension {\bf .s} and generate the output file with extension {\bf
.hex}. The temporary file is not removed automatically. The {\bf .hex}
file can be used as an input for the \SAM.

\Clop\ returns $0$ in case of successful compilation.
Otherwise the non--zero return value of the module in which
the error occurred, is returned.



\subsection{setheo}\label{sec:setheo}

\Se\ calls the whole SETHEO theorem prover to prove a formula
contained in a  file  {\it file.lop\/} using default options. \Se\ in turn
calls \inw, \wasm\ and \SAM. 
%In case, a proof is found, a proof tree is 
%generated which can be displayed using \xp. 

This is the synopsis:
\begin{verbatim}
setheo file
\end{verbatim}

The name of the input file must have the extension {\bf .lop}. During
computation temporary files with the extensions {\bf .s} and {\bf
.hex} are generated. These are not removed automatically. Two output
files are generated. The output file with the extension {\bf .tree}
can be used as an input for \xp\ to visualize a found
proof. The output file with the extension {\bf .lop} contains the
output of the prover \SAM, the same as written onto the screen.

The following options are used (for details concerning these options
see Sections~\ref{sec:inwasm}, \ref{sec:wasm} and \ref{sec:sam}): 
\begin{description}
      \item[Compiler:]
           {Constraints are generated  with {\bf --cons}.   The  output
            code is optimized with {\bf --opt}.}
      \item[Prover:]
           {Iterative deepening over the depth of  the  tableau  is
            performed   ({\bf --dr}).   Constraints  are  generated  and
            checked with {\bf --cons}.}
\end{description}

If the user wants to modify this default set of options, the
user  should  refer  to the commands \inw\ and \wasm\ for compilation
and to the \SAM\ for the prover itself.



\subsection{delta}
The {\sc Delta}-iterator is a preprocessing module which generates
unit-lemmata from the input formula in a bottom up way. These unit-lemmata
are then added to the input formula and then main top-down search for the
proof is started.
These additional unit-lemmata often abbreviate long (and difficult to find)
sub-tableaux and thus often leads to a dramatic decrease of search-time
for the overall system.

{\sc Delta} has a number of optional parameters and is invoked:

\begin{verbatim}
delta [-all] [-horn] [-nonhorn] [-noentry] [-debug] [-limit number] [-termsize number]\\
    [-query] [-nosubs] [-large] [-cputime number] [-lastlevels number] [-termdepth number]\\
    [-delta number] [-level number] [-pred predsymb arity] ... file
\end{verbatim}

{\sc Delta}
is a preprocessing tool for \Se. It generates new single-literal
clauses (facts) in a bottom-up fashion, using \Se.
A delta iteration method is used (see below). 
 The newly generated facts are appended to the original formula in the file
{\em file}.lop to obtain the output file {\em file}.out.lop.

The orginal file {\em file}.lop remains unchanged.
The number of newly generated clauses per level and the total
amount of time needed for compilation and execution of SETHEO
is printed at the end of the run.

The options for {\sc Delta} are:

\begin{description}
\item[-all]
With this option, unit-lemmata for all predicate symbols of the
formula are generated. Furthermore, it is automatically determined,
if the formula is Horn or Non-Horn (see {\bf -[non]horn}).

\item[-horn]
generate positive single-literal clauses only. (default)
\item[-nonhorn]
positive and negative facts are generated. These are
added as additional queries to
the formula unless {\bf -noentry} is given.
\item[-noentry]
all newly generated negative facts are added as such
to the formula.
\item[-level {\em number}]
the given number of iterations are performed, or the given limit
of newly generated clauses has been reached. (default: level = 1)
\item[-limit {\em number}]
generate only less or equal than 
{\em number\/}
new clauses. (default: limit = 999)
\item[-termsize {\em number}]
generate (and keep) newly generated clauses only, if the total
number of symbols in the terms of each clause is
less than or equal than {\em number\/} symbols.
Default: do not restrict the size.
\item[-termdepth {\em number}]
generate (and keep) newly generated clauses only, if the maximal
nesting level of symbols in the terms of each clause is less than or equal
{\em number\/}. Default: do not restrict the depth of the terms.
\item[-level {\em number}]
use the given number (default: 3) as the depth-bound for the
\SAM\ during the delta-iteration.
\item[-pred name arity]
this option specifies the predicates, for which
new clauses are to be generated. The arity of the predicate must be given.
If the option {\bf -all} is not set,
at least one predictate must be specified.
\item[-debug] causes
{\sc delta}
not to remove the temporary files after processing. Instead, their name
is given.
\end{description}

For {\sc Delta} the following algorithm is used:
\begin{enumerate}
\item
For each predicate given in the argument list or for all predicate
symbols, special code
is generated and added to the original formula (``special query'').
Furthermore, code for the meta-level control is generated.
\item
This formula is compiled and executed by the \SAM.

\item
The formula is entierly searched with a given low depth bound (which can be
set using {\bf -delta {\em number\/}}, default: 3).
During the run, the newly generated single-literal clauses are stored
in the Unit-lemma index.
The unit-clauses are kept only, if their term size and term depth do
not viaolate the given bounds, and as long the limit of generated lemmas
is reached.
\item
If a stopping condition is met (number of levels or number of
newly generated clauses is exceeded), the remaining clauses
are added to the original formula and written into the file
{\em file}.out.lop.
Then
{\sc Delta}
prints the execution and compilation times and exits.
\item
Otherwise, the search starts again (Step~3), whereby all unit-lemmata
generated in the previous levels can be used during the search.
\end{enumerate}

\subsection{xvtheo}
{\tt johann}
\subsection{xvdelta}
{\tt johann}
