%%%%%%%%%%%%%%%%%%%%%%%%%%%%%%%%%%%%%%%%%%%%%%%%%%%
%   SETHEO MANUAL
%	(c) J. Schumann, O. Ibens
%	TU Muenchen 1995
%
%	%W% %G%
%%%%%%%%%%%%%%%%%%%%%%%%%%%%%%%%%%%%%%%%%%%%%%%%%%%
\subsection{wasm}\label{sec:wasm}

\Wasm\ takes the output file of the \inw\ and generates
the \SAM\ assembler code which can be interpreted by the \SAM. This is
the synopsis of \wasm: 
\begin{verbatim}
wasm [-verbose][-[no]opt] file
\end{verbatim}

The input file has to be a file with extension {\bf .s}. From this, a
file with extension {\bf .hex} is generated. 

The parameters for controlling verbose mode and code optimization are
optional: 
\begin{description}
      \item[--verbose]
           {Turn on verbose mode.}
      \item[--opt]
           {Code optimization is done. The  connection  graph
            representation is optimized to use less space.}
      \item[--noopt]
           {Code optimization is switched off.}
\end{description}
