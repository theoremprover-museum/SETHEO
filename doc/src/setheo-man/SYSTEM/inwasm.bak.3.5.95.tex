%%%%%%%%%%%%%%%%%%%%%%%%%%%%%%%%%%%%%%%%%%%%%%%%%%%
%   SETHEO MANUAL
%	(c) J. Schumann, O. Ibens
%	TU Muenchen 1995
%
%	%W% %G%
%%%%%%%%%%%%%%%%%%%%%%%%%%%%%%%%%%%%%%%%%%%%%%%%%%%
\subsection{inwasm}\label{sec:inwasm}

\Inw\/ (INput for Warren ASeMbler) takes clauses in \LOP\/
notation and compiles them into \SAM\/ assembler code.  The
output  may  then  be  processed  by \wasm. As an internal
representation a weak--unification connection graph  is
generated and preprocessed. For that purpose constraints can be
computed in order to avoid redundant substitutions. A simple
reordering  of  input  clauses (short before long) and their
subgoals (according to the first fail principle) is  set  by
default.  For  the syntax of the input language \LOP, see the
description in section \ref{sec:lop}.  \Inw\/  destinguishes
between  declarative clauses (axioms) and procedural clauses
(rules). In the Non--Horn case all  declarative  clauses  are
fanned  into contrapositives. Also code to perform reduction
steps is added in this case.

\begin{description}
\item[Synopsis:]
     {\ \\
      \begin{tabular}{lp{10cm}}
      {\tt inwasm}  &  {\tt [-[no]purity] [-nofan] [-[no]reduct] 
                            [-nosgreord] [-noclreord] [-notree] 
                            [-randreord] [-eqpred] [-all] [-reg] 
                            [-subs] [-taut] [ -cons] [-foldup] 
                            [-foldupx] [-folddown] [-folddownx]
                            [-[no]partialtree] [-verbose  [number]]
                            [-lop]}  \\
      {\tt file}    &  
      \end{tabular}}
\item[Options:]
     {Switching off (or enforcing) default settings:
      \begin{description}
      \item[{\tt -[no]purity}]
           {\ \\
            Suppress or enforce purity. Purity is performed by default.}
      \item[{\tt -nofan}]
           {\ \\
            Suppress fanning of declarative clauses.}
      \item[{\tt -[no]reduct}]
           {\ \\
            Suppress or enforce code  generation  for  a  reduction 
            steps.  Reduction  steps  guarantee  competeness in the
            Non--Horn case.}
      \item[{\tt -nosgreord}]
           {\ \\
            Turns off subgoal reordering in clauses with more  than
            two literals.}
      \item[{\tt -noclreord}]
           {\ \\
            Turns off clause reordering.}
      \item[{\tt -notree}]
           {\ \\
            Writing the proof tree to a file is disabled.}
      \end{description}

      Modifying the search strategy:
      \begin{description}
      \item[{\tt -randreord}]
           {\ \\
            Insert code for random reordering of or--branches.}
      \item[{\tt -eqpred}]
           {\ \\
            Insert code for detecting 'identical--predecessor--fails'
            during the proof search.}
      \item[{\tt -all}]
           {\ \\
            With this option all possible proofs within the current
            bounds  are enumerated. For each proof \SE\/ fails and
            adds the new proof--tree to the .tree file. Before using
            \xp\/  the  file  must  be  splitted. The option is
            especially useful in combination with  write--statements
            to display answer substituitions of query variables.}
      \end{description}     

      Enabling constraints:
      \begin{description}
      \item[{\tt -reg}]
           {\ \\
            Instructions for full regularity check via  constraints
            are inserted.}
      \item[{\tt -subs}]
           {\ \\
            Instructions to activate 'unit subsumption constraints'
            are inserted.}
      \item[{\tt -taut}]
           {\ \\
            Instructions to add 'tautology constraints' are added.}
      \item[{\tt -cons}]
           {\ \\
            All kinds of constraints are computed and inserted into
            the code file.}
      \end{description}     

      New inference rules:
      \begin{description}     
      \item[{\tt -foldup}]
           {\ \\
            Insert code to factorize subgoals with previous ones.}
      \item[{\tt -foldupx}]
           {\ \\
            Insert code to use previously solved subgoals for pruning 
            purposes in an extended regularity check.}
      \item[{\tt -folddown}]
           {\ \\
            Insert code to factorize subgoals with later ones.}
      \item[{\tt -folddownx}]
           {\ \\
            Insert code to use later subgoals for pruning  purposes
            in an extended regularity check.}
      \end{description}   

      \begin{remark}
      Folding up and folding down are  not  compatible  with  each
      other.   The  extended regularity checks are not complete in
      combination with antilemmata.
      \end{remark}  

      Miscellaneous:
      \begin{description}
      \item[{\tt -[no]partialtree}]
           {\ \\
            Disables or enables access to untouched  nodes  in  the
            current tableau. The access option is set by default to
            guarantee consistent data  structures.  {\tt nopartialtree}
            is  save  iff  no  antilemmata  are  used  in the proof
            search.}
      \item[{\tt -verbose [number]}]
           {\ \\
            Achieves varying  verbosity  specified  with  {\tt number}.
            Useful only for debugging purposes.}
      \item[{\tt -lop}]
          {\ \\
            Writes preprocessing output into {\tt file\_pp.lop}.
            This is a series of \LOP\/ clauses with comments which can
            be used as an input file for \inw.}
      \end{description}}
\item[Inputfile:]
     {The name of the input file is {\tt file.lop}.}
\item[Outputfiles:]
     {The name of the output file is either {\tt file\_pp.lop} 
      (see option {\tt -lop}) or {\tt file.s} (default). 
      {\tt file\_pp.lop} is an input file for \inw\/ and {\tt file.s} is an
      input file for \wasm.}
%\item[Input language:]
%     {The input language used for \inw\/ is \LOP.}
%\item[Sources:]
%     {The sources are in directory {\tt $\sim$setheo/SOURCES/inwasm}.}
%\item[Bugs:]
%     {Take care with formulas containing a mixture of  declarative
%      and procedural clauses.}
%\item[Authors:]
%     {J.\ Schumann, J.\ Breiteneicher and others.}
\end{description}




