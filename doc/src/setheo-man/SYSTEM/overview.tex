%%%%%%%%%%%%%%%%%%%%%%%%%%%%%%%%%%%%%%%%%%%%%%%%%%%
%   SETHEO MANUAL
%	(c) J. Schumann, O. Ibens
%	TU Muenchen 1995
%
%	%W% %G%
%%%%%%%%%%%%%%%%%%%%%%%%%%%%%%%%%%%%%%%%%%%%%%%%%%%

\section{SETHEO System Overview}\label{sec:overwiew}

The SETHEO system consists of a set of (individual) programs,
as shown in Figure~\ref{fig.1}. Data are exchanged between the
modules using files. Parameters for each program are given in the
command line.

\begin{figure}[htb]
\vspace*{8cm}
\caption{Parts of the SETHEO system}
\label{fig.1}
\end{figure}

The indivual programs will be described in detail below. They are:

\begin{description}
%\item[plop]
%This program\footnote{This program is distributed as an ``as is'' version
%without support. For details see Section~\ref{sec:plop}.} converts
%formulae in first order notation (with quantifiers and standard
%(infix-) operators (e.g., $\wedge, \vee$) into a set of clauses.

\item[inwasm]
This program is the {\em compiler\/} for SETHEO. It takes a formula
(as a set of clauses) or a logic program as its input. Details of the
input language (LOP) and its syntax are described in detail in
Section~\ref{sec:lop-syntax}.
{\bf Inwasm} performs several preprocessing steps which try
to reduce the search space spanned by the formula. Typical preprocessing
steps are the generation of syntactical constraints or the elimination
of pure literals.
Then, binary code for the SETHEO Abstract Machine \SAM\ 
are generated. Optionally, {\bf inwasm} can output 
the preprocessed formula in LOP-syntax or \SAM-assembler code.
This option is helpful, if the preprocessed formula is to be
modified by hand (or some other filter), before it should be compiled
for the \SAM.
Details are discussed in Section~\ref{sec:inwasm}.

%\item[wasm]
%The assembler {\bf wasm} converts assembler instructions (as normally
%produced by {\bf inwasm} and converts them into a binary
%(currently actually hexadecimal) representation which can be directly
%read into the SETHEO Abstract Machine.
%Symbolic labels and constants are resolved. Furthermore, the
%code is somewhat optimized to reduce the size of the produced code file.
 
\item[sam]
This program is the interpreter of the SETHEO Abstract Machine
\SAM. After loading the code-file as prepared by {\bf wasm}, the
abstract machine is started with given parameters.
If a proof could be found, the given time-limit has been exceeded,
or upon request, statistical information is printed on the screen
and into a log-file. In case a proof could be found, the tableau
with instantiated literals is generated and written into a file.

\item[xptree]
This program is an X-based graphical viewer for tableaux, generated
by {\bf sam}. The tableau is displayed as a tree with literals
(and additional information) as its nodes. Scrolling, selection and
hiding of subtableaux can be accomplished using the mouse.

\item[setheo]
This script represents the top-level script of the SETHEO system.
Given a formula in LOP-notation, the compiler (inwasm) and the
prover (sam) are activated automatically, using {\em default parameters}.

\end{description}

Around these basic programs, additional modules are located which
use the basic programs. They increase the functionality of the
SETHEO system and facilitates its usage.

Not all of the modules are currently supported, nor are described in
this manual.
(Those described in the following are marked by a $\dagger$.)
If you are interested in one of the modules, please contact the
SETHEO-team.

New modules are likely to be added in further distribution versions
of SETHEO. Therefore, modules which are new for the current version of
SETHEO are marked clearly.
The modules are listed in alphabetical order.

\begin{description}
%\item[clop]
%This script combines the compiler {\bf inwasm} and the assembler
%{\bf wasm}. All command-line parameters are passed to the compiler
%{\bf inwasm}.


\item[delta]
is a preprocessor which takes a formula in LOP syntax, and produces
one in the same syntax. {\bf Delta} generates unit-lemmata in a bottom-up
way which are added to the original formula. In many cases, this combination
of bottom-up preprocessing (using {\bf delta}) and subsequent top-down
processing by the \SAM\ results in a dramatic increase of efficiency.
{\bf Delta} uses {\bf inwasm, wasm} and {\bf sam}.

\item[plop]
This program\footnote{This program is distributed as an ``as is'' version
without support. For details see Section~\ref{sec:plop}.} converts
formulae in first order notation (with quantifiers and standard
(infix-) operators (e.g., $\wedge, \vee$) into a set of clauses.
\item[rctheo]
is a parallel theorem prover, based on SETHEO \cite{Ert93}. 
It uses random competition
as the underlying model and is running on a network of workstations.
{\bf Rctheo} is available upon request.

\item[sicotheo]
is a prototypical implementation for exploiting parameter competition
with SETHEO. This implementation is based on {\bf pmake} and is available
upon request.

\item[sptheo]
exploits parallelism by Static Partinitioning. Implemented upon PVM,
it is running on networks of UNIX workstations \cite{Sut95}.

\item[wasm]
The assembler {\bf wasm} converts assembler instructions (as normally
produced by {\bf inwasm} and converts them into a binary
(currently actually hexadecimal) representation which can be directly
read into the SETHEO Abstract Machine.
Symbolic labels and constants are resolved. Furthermore, the
code is somewhat optimized to reduce the size of the produced code file.
 
\item[xvdelta]
This module is a primitve graphical user interface (GUI) for {\bf delta}
preprocessor.
It is based on XVIEW toolkit of X. {\bf Xvdelta} facilitates the selection
of parameters of {\bf delta} and to control its operation.

\item[xvtheo]
This module is a primitve graphical user interface (GUI) for SETHEO.
It is based on XVIEW toolkit of X. {\bf Xvtheo} allows to edit
a formula, to select appropriate parameters (using pull-down menues),
to start the prover and to view the results.
{\bf Xvtheo} calls all the basic progams.
\end{description}
