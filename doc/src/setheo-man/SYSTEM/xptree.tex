%%%%%%%%%%%%%%%%%%%%%%%%%%%%%%%%%%%%%%%%%%%%%%%%%%%
%   SETHEO MANUAL
%	(c) J. Schumann, O. Ibens
%	TU Muenchen 1995
%
%	%W% %G%
%%%%%%%%%%%%%%%%%%%%%%%%%%%%%%%%%%%%%%%%%%%%%%%%%%%
\subsection{xptree}\label{sec:xptree}

\Xp\ is an X--window based interactive  user  interface  for
displaying  proof  trees  generated  by \SE\ resp.\ by the \SAM. The user
can manipulate the outline of the tree  by  showing  interesting
information and hiding information, which is useless to him.
All ways of manipulation are mouse driven and so  \xp\  is easy  to
use.   The  program is based on the {\em Athena Widgets programming
library}, and can be operated with  an  arbitrary window  manager.
Predicates and functions can be displayed in prefix, infix, or postfix
notation in  order  to  enhance the  readability  of the proof tree.
The representation and binding power of these operators can be defined
by the user. The synopsis of \xp\ is the following:
\begin{verbatim}
xptree [-td number][-horiz][-vert][-ctab opfile] [file[.tree]]
\end{verbatim}

The parameters for \xp, even the filename, are optional. The name of
the tree file, which is displayed, must have the extension {\bf
.tree}. The extension can be omitted. If  the whole file  name  is
omitted,  input  is  read from standard input. The other optional
parameters are described below:
\begin{description}
      \item[--td number]
           {Set the shown term depth. Terms, cascaded  deeper  then
            the  specified  number of levels are replaced by '\dots'.
            {\it number\/} must be greater than $0$. The original  term  can
            be shown by pressing the middle mouse button (see also the
            description of the mouse buttons).}
      \item[--horiz]
           {Display the tree horizontally. Can be changed interactively.}
      \item[--vert]
           {Display the tree vertically. Can be changed interactively.}
      \item[--ctab opfile]
           {This  option  loads {\it opfile\/}  which   contains
            optional  definitions  of  infix,  prefix,  and postfix
            operators and their representation.  The format of this
            file is described in Section~\ref{sec:file-formats:opdefs}.}
\end{description}

Using the {\bf --ctab} option (see above for a decription of this
option) a file containing {\it operator definitions\/} can be loaded.
The format of the file, describing the type of the operators
(infix,  prefix,  postfix),  their  binding  power and their
print--names is as follows: Each line of  the  file  contains
one definition. Blank lines or comment lines are not permitted. 
A definition consists of four fields,  separated  by  a colon (':'):
\begin{description}
      \item[The parse--name]
           {is the name of the predicate or function symbol  as  it
            appears  in  the  tree file produced by the \SAM.  It
            starts with a lower case letter and may  contain  lower
            and  uppercase letters, digits, and the underscore '\_'.
            Blanks and other special characters are not allowed.}
      \item[The print--name]
           {is the string which is displayed whenever a  term  with
            the  given  operator is displayed. Print--names may
            contain arbitrary characters (except the colon, which  has
            to  be  preceeded  by  a back--slash).  In particular, a
            print--name can contain blanks to enhance the  readability
            of a term.}
      \item[The binding power]
           {is an integer, defining the binding power of the operator.}
      \item[The type]
           {is a character, indicating the type  of  the  operator:
            'i'  or 'I' for infix, 'p' or 'P' for prefix and 'q' or
            'Q' for postfix.}
\end{description}

Predefined operators are `$\sim$` and '$[$' as prefix operators with a
binding power of 50. All entries in the given file are processed
in reverse order.  Therefore, the definitions of '$\sim$' and '$[$'
can be changed by the user.  The following example shows the
format of the entries into the file:
\begin{center}
\begin{tabular}{ccccccc}
{\tt equal} & {\tt :} & {\tt =}    & {\tt :} & {\tt 100} & {\tt :} & {\tt i} \\
{\tt in}    & {\tt :} & {\tt in}   & {\tt }: & {\tt 100} & {\tt :} & {\tt i} \\
{\tt ispre} & {\tt :} & {\tt $[$=} & {\tt }: & {\tt 100} & {\tt :} & {\tt i} \\
{\tt pair}  & {\tt :} &            & {\tt :} &  {\tt 95} & {\tt :} & {\tt p} \\
{\tt add}   & {\tt :} & {\tt +}    & {\tt :} & {\tt 150} & {\tt :} & {\tt i} \\
{\tt mul}   & {\tt :} & {\tt *}    & {\tt :} & {\tt 200} & {\tt :} & {\tt i}
\end{tabular}
\end{center}

When the tree is displayed, the window can be  distinguished
into  three  areas:   the  menu bar, the panner and the tree
itself. In each area the user can perform different actions:
execute  commands  with  the  menu bar, move the tree in the
window and change display of nodes.  The  second  action  is
only available, if the tree is larger than the window. It is
reached by moving the mouse in the  panner  area  while  the
left button is pressed.

By using the menu bar, the following commands  may  be executed:
\begin{description}
      \item[Quit]
           {Leave \xp. Key shortcut: {\bf q}.}
      \item[View/Layout Horizontal]
           {Display the tree  with  the  root  to  the  left.   Key
            shortcut: \Cc.}
      \item[View/Layout Vertical]
           {Display the  tree  with  the  root  at  the  top.   Key
            shortcut: \Cv.}
      \item[Tree/Select All]
           {Show the additional information for every node.}
      \item[Tree/Unselect All]
           {Hide the additional information for every node.}
      \item[Tree/Select Children]
           {Show the additional information for  each  node,  which
            has a currently selected parent.}
      \item[Tree/Select Descendants]
           {Show the additional information for  each  node,  which
            has a currently selected ancestor.}
      \item[Tree/Set Term Depth]
           {Change the term depth dynamically (see also option {\bf --td}). 
            This option is not supported by now.}
      \item[About]
           {Show the current version number.}
\end{description}

The style of the tree can be changed by clicking  the  mouse
buttons  while  the  mouse pointer is on a node of the tree.
The node must be highlighted, before any  action  is
available. The actions are described below:
\begin{description}
      \item[Left Mouse Button]
           {Toggles the display of additional information for the node.}
      \item[Middle Mouse Button]
           {Toggles the display of term: brief vs.\ full display.
            The term depth option {\bf --td} must be set.}
      \item[Right Mouse Button]
           {Collapses the whole subtree of the node  and  the  node
            itself  to  a  small  triangle.  So  the  size  of  the
            displayed tree may be shrunk immensely.}
\end{description}

Any mouse button, if the node is a triangle, expands the hidden
subtree. 

