%%%%%%%%%%%%%%%%%%%%%%%%%%%%%%%%%%%%%%%%%%%%%%%%%%
%   SETHEO MANUAL
%       (c) J. Schumann, O. Ibens
%       TU Muenchen 1995
%
%       %W% %G%
%%%%%%%%%%%%%%%%%%%%%%%%%%%%%%%%%%%%%%%%%%%%%%%%%%%
\subsection{sam}\label{sec:sam}

The \SAM\ (= {\bf S}ETHEO {\bf A}bstact {\bf M}achine) is a theorem
prover for full first order logic. It is based  on  the  connection
tableau calculus and runs on many different machines. The implementation
of the \SAM\ is based on the {\em Warren Abstract Machine (WAM)}.

The synopsis of the \SAM\ is the following.

\begin{tabular}{lp{13.8cm}}
{\tt sam}   &  {\tt  [-d~number] [-i~number] [-cdd~number]
                     [-wd~number] [-wd1~number] [-wd2~number]
                     [-wd3~number] [-wd4~number] [-wd4flag]
                     [-tc~number] [-tcd~number] [-tcd1~number]
                     [-tcd2~number]
                     [-sig~number] [-sigd~number] [-sigd1~number]
                     [-sigd2~number]
                     [-fvars~number] [-sgs~number]
                     [-dr~[number]] [-ir~[number]] [-cddr~[number]]
                     [-wdr~[number]]
                     [-[no]reg] [-[no]st]
                     [-anl~[number]] [-noanl] [-hornanl~[number]]
                     [-nhornanl~[number]]
                     [-[no]cons]
                     [-dynsgreord~[number]]
                     [-singlealt~[number]] [-multialt~[number]]
                     [-spreadreducts] [-forcegr] [-forcedl]
                     [-forcefvar] [-forceps] [-[no]lookahead] [-eq]
                     [-altswitch~number] [-shortcl~number]
                     [-longcl~number]
                     [-code~number] [-stack~number] [-cstack~number]
                     [-heap~number] [-trail~number] [-symbtab~number]
                     [-seed~number]
                     [-v[erbose]~[number]]
                     [-realtime~number] [-cputime~number]
                     [-alltrees]
                     %[-lemmatree]
                     [-printlemmata]
                     [-batch] [-debug]} \\
{\tt file}  &
\end{tabular}

The filename must have the extension {\bf .hex}. During and after the
computation statistical information is generated. This
information is printed on the screen as well as in a file with the
extension {\bf .log}.
In case a proof is found, a proof tree is generated. The proof tree
file has the extension {\bf .tree} and can be displayed using \xp. 
The following parameters are optional.

The connection tableau calculus is sound and complete for full
first order logic, if the tableau branches may contain inifinitly
many nodes. To avoid an infinite search space {\it fixed
bounds\/} can be set: The depth of the tableau, the number of
inferences, the inference resources for each branch, the number
of free variables, the termcomplexity and the number of open
subgoals can be restricted. But note that completeness is not ensured
when using these restrictions\footnote{The \SAM\ distinguishes between
{\it bound failure\/} and {\it total failure\/}: {\it Bound failure\/}
means that this proof might have failed due to the chosen bounds and
{\it total failure\/} means that this problem can never
succeed.}. These are the bounds for a finite search space.  
\begin{description}
      \item[--d number]
           {\em Depth bound\/}. The (A-literal) depth of the
           tableaux to be  enumerated is less than {\it number\/} or
           equal to {\it number\/}. 
      \item[--i number]
           {\em Inference bound\/}. The maximum number of leaf nodes
           of the tableaux to  be enumerated  is  less  
           than {\it number\/} or equal to {\it number\/}.
      \item[--cdd number]
           {\em Clause dependent depth bound\/}. Something in between the
           {\bf --i} and the {\bf --d} option; in each inference with
           a clause of length~$n$, the resources for each new leaf
           node are determined by the resources of their parent
           minus~$n$. The root node starts with {\it number\/}
           inferences. 
      \item[--wd number]
           The {\em weighted depth bound\/} permits to combine the
           above bounds. The bound is computed by using four
           parameters~$\mbox{{\it wd\/}}_1, \dots, \mbox{{\it wd\/}}_4$
           described below. The maximum value of  this bound  is
           determined by {\it number\/}. 

           When entering a clause, its weighted depth is modified
           depending on the number of  subgoals  of  the  clause
           (according to~$\mbox{{\it wd\/}}_4$); this value is divided 
           into two parts, a {\em free\/} part and an {\em adaptive}
           part (according to~$\mbox{{\it wd\/}}_1, \mbox{{\it
           wd\/}}_2$); when a subgoal is selected, the adaptive
           resource depends on the inferences  performed since the
           clause has been entered (according to~$\mbox{{\it
           wd\/}}_1$). In the special case, where all
           parameters~$\mbox{{\it wd\/}}_1, \dots, \mbox{{\it
           wd\/}}_4$ are zero, the bound behaves like the depth bound.  

           The following options are for setting the
           parameters~$\mbox{{\it wd\/}}_1, \dots, \mbox{{\it
           wd\/}}_4$. Let~$n_{sgls}$ be the number of subgoals.  
           \begin{description}
           \item [--wd1 number]
                 Set~$\mbox{{\it wd\/}}_1$ to {\it number\/}.
                 The free depth is incremented by 
                 \[  
                     \frac{\mbox{{\it wd\/}}_1}
                          {100}
                 \] 
           \item [--wd2 number]
                 Set~$\mbox{{\it wd\/}}_2$ to {\it number\/}.
                 The free depth is decremented by 
                 \[  
                     \frac{n_{sgls} * \mbox{{\it wd\/}}_2}
                          {100}  
                 \]
           \item [--wd3 number]
                 Set~$\mbox{{\it wd\/}}_3$ to {\it number\/}.
                 Let~$\Delta i$ be the number of inferences performed
                 since entering the clause. The adaptive depth is
                 divided by
                 \[  
                     1 + \frac{\Delta i * \mbox{{\it wd\/}}_3}
                              {100}
                 \]
           \item [--wd4 number]
                 Set~$\mbox{{\it wd\/}}_4$ to {\it number\/}.
                 If the {\bf --wd4flag} is not set, then the weighted
                 depth is decremented  by 
                 \[  
                     1 + \frac{n_{sgls} * 
                               \mbox{{\it wd\/}}_4}
                              {100}  
                 \]
                 Otherwise the weighted depth is decremented by
                 \[  
                     \frac{100 + n_{sgls} * 
                                 \mbox{{\it wd\/}}_4}
                          {100 + \mbox{{\it wd\/}}_4}
                 \]
           \item [--wd4flag]
                 This flag is important in combination with the
                 {\bf --wd4} option (see above). 
                 Per default, the {\bf --wd4flag} is set. 
           \item [--eq]
                 This flag controls the combination of the weighted
                 depth bound with subgoal alternation and equality
                 handling by STE-modification. 
                 For details see the parameters concerning subgoal
                 alternation. 
                 Per default, {\bf --eq} is switched off. 
           \end{description}
      \item[--tc number]
           {\em Termcomplexity bound\/}. The maximum value of the sum
           of  term  complexities  of open  subgoals  is set to {\it
           number\/}. The complexity of a term is the sum over the
           complexities of the  contained subterms.  The  complexity
           of a free variable is~$0$ and the complexity of a constant
           or of a function symbol is~$1$.  Here the {\it
           dag-complexity\/}  is computed: The complexity of
           repeatedly occurring subterms is~$1$. 
      \item[--tcd number]
           {\em Depth-dependent termcomplexity bound\/}. 
           The maximum value of the sum of term complexities of open
           subgoals is set to a value depending on the permitted
           tableau depth. 

           The value of the depth-dependent termcomplexity bound~{\it
           tcd\/} is computed according to parameters~$\mbox{{\it
           tc\/}}_1, \mbox{{\it tc\/}}_2$.  Let~$d$ be the permitted
           tableau depth. 
           \[
               \mbox{{\it tcd\/}} = 
               \frac{\mbox{{\it tc\/}}_1}
                    {100} +
               \frac{d * \mbox{{\it tc\/}}_2}
                    {100}
           \] 
           
           Per default, $\mbox{{\it tc\/}}_1 = 1000$ and 
           $\mbox{{\it tc\/}}_2 = 100$.
           The following options are for setting the
           parameters~$\mbox{{\it tc\/}}_1, \mbox{{\it tc\/}}_2$.   
           \begin{description}
           \item [--tcd1 number] 
                 Set~$\mbox{{\it tc\/}}_1$ to {\it number\/}. 
           \item [--tcd2 number] 
                 Set~$\mbox{{\it tc\/}}_2$ to {\it number\/}. 
           \end{description}
      \item[--sig number]
           {\em Signature bound\/}. The maximum number of different
           signature symbols (i.e., predicate, function and constant
           symbols) is set to {\it number\/}.
      \item[--sigd number]
           {\em Depth-dependent signature bound\/}. 
           The maximum number of different signature symbols is set to
           a value depending on the permitted tableau depth. 

           The value of the depth-dependent signature bound~{\it
           sigd\/} is computed according to parameters~$\mbox{{\it
           sig\/}}_1, \mbox{{\it sig\/}}_2$.  Let~$d$ be the permitted
           tableau depth. 
           \[
               \mbox{{\it sigd\/}} = 
               \frac{\mbox{{\it sig\/}}_1}
                    {100} +
               \frac{d * \mbox{{\it sig\/}}_2}
                    {100}
           \] 

           Per default, $\mbox{{\it sig\/}}_1 = 150$ and 
           $\mbox{{\it sig\/}}_2 = 200$.
           The following options are for setting the
           parameters~$\mbox{{\it sig\/}}_1, \mbox{{\it sig\/}}_2$.   
           \begin{description}
           \item [--sigd1 number] 
                 Set~$\mbox{{\it sig\/}}_1$ to {\it number\/}. 
           \item [--sigd2 number] 
                 Set~$\mbox{{\it sig\/}}_2$ to {\it number\/}. 
           \end{description}
      \item[--fvars number]
           {\em Variable bound\/}. Only {\it number\/} free variables
           are allowed at the same 
           time. This option is not complete in combination with
           antilemmata constraints (see below).
      \item[--sgs number]
           {\em Subgoals bound\/}. Only {\it number\/} open subgoals
           are allowed at the same time. 
\end{description}

A finite search space without loosing completeness is provided by the
following {\it iterative bounds\/}: 
An initial bound is increased until either a proof has been found or
the problem has failed totally. 
Iterative search can be done on the (weighted) tableau depth, on the
number of inferences or on the inference resources of each branch. 
These are the options for iterative search.
\begin{description}
      \item[--dr \lb number\rb]
           {Use iterative-deepening search with depth  bounds.  The
            depth bound  increment  after  each  cycle  is given by
            {\it number\/}. The default increment is~$1$ and the start
            depth bound is~$2$.}
      \item[--ir \lb number\rb]
           {Use iterative-deepening search with  inference  bounds.
            The  bound  increment  after  each  cycle  is  given by
            {\it number\/}. The default increment is~$3$ and the start
            inference bound is~$5$.}
      \item[--cddr \lb number\rb]
           {Use iterative-deepening search with clause dependent
            depth bounds. The bound increment after each cycle is given 
            by {\it number\/}. The default increment is~$1$ and the start
            clause dependent depth bound is~$5$.}
      \item[--wdr \lb number\rb]
           Use  iterative-deepening  search  with  weighted  depth
           bounds.  The depth bound  increment after each cycle is
           given by {\it number\/}. The default increment is~$1$  and  the
           start  depth  bound  is~$2$. See also the parameters for
           weighted depth bound search above.
\end{description}

\begin{remark}
From the recursive bounds {\bf --dr}, {\bf --ir},  {\bf --cddr}, {\bf --wdr}  
only  one option  can  be  selected.  For the fixed bounds we can
have as many as we want. Recursive and fixed bounds can be combined.
\end{remark}

Furthermore, the search space can be restricted by the use of
{\it constraints\/}. The constraint choices are the following.
\begin{description}
      \item[--reg] 
           {Use {\it regularity constraints\/} in order to  avoid  multiple
            occurrences of the same literal in a tableau path.}
      \item[--noreg] 
           {Deactivate regularity constraints.}
      \item[--st] 
           {Activate {\it subsumption\/} and {\it tautology constraints\/}.}
      \item[--nost] 
           {Deactivate subsumption and tautology constraints.}
      \item[--anl \lb number\rb] 
           {Use {\it antilemma constraints\/} in order to  avoid  repetitive
            solutions of the same subgoals. {\it number\/} characterizes
            the minimum search that must have been explored to generate an
            antilemma. The default is {\it number\/} = 25.}
      \item[--noanl] 
           {Deactivate generation of antilemma constraints.}
      \item[--hornanl \lb number\rb] 
           {Provides an optimized version of antilemmata which
            generates antilemma constraints only for {\it relevant\/}
            variables. This option is not complete.}
            % in combination with
            % regularity, subsumption or tautology constraints and not
            % complete for {\it nonhorn clauses\/}.}  
      \item[--nhornanl \lb number\rb] 
           {Similar to {\bf --hornanl}. This option is complete, but
            the pruning effect is less than by using {\bf --hornanl}.} 
      \item[--cons] 
           {Use antilemma, regularity, subsumption and tautology
            constraints.} 
      \item[--nocons] 
           {Deactivates antilemma, regularity, subsumption and
            tautology constraints.} 
\end{description}

\begin{remark}\label{rem:constr-options}
\
\begin{enumerate}
      \item{If you did not select  the
            corresponding  options  in the preprocessing phase of
            \inw\ the options {\bf --reg}, {\bf --st} have no effect and
            the {\bf --cons} option will only turn on antilemma
            constraints. See also Section~\ref{sec:inwasm}.} 
      \item{The {\bf --cons} and {\bf --no...} options can be combined
            to switch on all constraints except for one. For example
            {\bf --cons --noanl} will activate all constraints exept for
            the antilemma constraints. But do not turn around the order
            of these options! \linebreak
            {\bf --noanl --cons} means: {\it First\/}
            switch off the antilemma constraints (even if they are not
            yet activated) and {\it then\/} switch on all constraints
            (including the antilemma constraints).}
\end{enumerate}
\end{remark}

In finding proofs it is useful first to call the subgoals which are
difficult to prove and try the easy ones later. The reason is to
detect as early as possible, if a clause will lead to a fail. That is
why {\it subgoal reordering\/} is a helpful method. A 
{\it static\/} subgoal reordering can already be done during the
preprocessing phase of \inw\ (see option {\bf --sgreord} in
Section~\ref{sec:inwasm}). But during computation subgoals get
instantiated and the initial order might not be optimal any longer. So
a {\it dynamical\/} subgoal reordering becomes useful.

\begin{description}
\item[--dynsgreord \lb number\rb]
      If the clause is declarative, a subgoal is chosen with respect
      to the strategy indicated by {\it number\/}. If the clause is
      procedural, no dynamical subgoal reordering is performed,
      i.e., the first open subgoal is taken (see also
      Remark~\ref{rem:sgreord}).   
      The default value for {\it number\/} is 1.
      \begin{description}
      \item[number = 1]
           {The subgoal with the highest dag-complexity is chosen
            (The dag-complexity of a subgoal is the sum of the
            dag-complexities of its arguments.). If there is more
            than one subgoal with highest dag-complexity, the first of
            them is taken.}
      \item[number = 2] 
           {If there are ground subgoals, only ground subgoals are
            considered in choosing. The subgoal with the highest
            dag-complexity is chosen. If there is more than one
            subgoal with highest dag-complexity, the one of them with
            the least free variables is chosen.}
      \item[number = 3] 
           {The subgoal with the least free variables is chosen. If
            there is more than one subgoal with least free variables,
            the one of them with the highest dag-complexity is chosen.}
      \item[number = 4]
           {Like {\bf --dynsgreord 1}, but do not switch to a subgoal
            with a different predicate symbol, according to the order
            of subgoals computed statically.}
      \item[number = 5]
           {Like {\bf --dynsgreord 2}, but do not switch to a subgoal
            with a different predicate symbol, according to the order
            of subgoals computed statically.}
      \item[number = 6]
           {Like {\bf --dynsgreord 3}, but do not switch to a subgoal
            with a different predicate symbol, according to the order
            of subgoals computed statically.}
      \end{description}
\end{description}

One common principle of standard backtracking search procedures is
that the possible inference steps at a subgoal (the {\em choice
point}) cannot be partially processed.
More precisely, when backtracking to a choice point, another
alternative in the {\em same\/} choice point must be tried. 
This methodology has search-theoretic weaknesses, as demonstrated in
\cite{IL96}.
With {\em subgoal alternation\/} a subgoal can be delayed, even if its
choice point is not completely processed, and another subgoal may be
selected.
For procedural clauses, subgoal alternation is suppressed.
The following options can be used to control subgoal alternation.

\begin{description}
      \item [--singlealt \lb number\rb]
            Invoke the {\em single-alternation\/} variant of subgoal
            alternation, i.e., move away from a subgoal at most once,
            namely after the reduction steps and the extension steps
            with unit clauses have been tried.
            The move is suppressed if the remaining depth resources
            are less than {\it number\/} (default: {\it number = 0}).
      \item [--multialt \lb number\rb]
            Invoke the {\em multi-alternation\/} variant of subgoal
            alternation, i.e., a subgoal may be left whenever all its
            extension steps with clauses of a certain length have been
            processed. 
            Leaving a subgoal is suppressed if the remaining depth
            resources are less than {\it number\/} (default: {\it
            number = 0}). 
      \item [--spreadreducts] 
            Per default, when selection another subgoal, the one is
            selected which allows an inference step with adding the
            least subgoals to the tableau.
            Obviously, the number of added subgoals is zero for
            reduction steps and extension steps with unit clauses and
            $n\/$ for extension steps using clauses with $n\/$
            subgoals. 
            If {\bf --spreadreducts} is switched on, not zero is
            associated to reduction steps but the least $n\/$
            available in the same choice point.
            Per default, {\bf --spreadreducts} is switched off.
      \item [--forcegr]
            Suppress subgoal alternation if the current subgoal is
            ground (default: switched off). 
      \item [--forcedl]
            Extend subgoal alternation from the use after backtracking
            to the use after successfull unifications, i.e., the
            length of the available alternatives is used as an
            additional criterion for subgoal selection (default:
            switched off). 
      \item [--forcefvar]
            Switch to a subgoal which has at least one variable in
            common with the current subgoal (default: switched off). 
      \item [--forceps]
            Switch to a subgoal which has the same predicate symbol as
            the current subgoal (default: switched off).
\end{description}

\begin{remark}\label{rem:sgreord}
If you are running the \SAM\ on a file with the extension {\bf \_pp}
(generated by inwasm with the {\bf --lop} option), dynamic subgoal
selection and subgoal alternation have no effect. 
This is because a file with the extension {\bf \_pp} contains only
procedural clauses. 
See also option {\bf --lop} in Section~\ref{sec:inwasm}.
\end{remark}

\begin{description}
      \item [--altswitch number]
            This option allows to switch dynamically between two
            parameter selections for subgoal alternation. The possible
            parameter selections are the following: 
            \begin{description}
              \item [{\rm 0.}] no subgoal alternation
              \item [{\rm 1.}] {\bf --singledelay}
              \item [{\rm 2.}] {\bf --singledelay 2}
              \item [{\rm 3.}] {\bf --singledelay --forcedl}
              \item [{\rm 4.}] {\bf --singledelay --spreadreducts}
              \item [{\rm 5.}] {\bf --multidelay}
              \item [{\rm 6.}] {\bf --multidelay 2}
              \item [{\rm 7.}] {\bf --multidelay --forcedl}
              \item [{\rm 8.}] {\bf --multidelay --spreadreducts}
            \end{description}
            The switch is performed according to the length of the
            current clause: clauses of length $\leq$ {\it number\/}
            are viewed as short clauses, clauses of length $>$ {\it
            number\/} are viewed as long clauses. Per default, for
            short clauses the parameter selection~2 and for long
            clauses the parameter selection~7 are used.
            In order to change the these default setting, use the
            following commands:
            \begin{description}
              \item [--shortcl number]
                    For short clauses the parameter selection
                    indicated by {\it number} is used.
              \item [--longcl number]
                    For long clauses the parameter selection
                    indicated by {\it number} is used.
            \end{description}
\end{description}

\begin{remark}
  Of course the {\bf --altswitch} option can be combined
  with any of the above options for controlling subgoal
  alternation, e.g., the {\bf --forcegr} option. But note
  that in this case the additional control option refers to
  both short clauses and long clauses.
\end{remark}

\begin{description}
      \item [--\lb no\rb lookahead]
            Subgoal alternation leads to processing several choice
            points simultaneously.
            This offers the possibility of computing the minimal
            number of subgoals that still will be added to the
            tableau and -- since at each subgoal at least one
            inference step has to be aplied -- the minimal number if
            inferences still needed for closing the tableau.
            This {\em look-ahead information\/} is used for search
            pruning when using the inference bound, the clause
            dependent depth bound, or a weighted-depth bound.
            Per default, {\bf --lookahead} is switched on.
            Use {\bf --nolookahead} to suppress the use of look-ahead
            information.  
      \item [--eq]
            For using look-ahead information in combination with the
            weighted-depth bound two different strategies have been
            developed. 
            One of these is especially suitable when performing
            equality handling via STE-modification, and the other when
            performing axiomatic equality handling.
            Use {\bf --eq} for equality handling via STE-modification.
            Per default, {\bf --eq} is switched off. 
\end{description}

\begin{remark}
  Subgoal alternation expects the alternatives in a choice point to be
  ordered by static clause selection
  So do not invoke subgoal alternation after having compiled an input
  file with \inw\ {\bf --noclreord}.
  This would lead to undesirable swichting orders and, in combination
  with the {\bf --lookahead} option, to delaying proofs to higher
  iterative deepening levels.
\end{remark}

The \SAM\ might stop with an {\it overflow error\/}. This means
that the default sizes of the \SAM's memory areas have been too
small. You can enlarge the allocated memory with the following
options. 

\begin{description}
      \item[--code number] 
           {Set the maximal size of the {\it code area\/}\footnote{The
            size of the code area is different to the size of the
            actual formula.} (default = 100 KBytes). The code area is
            the place to store the compiled 
            input formula, i.e., the compiler output as generated by
            \wasm.}
      \item[--stack number] 
           {Set the maximal  size  of  the  {\it stack\/}  (default = 100
            KBytes). The stack is the main working space of the
            \SAM. For example, choicepoints with open
            alternatives are stored on the stack.}
      \item[--cstack number] 
           {Set the maximal size of the {\it constraint stack\/} 
            (default = 100 KBytes). Constraints are not stored on the
            stack. They have their own memory area, the constraint
            stack.} 
      \item[--heap number] 
           {Set the maximal size of the {\it heap\/} (default = 300
            KBytes). Things in the code area can not be changed during
            computation. So non-ground terms have to be copied on the
            heap before they can be instantiated.}
      \item[--trail number] 
           {Set the maximal size of the {\it trail\/} (default = 200
            KBytes). On the trail all changes are written that must be
            undone in case of backtracking.} 
      \item[--symbtab number] 
           {Set the maximal size of the {\it symbol table\/} (default = 1000
            entries). The symbol table contains all pedicate, function
            and constant symbols. Increasing the size of the
            symbol table means increasing the number of allowed
            symbols.} 
\end{description}

\begin{remark}
The size of the code area and the size of the symbol table are set
automatically to values appropriate for the given formula. Therefore,
these options should be given in specific (e.g.\ debugging) situations
only. 
\end{remark}

These are miscellaneous options.
\begin{description}
      \item[--seed number] 
           {The  random  number  generator  is   initialized   with
            {\it number\/}.  The random number generator is used, if  we
            have  set  the compiler option \inw\ {\it --randreord\/} to
            select the clauses at each orbranch in an arbirtrary
            permutation.} 
      \item[--v\lb erbose\rb \lb number\rb] 
           {Turn  on  verbose  mode  (for  debugging).    Different
            options for display are bitwise OR-ed.}
      \item[--realtime number] 
           {This option limits the run of the \SAM\ abstract machine to
                at most {\rm number\/} seconds (wall clock). If after that
                time no proof has been found, the \SAM\ is aborted with
                {\tt REALTIME FAILURE}. Please note that the accuracy of the
                alarm clock is 1 second.}
      \item[--cputime number] 
           {If this option is used, the \SAM\ is aborted (with
                {\tt CPUTIME FAILURE}) after the \SAM\ has consumed
                {\em number\/} seconds CPU (user-) time. Due to
                the timer's accuracy of 1 second, the absolute run-time
                may differ up to 1 second from {\em number}.}
      \item[--alltrees]
           Per default, only the first generated proof-tree is written
           into the {\bf .tree} file. This option causes {\it all\/}
           generated proof-trees to be written into the {\bf .tree}
           file. 
%      \item[--lemmatree] 
%           {This option causes proof-trees for all generated unit-lemmata
%                to be added into the file {\bf file.tree}. This
%                option is experimentally for the current version.}
      \item[--printlemmata]
           If the {\bf --printlemmata} option is set, all generated
           unit lemmata are put out. The output is either written on the
           screen or into an output file which has to be specified by
           using the built-in \$tell/1 (see Section~\ref{sec:tell}).
      \item[--batch] 
           {This option disables the interactive monitor mode after
                \Cc\ is pressed. This option should be used, if
                SETHEO is used within a script or a system. The \SAM\ is
                aborted with the message {\tt INTERRUPT FAILURE}.}
      \item[--debug number] 
           This option can be used to obtain debug information. {\it
           number\/} is used to control the generation of debug
           information. The default value for {\it number\/} is zero
           which means that no debug information is given.
\end{description}

During the proof, the \SAM\ can be interrupted by pressing \Cc. Then
after entering {\bf help} or \qm\ the following commands can be
entered. 
\begin{description}
\item[stat]{Display statistics.}
\item[r\lb eg\rb]{Display registers.}
\item[stack]{Display stack.}
\item[trail]{Display trail.}
\item[heap]{Display heap.}
\item[tree]{Display proof-tree.}
\item[links]{Display links.}
\item[choice]{Display current choices.}
\item[todo]{Display what is to be done.}
\item[cont]{Continue.}
\item[re]{Reproof.}
\item[help]{Help.}
\item[?]{Help.}
\item[!]{Shell escape.}
\item[quit]{Quit SETHEO.}
\end{description}
