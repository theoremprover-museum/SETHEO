%%%%%%%%%%%%%%%%%%%%%%%%%%%%%%%%%%%%%%%%%%%%%%%%%%%
%   SETHEO MANUAL
%	(c) J. Schumann, O. Ibens
%	TU Muenchen 1995
%
%	%W% %G%
%%%%%%%%%%%%%%%%%%%%%%%%%%%%%%%%%%%%%%%%%%%%%%%%%%%

\section{Assembly and Display of Answer Substitutions}

{\tt ueberarbeiten, Johann}


   To obtain the full answer substitution the write statement 
         must be the last subgoal of a QUERY (rather than a GOAL). 
	 This is because goal clauses are fanned.

	 In order to output disjunctive answer substitutions 
	 global variables are needed:

Example:

   Given the clause

\begin{verbatim}
        p(a) ; p(b) <-.
\end{verbatim}

   the `disjunctive answer' for the query

\begin{verbatim}
	<- p(X).
\end{verbatim}

   is {\tt X = a ; X = b}.


  A program which computes this answer needs a new `artificial query' 
  containing a subgoal to enter the original goal. In this way, the 
  goal is transformed into an axiom. The axiom however, is fanned into
  contrapositives. The program is given below.

\begin{verbatim}
# In order to initialize list $X:

   ?- $X := [], query, write_list($X).
   
# In order to collect the answer substitutions in the list $X:

   query <- $X := [X|$X], p(X).

# The disjunctive fact remains unchanged:

   p(a) ; p(b) <-.

# In order to output the elements of $X:

   write_list([]) :- write(".").
   write_list([H|T]) :- write(H), write(";"), write_list(T).
   
\end{verbatim}

Note: It would be helpful if you try this program by your own with SETHEO 
      and then have a closer look onto the proof tree using `xptree'.



