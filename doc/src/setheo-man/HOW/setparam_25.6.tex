%%%%%%%%%%%%%%%%%%%%%%%%%%%%%%%%%%%%%%%%%%%%%%%%%%%
%   SETHEO MANUAL
%	(c) J. Schumann, O. Ibens
%	TU Muenchen 1995
%
%	%W% %G%
%%%%%%%%%%%%%%%%%%%%%%%%%%%%%%%%%%%%%%%%%%%%%%%%%%%

\section{Set Parameters}

In this section some hints are given how to set SETHEO's parameters in
case of special problems. But these are just rules of thumb, sometimes
you might find better parameter settings by yourself. For a detailed
parameter description see Section~\ref{sec:basic-progr}.

\begin{description}
\item[Logic Programming:]
     {In the compilation phase the {\it subgoal reordering\/},
      the {\it clause reordering\/} and the generation of {\it
      overlap constraints\/} should be switched off. 
      %During the run of
      %the prover the antilemma constraints 
      %should be restricted to {\it weak antilemmata\/}. This
      %corresponds more to the view of a logic programmer.\\
      So call the \inw\ with the {\bf --nosgreord --noclreord}
      options and make sure not to call {\bf --oconsx} ({\bf
      --ocons} is possible) 
      %Call the \sam\ with the {\bf --weakanl
      %[n]} option (where {\bf n} is an optional  
      %number) instead of {\bf anl [n]}. If you run the \sam\ with
      %the option {\bf --cons} (which will cause the \sam\ to use its
      %whole constaint mechanism) just add the option {\bf --weakanl
      %[n]}
      }.
\item[Small Formulae:]
     {Small formulae do not cause any problems. Call the \inw\ with
      the {\bf --cons} option, the \wasm\ with the {\bf -opt}
      option and the \sam\ with the {\bf --dr --cons} options. These
      are good standard parameters.} 
\item[Large Formulae:]
     {When using large formulae the preprocessing is very time
      consuming. The 
      preprocessing can be fastened, if {\it overlap constraints\/},
      {\it subsumption constraints\/} and {\it link subsumption\/} are
      avoided. If there are only a few rules, but many facts, the \inw\
      can be called with the {\bf --rsubs~[n] --rlinksubs~[m]}
      options because this options do not consider facts. Otherwise
      the options {\bf --[r]subs~[n]} and {\bf 
      --[r]linksubs~[m]} should be avoided. The option {\bf
      --overlap[x]~[n]} should generally not be used for proving large
      formulae.} 
\item[Propositional Logic:]
     {}
\item[Long Proofs:]
     {During long proofs it is important to reduce the search space.
      There are several bounds to reduce the search space\footnote{See
      Section~\ref{sec:sam}.}. In general you do not know the size of
      the proof, so an iterative bound might be best.\\
      First try the combination of {\it iterative deepening\/} and
      {\it constraints\/}: Call the \inw\ with the {\bf --cons}
      option, the \wasm\ with the {\bf --opt} option and the \sam\
      with the {\bf --dr --cons} options. Iterative deepening
      is in most cases the best iterative bound, and 
      constraints additionally reduce the search space.\\
      If you can estimate the number of inferences, the number of free
      variables, the termcomplexity or the number of open subgoals you
      can use this knowledge by setting the corresponding bounds. Note
      that the free variables bound is not complete in combination
      with antilemmata.\\ 
      There are two more iterative bounds: The {\it iterative
      inferencing\/} and the {\it iterative local
      inferencing\/}. So you can also call the \SAM\
      with the {\bf --ir} resp.\ the {\bf --locir} option. Note that
      only one iterative bound is possible at the same time. For using
      the {\it iterative inferencing\/}, it is very useful to estimate
      the depth of the proof and call the \SAM\ with {\bf --ir 1 --d n 
      --cons} where {\it n\/} is the estimated depth.\\
      During large proofs the \sam\ might stop with an {\it overflow
      error\/}. In such a case you have to start the \SAM\ again with
      the overflowed memory area enlarged. Read Section~\ref{sec:sam}
      for the enlarging options.\\
      The most frequent overflow error is the {\it constraint
      stack overflow\/} error. In this case you can either reduce
      the number of 
      generated antilemma constraints\footnote{There is no possibility
      to reduce the other kinds of constraints, but its usually the
      antilemma constraints that cause a constraint stack
      overflow.}.
      The option {\bf -anl [n]} reduces the generated
      antilemmata to such for which at least {\it n\/} inference steps
      have been made. The default is {\it n\/}=25, so if you want to
      reduce the number of generated antilemma constraints, set {\it
      n\/}$>$25. This should decrease the runtime, but will increase
      the number of inferences.\\
      In some cases the {\it foldup\/} mechanism leads to good
      results. This mechanism generates lemmata from closed
      branches. Call the \inw\ with the {\bf --cons --foldup} options 
      and the \wasm\ and the \sam\ as above. But note that all methods
      to reduce the search space will reduce the number of generated
      lemmata, too. So don't be surprised if the combination of such
      two powerful methods as foldup and antilemmata doesn't lead to
      the expected results. In a few cases the combination works even 
      worse than antilemmata or foldup at its own. When using foldup 
      it might eventually be better to call the \sam\ with the {\bf
      --noanl} option, just try.}  
\end{description}


