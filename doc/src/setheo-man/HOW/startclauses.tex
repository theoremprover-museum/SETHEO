%%%%%%%%%%%%%%%%%%%%%%%%%%%%%%%%%%%%%%%%%%%%%%%%%%%
%   SETHEO MANUAL
%	(c) J. Schumann, O. Ibens
%	TU Muenchen 1995
%
%	%W% %G%
%%%%%%%%%%%%%%%%%%%%%%%%%%%%%%%%%%%%%%%%%%%%%%%%%%%
\section{Selection of Start Clauses}
\label{sec:startclauses}

Per default, SETHEO takes all clauses which contain negative
literals only, as possible start clauses.
In LOP syntax, such clauses have the following form:

\[ \mbox{\tt <-} L_1, \ldots, L_n. \]

with atoms (non-negated literals) $L_i$.
These clauses are also referred to as {\em queries}.

The Model Elimination Calculus, however, allows to select
an {\em arbitrary\/} clause as start clause.
This can be of interest if the conjecture to be proven is of the form
$A_1 \wedge \ldots \wedge A_n$.
If a true goal-oriented search is intented,
the search should start with this clause.
On the other hand, axioms like {\tt <-equal(X,succ(X)).} should in general
not be used as a starting clause.

Cases like this can be handled by SETHEO.
In the following, we describe a way to (a) disable a query, and to
(b) convert an arbitrary clause into a query.

\subsection{Disabling a Query}
A query (as shown above) can be disabled by replacing that clause
with the following one (for any $1 \leq i \leq n$):

\[ ^\sim L_i \mbox{\tt <-} L_1, \ldots, L_{i-1},L_{i+1},\ldots,L_n. \]

This rule cannot be used as a starting clause by SETHEO. Nevertheless,
all necessary contrapositives are being generated.

Please note that disabling starting clauses may lead to incompleteness.

\subsection{Converting a Clause into a Query}
Given a clause
\[ H_1 ; \ldots ; H_m \mbox{\tt <-} L_1, \ldots,L_n. \]
with atoms (non-negated literals) $L_i,H_i$.
Then this clause can be converted into a query (starting clause)
by replacing this clause by:
\[ \mbox{\tt <-}
^\sim H_1, \ldots , ^\sim H_m, L_1, \ldots,L_n. \]

For this clause, all contrapositives are generated.
